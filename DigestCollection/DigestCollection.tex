\documentclass[
12pt,
oneside,
onehalfspacing,
headsepline
]{DigestCollection}
\usepackage[titles]{tocloft}
\usepackage{etoolbox}
\usepackage{xcolor}
\usepackage{caption}
\usepackage{subcaption}
\usepackage{tabularx}
\usepackage{cprotect}
\usepackage{amsmath}
\usepackage[export]{adjustbox}
\usepackage{textcomp}
\usepackage{float}
\usepackage{tabularray}
\usepackage{ragged2e}
\usepackage[style=english]{csquotes}
\usepackage{pdfpages}
\usepackage{todonotes}
\usepackage{listings}
\usepackage{makecell}
\DeclareCaptionType{equ}[Equation][]
\AtBeginDocument{
  \renewcommand{\contentsname}{Contents}
  \renewcommand{\bibname}{References}
  \renewcommand{\figurename}{Figure}
  \renewcommand{\tablename}{Table}
  \renewcommand{\chaptername}{Chapter}
  \renewcommand{\listfigurename}{List of Figures}
  \renewcommand{\listtablename}{List of Tables}
  \renewcommand{\thechapter}{\Roman{chapter}}
}

\geometry{
paper=a4paper,
inner=2cm,
outer=2cm,
top=1cm,
headheight=1.5cm,
bottom=1cm,
}

\setlength{\marginparwidth }{2.5cm}
\thesistitle{Collection Name}
\supervisor{Professor Name}
\examiner{}
\author{Student Name}
\keywords{}
\university{Jose Rizal University}
\AtBeginDocument{
  \hypersetup{pdftitle=\ttitle}
  \hypersetup{pdfauthor=\authorname}
  \hypersetup{pdfkeywords=\keywordnames}
}

\makeatletter
  \def\@makechapterhead#1{%
  {\parindent \z@ \raggedright \small
    \ifnum \c@secnumdepth >\m@ne
        \Large\bfseries \@chapapp\space \thechapter
        \par\nobreak
    \fi
    \interlinepenalty\@M
    \Large \bfseries #1\par\nobreak
    \vskip 30\p@
  }}
\makeatother

\begin{document}

\frontmatter
\pagestyle{plain}
\begin{titlepage}
\begin{center}
\vspace*{.05\textheight}
{\LARGE \univname}\\[0.5cm]
\includegraphics[width=0.15\textwidth]{jrulogo.jpg}\\[0.5cm]
{\normalsize Subject | Deliverable Name}\\[0.5cm]
\HRule\\[0.4cm]{\huge \bfseries \ttitle\par}\vspace{0.4cm}\HRule\\[1.5cm]
\begin{minipage}[t]{0.4\textwidth}
\begin{flushleft}\emph{Student:}\\\authorname\end{flushleft}
\end{minipage}
\begin{minipage}[t]{0.4\textwidth}
\begin{flushright}\emph{Professor:}\\\supname\end{flushright}
\end{minipage}\\[3cm]
\vfill
Class M2,  \the\year{}
\vfill
\end{center}
\end{titlepage}
\cleardoublepage

\setlength{\cftchapnumwidth}{16mm} 
\setlength{\cftsecnumwidth}{16mm} 
\setlength{\cftsecindent}{0cm}
\renewcommand{\cftsecfont}{\small}
\tableofcontents
\cleardoublepage
\mainmatter


\pagestyle{thesis}

\chapter{RULE 7. FORMAL REQUIREMENTS OF PLEADINGS }

\ssection{GR175512 May 30, 2011}

\noindent\textbf{Vallacar Transit, INC., Petitioner, v. \\Jocelyn Catubig, Respondent. LEONARDO-DE CASTRO, J.}\vspace{0.4cm}

On January 27, 1994, a collision occurred between the bus driven by Cabanilla and a motorcycle carrying Quintin Catubig. Catubig died on the spot and Cabanilla was charged with reckless imprudence resulting in double homicide but was later acquitted after a preliminary investigation. The MCTC issued a resolution dismissing the criminal charge against Cabanilla.

Jocelyn, the widow of Catubig, subsequently filed a complaint for damages against the Vallacar Transit before the RTC, alleging negligence on the part of their employee - Cabanilla. As a special and affirmative defense, Vallacar Transit asked for the dismissal of respondent’s complaint for not being verified and/or for failure to state a cause of action. Regional Trial Court dismissed the complaint for damages. Finding Catubig solely responsible for the collision. The case reached the Court of Appeals, which in turn modified the RTC decision, holding both Catubig and Cabanilla negligent. Damages are awarded to the heirs of Catubig. This prompted Vallacar Transit to reach the SUpreme Court and assail the CA decision under Rule 45.

\paragraph{Issue:}

Should the respondent's complaint be dismissed for lack of proper verification? Can the petitioner should be held liable for damages under Article 2180 of the Civil Code?

\paragraph{Decision:}

The Supreme Court found no procedural defect in respondent's complaint for damages.

As to the liability of the petitioner, the Supreme Court held that it cannot be held liable under Article 2180 because the negligence of its employee driver, Cabanilla, was not proven. The Court concluded that Catubig's negligence was the proximate cause of the collision, absolving petitioner from liability. The Court reversed the decision of the Court of Appeals and reinstated the RTC decision dismissing the complaint for damages against the petitioner.

As for the respondent's complaint lacking proper verification. The Court expounded that verification is a formal and not a jurisdictional requirement. Mainly intended to secure an assurance that matters which are alleged are done in good faith or are true and correct and not of mere speculation. If necessary the court may order the correction of unverified pleadings or act on it and waive strict compliance with the rules. No procedural defect was found in the respondents complaint.

\ssection{GR214526 November 03, 2020}

\noindent\textbf{Heirs of Mampos, Petitioner, v. \\Josefina Mampo Morada, Respondent. CAGUIOA, J.}\vspace{0.4cm}

This is a dispute over the recovery of possession of five parcels of land in Camarines Sur between the Heirs of Mampos and Josefina Morada. The heirs of Mampo filed a complaint before the Provincial Agrarian Reform Adjudicator(PARAD) against Nelida and Alex Severo for recovery of possession of the subject lots. Their complaint was dismissed, and then appealed to the Department of Agrarian Reform Adjudication Board (DARAB), which ruled in their favor and ordered Morada vacate the subject landholdings. A writ of execution was issued by the PARAD upon motion of the Heirs of Mampos.

Josefina Mampo Morada filed a third-party claim, which was granted by the PARAD, ordering parties to respect Morada's possession and cultivation and recalling the writ of execution. The Heirs of Mampos filed motions for reconsideration and for the implementation of the DARAB decision. DARAB initially dismissed the motion for lack of jurisdiction but later granted the motion for reconsideration and ordered the revival of the writ of execution.

This prompted Morada to file a petition for certiorari (Rule 65 action) and a petition for review (Rule 43 action) before the CA. The CA granted Morada's petition in the Rule 43 action, nullifying the DARAB resolution and affirming the PARAD order granting Morada's third-party claim. The CA ruled that the third-party claim was valid under Rule 39, Section 16 of the Rules and that the PARAD order became final and executory as it was not appealed by the petitioners. However, the CA's decision did not address the matter of forum shopping raised by both parties.

\paragraph{Issue:}

Whether or not the Court of Appeals (CA) erred in nullifying the resolution of the Department of Agrarian Reform Adjudication Board (DARAB) dated September 19, 2011, and reinstating the Order of the Provincial Agrarian Reform Adjudicator (PARAD) dated February 26, 2010, despite the violation against the rule on forum shopping.

\paragraph{Decision:}

The Supreme Court found merit in the petition and reversed the decision of the Court of Appeals. Morada was found guilty of forum shopping by committing two distinct acts:

Willfully and deliberately instituting two actions in two different divisions of the CA to avail of remedies founded on similar facts. 2. Submitting false certifications of non-forum shopping and not observing the undertakings mandated by Rule 7, Section 5 of the Rules.

The respondent and her counsel to show cause within ten (10) days why they should not be cited for direct contempt for committing willful and deliberate forum shopping. The Court ruled that both the Rule 65 and Rule 43 actions were dismissible due to forum shopping.

\chapter{RULE 8. ALLEGATIONS IN PLEADINGS }

\ssection{GR87434 August 5, 1992}

\noindent\textbf{Philippine American General Insurance, petitioners, v. \\Sweet Lines Inc. et al, Respondents. REGALADO, J.}\vspace{0.4cm}

On March 1977 a ship with a cargo consisting of Low-Density Polyethylene bags sailed from US, to Manila. Upon arrival in Manila, the cargoes were transshipped to Davao City using the vessel M/V "Sweet Love," owned by Sweet Lines, Inc. Upon delivery to the consignee, it was discovered that a significant number of bags were missing, damaged, or contaminated. A survey conducted on July 8, 1977, showed the extent of the loss and damage.

A maritime suit was initiated on May 12, 1978, by Philamgen and Tagum Plastics against Sweet Lines Inc along with other co-defendants. The trial court rendered judgment in favor of the petitioners against Sweet Lines, Inc. and Davao Veterans Arrastre and Port Services, Inc. However, the decision was reversed on appeal by the Court of Appeals on the ground of prescription, effectively dismissing the complaint of the petitioners.

Petitioners then filed for a review on certiorari before the Supreme Court, challenging the appellate court's decision.

\paragraph{Issue:}


The issue revolves around the timeliness of filing a claim for loss or damage to cargo against the carrier.

\paragraph{Decision:}

The Court recognized the principle that parties to a contract of shipment can agree on a limitation of time shorter than the statutory period for bringing action for breach of contract, as long as such limitation is reasonable and not prohibited by law. And clarified that a carrier's report on losses or damages, while standard procedure, does not serve the same purpose as a notice of claim. The report does not fix responsibility for the loss or damage but merely states the condition of the goods shipped. Therefore, it cannot be equated with the notice of claim required by bills of lading.

Also stated is the failure of the petitioners to comply with the notice and filing requirements, and the sufficiency of evidence to prove the defendants' liability. Petition is DENIED and the dismissal of the complaint and the Court of Appeals challenged judgment is AFFIRMED.

\ssection{GR241774 September 25, 2019}


\noindent\textbf{Francisco Delgado, Petitioner, v. \\GQ Realty, Respondents. CAGUIOA, J.}\vspace{0.4cm}

Disputed in the case is the ownership of a condominium unit located at Urdaneta Apartments Condominium, Makati City. Petitioner Francisco purchased the subject property using his own funds but had it registered in the name of respondent GQ Realty. Francisco entered into a relationship with Victoria Quirino Gonzales, a widow formed a corporation with her four children - GQ Reality. After the death of Victoria, petitioner Francisco learned that the subject property was transferred from GQ Realty to respondent Rosario, one of Victoria's children. Francisco then filed a Complaint for Reconveyance, Declaration of Nullity of Sale, and Damages against the respondents before the RTC. Francisco claimed to have purchased the subject property using his own funds but registered it in the name of respondent GQ Realty. Respondents argued that the subject property was acquired by respondent GQ Realty with its own funds and that it belonged to Victoria, who had the right to dispose of it as she saw fit. The respondents also presented evidence, including condominium certificates of title and a deed of absolute sale, to support their claim of ownership. The RTC dismissed the Complaint based on affirmative defenses raised by the respondents, including prescription and waiver, abandonment, or extinguishment of petitioner Francisco's claim.

Francisco appealed at CA on whether the Complaint had prescribed and his claim had been waived, abandoned, or otherwise extinguished. The CA held that the Complaint had not prescribed since petitioner Francisco was in actual and continuous possession of the subject property. However, the CA upheld the RTC's dismissal of the Complaint based on the finding that petitioner Francisco's claim had been waived, abandoned, or extinguished through the execution of the Ante-Nuptial Agreement. The CA partially affirmed the RTC's decision, dismissing the action on the ground of waiver only. Petitioner Francisco's Motion for Reconsideration was denied by the CA in a subsequent resolution. Francisco reached the supreme court seeking a favorable resolution on his claim.

\paragraph{Issue:}


The central issue is whether petitioner Francisco waived, abandoned, or otherwise extinguished his alleged interest over the subject property by executing the Ante-Nuptial Agreement.

\paragraph{Decision:}


The Supreme Court found the petition unmeritorious. The defense of waiver, abandonment, and extinguishment was sufficiently alleged in the Amended Answer, even though specific legal terms were not used. The substance of the defense was clearly presented, and the Ante-Nuptial Agreement was appended to the pleading. The issue of waiver did not necessitate a full-blown trial on the merits. The Regional Trial Court (RTC) based its finding on evidence presented during the preliminary hearing, where petitioner Francisco failed to participate.

The Court affirmed the decision of the Court of Appeals, denying the appeal and upholding its earlier rulings.

\ssection{GR222821 August 09, 2017}


\noindent\textbf{North Greenhills Association, Petitioner, v. \\Atty. Morales, Respondent. MENDOZA, J.}\vspace{0.4cm}

On June 2003, North Greenhills Association(NGA) started construction of a pavilion and restroom in McKinley Park adjacent to Atty. Morales' residence. In a complaint to the HLURB, Atty. Morales claimed uninterrupted access to McKinley Park through his side door for 33 years, alleging construction obstructed his access and violated regulation. With a Compulsory Counterclaim, in its answer, NGA rejected the assertions of Atty. Morales. After an ocular inspection, the HLURB Arbiter rendered a decision ordering removal of the pavilion and removal of obstructions to Atty. Morales' side access door. All other claims and counterclaims are dismissed for lack of merit. NGA appealed to the Office of the President(OP), but the they affirmed HLURB Board's decision and denied the subsequent motion for reconsideration. NGA reached CA, but the court affirmed the OP's decision, finding no error in characterizing the restroom as a nuisance. And the counterclaim for association dues, was in the nature of a permissive counterclaim, which was correctly dismissed. This prompted them to raise issues to the Supreme Court.

\paragraph{Issue:}

Aside from the issue on Atty. Morales' access to the park, there is a challenge to HLURB's jurisdiction over the complaint filed by Atty. Morales, and if the counterclaim of NGA against Atty. Morales for unpaid association dues was a permissive counterclaim.

\paragraph{Decision:}


The Supreme Court ruled that jurisdiction over the subject matter of a case is determined by the allegations in the complaint, irrespective of the plaintiff's entitlement to recover upon all or some claims therein. Atty. Morales' status as a member was not questioned, as evidenced by NGA's counterclaim for unpaid association dues. Therefore, HLURB had jurisdiction over the case. On Finding the Restroom as a Nuisance per Accidens. Atty. Morales failed to prove any legal easement or right of way to access the park through his side door. herefore, NGA has the right to block Atty. Morales' access through his side door. On the nature of NGA's claim. The counterclaim for unpaid association dues was permissive and not compulsory, as it did not involve the same factual and legal issues as the main complaint. Therefore, the dismissal of NGA's counterclaim was affirmed. However, NGA is not barred from filing a separate action to collect. The petition is PARTLY GRANTED.

\chapter{RULE 9. EFFECT OF PLEADINGS}


\ssection{GR243366 September 8, 2020}


\noindent\textbf{Felicita Z. Belo, Petitioner, v. \\Carlita C. Marcantonio, Respondent. REYES, J. JR.}

The main dispute revolves around the validity of the service of summons in a foreclosure case filed by the petitioner against the respondent. On January 12, 2015, Felicita Z. Belo filed a complaint for foreclosure of mortgage against Carlita C. Marcantonio. Summons was issued on January 26, 2015, addressed to the respondent's known address at Mandaluyong City. The Sheriff's Return dated January 29, 2015, stated that the summons was served through substituted service to a certain Giovanna Marcantonio, allegedly the respondent's niece. Respondent was declared in default as no responsive pleading was filed.

Before judgment was rendered, respondent learned about the case and filed a Motion to Set Aside/Lift Order of Default and to Re-Open Trial on the ground of defective service of summons. Despite respondent's motion, the RTC held that the substituted service of summons was valid and denied the motion. Respondent filed a motion for reconsideration, which was also denied.

The respondent appealed at CA the issue on the validity of the substituted service of summons. And whether the respondent's actions constituted voluntary appearance in court. The CA ruled that there was improper resort to substituted service of summons as the sheriff's single attempt to effect personal service was insufficient.The CA also ruled that respondent's motions cannot be deemed as voluntary appearance because they were filed to question the court's jurisdiction. The CA granted the petition for certiorari, annulled the RTC orders. The petitioner's motion for reconsideration was denied in the November 23, 2018 resolution of CA. Thus, bringing a petition to the Suprme Court.

\paragraph{Issue:}

Can the respondent be granted relief from the RTC's default order? Are the respondent's acts, the filing of a Motion to Set Aside/Lift Order of Default and Re-open Trial, a voluntary submission to the jurisdiction of the trial court?

\paragraph{Decision:}

The Supreme Court affirmed the Court of Appeals' ruling regarding the invalidity of the substituted service of summons. The Court emphasized the principles established in the landmark case of Manotoc v. \\CA regarding the requirements for substituted service. A single attempt to personally serve summons upon the respondent, which was insufficient to justify substituted service. Also the Court rejected petitioner's argument that respondent's filing of a Motion to Set Aside/Lift Order of Default and Re-open Trial cured the defect in the service of summons.

The importance of due process, which consists of both notice and hearing, was emphasized.

The Supreme Court affirmed the Court of Appeals' decision, annulled the RTC's orders, and directed the RTC to allow the respondent to file a responsive pleading and participate in the foreclosure proceedings.

\ssection{GR217138 August 27, 2020}

\noindent\textbf{Vitarich Corporation, Petitioner, v. \\Femina Dagmil, Respondent. LOPEZ, J.}\vspace{0.4cm}

Vitarich Corporation filed an action for sum of money against Femina Dagmil on January 15, 2010 ar RTC Malolos. Femina Dagmil's counsel moved to dismiss the case on grounds of improper venue, but the RTC denied the motion and ordered Femina Dagmil to answer the complaint.

Dagmil's counsel received the order but did not submit a responsive pleading, thus Vitarich Corporation sought to declare Femina Dagmil in default. Femina Dagmil's with a new counsel entered an appearance and filed a motion to admit an answer on January 31, 2011. RTC denied Femina Dagmil's new counsel's entry of appearance and motion to admit an answer. Then granted the filed complaint and ordered Femina Dagmil to pay Vitarich Corporation.

At CA, Femina Dagmil appealed the default judgment, seeking relief from judgment and a new trial based on her former counsel's excusable negligence and her meritorious defense.CA reversed the default judgment and remanded the case for further proceedings. The court ordered RTC to admit Femina Dagmil's answer, emphasizing Femina Dagmil's efforts to avail herself of various legal remedies to present her defense.The added that, Femina Dagmil had shown a strong desire to file an answer and prove her defense, which the trial court should not have disregarded. Aggrieved, Vitarich Corporation sought a resolution by the Supreme Court.

\paragraph{Issue:}


Did Femina Dagmil's motion to admit answer filed before the RTC declared her in default? Are health issues of Femina Dagmil's former counsel and the mistake of his secretary excusable negligence?

\paragraph{Decision:}

The Supreme Court cited the principle established in Sablas v. \\Sablas, which states that it is within the discretion of the trial court to permit the defendant to file an answer even after the reglementary period has expired, as long as the answer is filed before a declaration of default and no prejudice is caused to the plaintiff.

The Court emphasized that the RTC should have considered Femina Dagmil's answer since it was filed before the declaration of default. The fact that Femina Dagmil's counsel filed the motion to admit answer through registered mail on January 31, 2011, while the order of default was issued on February 8, 2011, supports Femina Dagmil's position. There was no showing that Femina Dagmil intended to delay the proceedings, as evidenced by her efforts to avail several post-judgment remedies. The Supreme Court concluded that the RTC gravely abused its discretion in rendering the judgment of default and affirmed the Court of Appeals' decision to reverse the judgment of default.

\chapter{RULE 13. SERVICE AND FILING OF PLEADINGS\\ AND OTHER PAPERS }


\ssection{GR244129 December 09, 2020}


\noindent\textbf{Eleonor Sarol, Petitioner, v. \\Spouses Diao, et al, Respondent. CARANDANG, J.}\vspace{0.4cm}

The dispute revolves around a parcel of land, Lot Number 7150, located in Guinsuan, Poblacion, Zamboanguita, Negros Oriental. Sarol claims to have purchased this property from Claire Chiu, but Spouses Diao, claiming that their property is adjacent to Lot Number 7150, allege that an erroneous survey of the property included 464 square meters of their land. Spouses Diao sought to partially cancel the contracts from which Claire Chiu derived ownership over Lot Number 7150, to reconvey the 464-square-meter portion of the property to them, and to hold Claire Chiu and Sarol liable for damages.

In 2007, Sarol purchased Lot Number 7150 from Claire Chiu. Sarol left the Philippines to reside in Germany, leaving her father and another individual to manage her assets, including the beach resort and Lot Number 7150. Spouses Diao discovered the alleged overlap of their property with Lot Number 7150 in 2009 and demanded the return of their portion from Claire Chiu and Sarol. In 2015, Spouses Diao filed a complaint with the RTC, seeking partial cancellation of contracts, reconveyance of their portion of the property, and damages. Sarol failed to file any pleadings with the RTC and failed attempts to be served summons personally, the RTC ordered service by publication. This leads to Sarol being declared in default. The RTC rendered a decision in favor of Spouses Diao in 2017, ordering the partial nullification of contracts, reconveyance of the 464-square-meter portion of Lot Number 7150 to Spouses Diao, and payment of damages by Sarol.

Upon learning of the events, Sarol appealed to the Court of Appeals, seeking the annulment of the RTC decision on the grounds that she was not properly served with summons and that the court lacked jurisdiction over her. The CA dismissed Sarol's petition for annulment of judgment, stating that Sarol, as a Filipino resident temporarily out of the country, was properly served with summons by publication. The CA found that Sarol failed to show clear facts and laws to support her petition for annulment of judgment. Aggrieved, Sarol went to the Supreme Court.

\paragraph{Issue:}

Did the RTC acquired jurisdiction over Sarol's person, and was there was proper service of summons to the latter??

\paragraph{Decision:}


The summons issued by the RTC indicated Sarol's address as "Guinsuan, Poblacion, Zamboanguita, Negros Oriental," which was the location of the disputed property. However, Sarol's actual residence was in Tamisu, Bais City, Negros Oriental, as evidenced by documents such as the Deed of Sale and Transfer Certificate of Title.

Substituted service requires leaving copies of the summons at the defendant's residence or place of business, which was not properly executed in this case. Extraterritorial service of summons, including service by publication, was ordered by the RTC. However, strict compliance with all requirements, including mailing copies of the summons and order of the court to the defendant's last known correct address, is necessary. In this case, such compliance was lacking.

Lack of proper service of summons deprived Sarol of the opportunity to avail ordinary remedies such as new trial, appeal, or petition for relief from judgment. Thus, her resort to the extraordinary remedy of annulment of judgment under Rule 47 of the Rules of Court was proper.

The petition for annulment of judgment is granted, and the decisions of the RTC and Court of Appeals are reversed and set aside. Lack of jurisdiction over the person of Sarol renders the RTC's decision and writ of execution null and void.

\ssection{GR214241 January 13, 2016}


\noindent\textbf{Spouses Gonzales, Petitioners, v. \\Marmaine Realty Corporation, Respondent. PERLAS-BERNABE, J.}\vspace{0.4cm}

Sps. Gonzales filed a Complaint for Recognition as Tenant with Damages and Temporary Restraining Order against Marmaine before the Office of the Provincial Adjudicator, Department of Agrarian Reform Adjudication Board (DARAB). After due proceedings, PARAD dismissed the spouses complaint(tenancy case) for lack of merit. Spouses appealed to the DARAB, but the decision was affirmed and the decision became final and executory. Marmaine, then sought the cancellation of the Notice of Lis Pendens filed by the spouses after the finality of the ruling in the Tenancy Case.

Parad initially denied Marmaine's motion for cancellation but later reversed its decision and ordered the cancellation of the Notice of Lis Pendens. The spouses petitioned the CA, however it was dismissed due to non-exhaustion of administrative remedies. Subsequently CA denied Sps. Gonzales' motion for reconsideration. Thus resulting into an issue brought before the Supreme Court.

\paragraph{Issue:}


Is it erroneous of CA to dismiss the petition for review due to the failure of the petitioners to exhaust administrative remedies? The correctness of the PARAD order, the cancellation of the notice of lis pendens annotated on the certificates of title of Marmaine's properties.

\paragraph{Decision:}

The doctrine of exhaustion of administrative remedies is a fundamental principle in the judicial system. However, there are exceptions to this doctrine, particularly when the issue involved is purely legal and must ultimately be decided by the courts. The Supreme Court found that the issue raised before the Court of Appeals was a purely legal question concerning the cancellation of the Notice of Lis Pendens, which falls under the exception to the exhaustion of administrative remedies doctrine. Therefore, the Court of Appeals erred in dismissing the petition, and the Supreme Court proceeded to resolve the issue on its merits.

The filing of a notice of lis pendens serves as a warning to third parties that the property is in litigation. However, a notice of lis pendens may be canceled under certain circumstances, such as when it is no longer necessary to protect the rights of the party who caused it to be recorded. In this case, since the Tenancy Case had already been decided against the petitioners with finality, the PARAD correctly ordered the cancellation of the notice of lis pendens. The cancellation only pertained to the Tenancy Case and did not affect any other pending cases involving the same parties. Therefore, the PARAD's decision to cancel the notice of lis pendens was upheld by the Supreme Court.

The Supreme Court denied the petition for lack of merit and affirmed the decision of the PARAD to cancel the notice of lis pendens.

\chapter{RULE 14. Summons}

\ssection{GR194751 November 26, 2014}


\noindent\textbf{Aurora De Pedro, Petitioner, v. \\Romasan Development Corporation, Respondent. LEONEN, J.}\vspace{0.4cm}

Romasan Development Corporation discovered that De Pedro and other defendants had obtained titles and free patents for a parcel of land owned by Romasan. Romasan alleged that De Pedro and other defendants obtained titles and free patents unlawfully for a parcel of land owned by the corporation. Romasan filed complaints before the Regional Trial Court of Antipolo City on July 7, 1998, seeking the nullification of the titles and free patents obtained by the defendants. Despite attempts to serve summons personally on De Pedro, it failed. As a result, the court granted a motion to serve summons and the complaint by publication. The court declared De Pedro and other defendants in default for failure to file their answers and allowed Romasan to present evidence ex parte. On January 7, 2000, the RTC issued a decision declaring null and void the titles and free patents issued to all defendants, including De Pedro. De Pedro opted to motion for new trial which was denied by the RTC.

At CA, De Pedro appealed the RTC's denial of her motion for new trial, arguing lack of jurisdiction and litis pendentia. She also filed a petition for annulment of the RTC's judgment, also citing lack of jurisdiction, litis pendentia, and denial of due process. The CA dismissed De Pedro's petition for certiorari and petition for annulment of judgment for lack of merit. The CA ruled that since De Pedro already availed herself of the remedy of a new trial and filed a petition for certiorari, she could no longer file a petition for annulment of judgment.

\paragraph{Issue:}

Is the RTC decision void due to the failure of the court to acquire jurisdiction over her person? That the RTC allowed service of summons by publication instead of ordering substituted service, rendering the trial court decision null and void?

\paragraph{Decision:}


Personal service of summons is the preferred mode, and other modes of service are justified only when necessary.

Substituted service or service by publication requires serious attempts to serve summons personally before resorting to alternative methods. In this case, summons was served by publication. The sheriff’s return in this case was defective. No substituted service or service by publication will be allowed based on such defective return.

This would have made action for annulment of judgment proper. However, the petitioner instead for a new trial, wherein, lack of jurisdiction is not a ground.

The Supreme Court upheld the decision of the Court of Appeals, denying De Pedro's petition and affirming the validity of Romasan Development Corporation's title over the disputed property.

\ssection{GR194262 February 28, 2018}

\noindent\textbf{Bobie Rose D.V. Frias, Petitioner v. \\Rolando Alcayde, Respondent. TIJAM, J.}\vspace{0.4cm}

Frias filed a Complaint for Unlawful Detainer with the Metropolitan Trial Court (MeTC) against the respondent where a decision was rendered in his favor. Thus, petitioner filed a Motion to execute the MeTC Decision.

Respondent filed a Petition for Annulment of Judgment with the RTC, alleging lack of jurisdiction and other grounds. As a result, RTC issued a TRO enjoining the MeTC from implementing its Decision. Before the RTC, the petitioner argued that the MeTC's Decision was valid, while the respondent claimed lack of jurisdiction and other procedural defects. The petitioner contested the validity of the service of summons, asserting that it was defective, therefore, the RTC lacked jurisdiction over the case. The RTC initially dismissed the Petition for Annulment of Judgment due to defective service of summons. However, it later granted respondent's motion for reconsideration, allowing further submissions and maintaining the preliminary injunction. The RTC ordered the petitioner to file an answer within a specific period.

The petitioner then appealed to the CA, contesting the RTC's decision and seeking relief through a Petition for Certiorari. The CA denied the petitioner's Petition for Certiorari, affirming the RTC's decision. The CA found no merit in the petitioner's arguments, leading to the rejection of the appeal. The petitioner filed a Petition for Review on Certiorari with the SC after the CA denied the Motion for Reconsideration.

\paragraph{Issue:}

Did the pairing judge of RTC 203 commit grave abuse of discretion by not dismissing the respondent's petition for annulment of judgment on the grounds of lack of jurisdiction over the petitioner's person. Is the nature of a petition for annulment of judgment in personam, not in rem or quasi in rem, as it affects only the parties involved and does not bind the whole world. Thus the need to acquire jurisdiction over the person of the petitioner

\paragraph{Decision:}

The Court emphasized the importance of proper service of summons in court actions to ensure due process. It stated that regardless of the type of action (in personam, in rem, or quasi in rem), proper service of summons is imperative. The court analyzed the validity of the substituted service of summons in this case and found it to be invalid and ineffective, thus violating due process. It is also ruled that the petition for annulment of judgment was improper as it cannot substitute for the lost remedy of appeal. Once a decision has become final and executory, it becomes immutable and unalterable. Therefore, the CA decision to annul the MeTC's judgment was reversed, and the petition for annulment of judgment was dismissed. The Supreme Court granted the petition, reversed the decision of the Court of Appeals, and ordered the dismissal of the respondent's petition for annulment of judgment.

\ssection{GR235099 March 29, 2022}

\noindent\textbf{Estate of Spouses Solis, Petitioners, v. \\Solis-Laynes et al, Respondents. GAERLAN, J.}\vspace{0.4cm}

The dispute revolves around the ownership of a fishpond previously owned by the Spouses Solis, with allegations of fraudulent changes in ownership and tax declarations.

Spouses Solis owned a fishpond in Romblon. After the Spouses Solis' death, changes in ownership occurred, leading to the fishpond being registered in Marivic Solis-Laynes' name. Salvador M. Solis, one of the heirs, discovered discrepancies and alleged fraud, leading to him filing a complaint before the RTC.

Set before the Regional Trial Court are allegations of fraudulent changes in ownership and tax declarations regarding the fishpond. RTC granted the motion for service by publication, and summons were published in a newspaper, after attempts to serve Marivic Solis-Laynes with summons. Marivic Solis-Laynes failed to respond within the given time frame, leading to the RTC declaring her in default. RTC rendered a decision in favor of Salvador M. Solis, declaring the free patent and original certificate of title null and void, and ordering the cancellation of related tax declarations.

Marivic Solis-Laynes contested the decision through an appeal at the CA. He challenged the validity of service of summons to him, and the compliance with procedural requirements in serving summons. CA reversed the RTC decision, dismissing the complaint filed by Salvador M. Solis. CA ruled that there was no valid service of summons on Marivic Solis-Laynes as required by Section 15, Rule 14 of the 1997 Rules of Civil Procedure. The CA found that although service by publication was ordered, Salvador M. Solis failed to mail a copy of the summons to Marivic Solis-Laynes' last known address in the USA, rendering the service of summons defective. CA held that such failure constituted a fatal defect in the service of summons, leading to the dismissal of the complaint. This prompted the spouses to seek a resolution from the SC.

\paragraph{Issue:}


Did CA fail to act on the motion of the petitioners to admit their belatedly filed Appellees' Brief, and in relying heavily on technicalities rather than the merits of the case?

\paragraph{Decision:}


The extraterritorial service of summons on Marivic was defective because the petitioner failed to strictly comply with the requirements of service by publication and failed to mail copies of the summons and complaint to Marivic's last known address in the USA, as mandated by Section 15, Rule 14 of the 1997 Rules of Civil Procedure. The defective service of summons was not cured by Marivic's voluntary appearance before the RTC. While voluntary appearance can cure defective service of summons, it does not confer jurisdiction if there are other jurisdictional defects, such as improper extraterritorial service. Marivic's filing of a Motion for New Trial did not cure the defect in the service of summons but did constitute voluntary submission to the jurisdiction of the RTC, which, however, did not satisfy the hearing aspect of due process.

The SC decided to remand the case to the RTC for trial anew, allowing Marivic to participate fully in the proceedings and present her evidence, thus affording both parties the opportunity to enforce and protect their respective rights and ensuring compliance with the constitutional guarantee of due process.

\ssection{GR206005 April 12, 2023}

\noindent\textbf{Survivors of Agrichemicals in Gensan(SAGING), Petitioners, v. \\Standard Fruit Company et al, Respondents. LEONEN, J.}\vspace{0.4cm}

SAGING and its members filed a complaint for damages against the foreign corporations for injuries suffered due to exposure to nematodes containing DBCP. Presented before the RTC are allegations of exposure to nematodes containing DBCP causing serious health issues, including cancer and reproductive system injuries. Claims of negligence on the part of the foreign corporations in manufacturing, distributing, and using DBCP without adequate warnings. The RTC dismissed SAGING's complaint for lack of jurisdiction over the foreign corporations and for failure to state a cause of action. SAGING and its members reached the CA to decide on the matter. The Court of Appeals affirmed the dismissal of SAGING's complaint by the Regional Trial Court. It upheld the rulings regarding jurisdiction, cause of action, real party in interest, and other procedural matters. Aggrieved, SAGING sought as resolution from the SC on the same issues presented before the CA.

\paragraph{Issues:}


The trial court's jurisdiction over the foreign corporations. The sufficiency of complaint's stated a cause of action. Correctness of deeming SAGING as the real party in interest. And are the claims already barred by prescription or laches?

\paragraph{Decision:}

The principal issue revolves around whether the summonses served on the foreign corporations were valid, allowing the trial court to acquire jurisdiction over them. The SC ruled that the service of summons on the respondents is presumed valid. The rule for service of summons on foreign private juridical entities that have transacted business in the Philippines is outlined in Rule 14, Section 12 of the Rules of Court. The SC clarified that foreign corporations need not be "doing business" in the Philippines for the provision to apply to them; it is sufficient that they have "transacted business" in the Philippines.

The SC discussed the recognized modes of serving summonses on foreign private juridical entities, which include service on their resident agent, government official designated by law, or any of the corporation's officers or agents within the Philippines. If none of these options are available, extraterritorial service of summons may be allowed under certain conditions. The Court noted that the amendment of Rule 14, Section 12 allows for extraterritorial service of summons for foreign private juridical entities not registered in the Philippines or those without a resident agent. This amendment was published on March 14, 2011, and applies retroactively.

The SC rejected the argument that the extraterritorial service of summons was not validly served, as the evidence presented did not sufficiently prove that summons were served only by mail. The trial court also erred in dismissing the complaint for failure to state a cause of action. It stated that while the petitioner SAGING is not the real party in interest, the complaint sufficiently states that it is being filed by SAGING with its individual members, supported by special powers of attorney executed by the members.

The SC determined that the action is not barred by prescription or laches. It noted that the interruption of the prescriptive period occurred when the previous complaint was filed and gave petitioners a fresh period for refiling the action. Moreover, laches was not proven by the respondents.

The SC granted the petition, reversed the trial court's orders dismissing the complaint, and remanded the case to the Regional Trial Court for a resolution on the merits.

\ssection{GR210302 August 27, 2020 }

\noindent\textbf{Integrated Micro, Petitioner, v. \\Standard Insurance, Respondent. LOPEZ, J.}\vspace{0.4cm}

Integrated Micro's property was insured by Standard Insurance. The property on May 24, 2009. Integrated Micro filed a claim for indemnity from Standard Insurance after the incident. Standard Insurance rejected the claim on February 24, 2010. Afterwards, Integrated Micro filed a complaint for specific performance and damages against Standard Insurance on April 11, 2011. Following Standard Insurance's motion to dismiss the complaint, the RTC decided deny and directed the former to file a responsive pleading. Standard Insurance filed a petition for certiorari with the CA prsented the following arguments:

- Prescription of Integrated Micro's cause of action.

- Validity of the service of summons.

The CA granted Standard Insurance's petition for certiorari. Integrated Micro's cause of action had prescribed, as the complaint was filed beyond the 12-month period from the rejection of the claim. The CA also ruled that the service of summons was invalid because it was served on the legal assistant of the in-house counsel, contrary to Section 11, Rule 14 of the 1997 Rules of Court. Thus, CA nullified and set aside the RTC's orders andd prompted the petitioner to seek resolution from the Supreme court.

\paragraph{Issues: }

The validity of the service of summons upon the legal assistant of Standard Insurance's in-house counsel. Has Integrated Micro's cause of action prescribed?

\paragraph{Decision:}


The Court ruled that the cause of action for filing an insurance claim commences when there is a final rejection by the insurance company, not upon the rejection of a petition for reconsideration. The insurance policy explicitly states that if a claim is made and rejected, an action or suit should be commenced within a period of 12 months from the receipt of notice of such rejection. 

Integrated Micro's cause of action had prescribed as it filed the complaint beyond the 12-month prescriptive period from the initial rejection of the claim.

On the validity of service of summons. Rule 14, Section 11 of the 1997 Rules of Court provides the manner of serving summons to a domestic private juridical entity, including corporations. The rule specifies the persons authorized to receive summons for juridical entities, which include the president, managing partner, general manager, corporate secretary, treasurer, or in-house counsel. The Court emphasized that the amendment to the rule effectively abandoned the substantial compliance doctrine and restricted the persons authorized to receive summons for juridical entities. Since Integrated Micro's service of summons upon the legal assistant of Standard Insurance's in-house counsel did not comply with the enumerated persons in Rule 14, Section 11, the service was deemed invalid.

The SC denied Integrated Micro's petition, affirmed the CA's decision, and ruled that Integrated Micro's cause of action had prescribed, and the service of summons was invalid.

\ssection{GR204753 March 27, 2019}


\noindent\textbf{United Coconut Planters Bank, Petitioner v. \\Spouses Alison Ang-Sy, Respondents. CAGUIOA, J.}\vspace{0.4cm}

On November 27, 2006, UCPB filed a Complaint before the RTC for sum of money and/or damages against NGI, the spouses Ang-Sy, Renato Ang, Nena Ang, Ricky Ang, Derick Chester A. Sy, and NPGI. The Complaint alleged non-payment of credit accommodations granted by UCPB to NGI, with surety agreements executed by NPGI and the spouses Ang-Sy.

On November 30, 2006, the RTC granted UCPB's prayer for a writ of preliminary attachment. Summonses and copies of the order granting the writ were served on the defendants on December 4, 2006. The Sheriff levied assets allegedly owned by the defendants on December 4 and 5, 2006.And on December 18, 2006, the defendants filed a Motion to Dismiss alleging lack of jurisdiction over their persons due to improper service of summons. The RTC granted the suspension of proceedings with respect to NGI and NPGI but denied the defendants' Motion to Dismiss. Defendants' Motion for Reconsideration was also denied.

The defendant appealed the CA on whether the RTC acquired jurisdiction over the persons of the defendants due to proper service of summons. The CA granted the Rule 65 Petition filed by the defendants, reversing and setting aside the RTC's Order dated June 8, 2007, on the grounds of improper service of summons. The CA held that the RTC failed to acquire jurisdiction over the persons of the defendants due to improper service of summons, rendering all proceedings before the RTC void. UCPB's Motion for Reconsideration of the CA's decision was denied, this eventually lead UPCB to a presentation of issues to the SC.

\paragraph{Issue:}

RTC's acquisition of its jurisdiction over the defendant corporations and the individual defendants. Did CA commit an error of law in finding that the RTC did not acquire jurisdiction over the individual defendants despite their motion seeking an affirmative relief?

\paragraph{Decision:}

The SC reiterated that jurisdiction over a defendant in a civil case is acquired either through proper service of summons or voluntary appearance in court. The Rules of Court prescribe the manner of service of summons, particularly for domestic private juridical entities, where service may be made only upon specific officers enumerated in Section 11, Rule 14. The absence of valid service of summons renders the court's proceedings and judgments null and void.

The SC found that the service of summons in this case was undoubtedly defective. Substituted service of summons was attempted without multiple tries, as required, and the person who received the summons was not a proper officer authorized to receive it on behalf of the corporations. The SC emphasized the importance of strict compliance with procedural rules regarding service of summons. Due to the defective service of summons, the RTC did not acquire jurisdiction over the defendants, and any proceedings and judgments rendered by the court are void.

The SC discussed the concept of voluntary appearance as an alternative means by which a court may acquire jurisdiction over a defendant. However, it clarified that the inclusion of other grounds in a motion to dismiss, aside from lack of jurisdiction over the person of the defendant, does not constitute voluntary appearance.

SC affirmed the CA's decision, holding that the RTC did not acquire jurisdiction over the defendants due to defective service of summons. Additionally, the defendants' actions did not constitute voluntary submission to the court's jurisdiction, as they explicitly objected to the court's authority over their persons. 

\ssection{GR175799 November 28, 2011}


\noindent\textbf{NM Rothschild \& Sons Australia Limited(Rothschild), Petitioner, v. \\Lepanto, Respondent. LEONARDO-DE CASTRO, J.}\vspace{0.4cm}

The issues in this case revolve around the validity of the service of summons, the sufficiency of the complaint, and the jurisdiction of the trial court over the petitioner. This arisen from the peculiarly issue - that the change of corporate name can affect being a party in the issue.

On August 30, 2005, Lepanto Consolidated Mining Company filed a Complaint against Rothschild with the RTC of Makati City. The Complaint sought a judgment declaring the loan and hedging contracts between the parties void and claimed damages. October 20, 2005, Rothschild filed a Special Appearance With Motion to Dismiss, requesting the dismissal of the Complaint on various grounds, including lack of jurisdiction due to defective service of summons, failure to state a cause of action, estoppel, and lack of clean hands on the part of the plaintiff. 

On December 9, 2005, the RTC issued an order denying the Motion to Dismiss. The court held that there was proper service of summons through the Department of Foreign Affairs (DFA) since the defendant had not applied for a license to do business in the Philippines or filed a Written Power of Attorney with the Securities and Exchange Commission (SEC) designating someone for service of legal processes. The court found that the Complaint sufficiently stated a cause of action. On December 27, 2005, Rothschild filed a Motion for Reconsideration, which was denied by the RTC, thus affirming the validity of the service of summons and sufficiency of the Complaint. Redress was sought in CA, which is then dismissed, stating that the denial of a motion to dismiss is an interlocutory order and cannot be the subject of a petition for certiorari. The CA upheld the trial court's decision and denied the petitioner's motion for reconsideration.

Still, the petitioner insists that an order denying a motion to dismiss may be the proper subject of a petition for certiorari; and the trial court committed grave abuse of discretion in not finding that it had not validly acquired jurisdiction over petitioner and that the plaintiff had no cause of action. Petitioner reached for resolution from the Supreme Court and refered to itself as Investec Australia Limited.

\paragraph{Issue:}

The effect of the change in its corporate name to petitioner's involvement to the case as a real party in interest. Did the RTC acquire jurisdiction over the person of the petitioner?

\paragraph{Decision:}

The Supreme Court found that the petitioner adequately demonstrated its identity and status as the real party in interest despite a change in its corporate name. The court emphasized that while the corporate name is crucial, the dismissal of a case solely on the basis of a change in name would be unwarranted if the identity of the party can be sufficiently established.

The Court reiterated the general rule that an order denying a Motion to Dismiss is an interlocutory order and may not be questioned through a special civil action for Certiorari. However, it acknowledged that Certiorari may be warranted in cases where the denial of the motion is tainted with grave abuse of discretion.

The Court affirmed that certain issues raised by the petitioner, such as absence of cause of action, estoppel, and unclean hands, were more appropriately dealt with during trial rather than through a Motion to Dismiss.

The Court clarified the jurisdictional aspects of extraterritorial service of summons. It determined that the action in this case was in personam and that the petitioner had voluntarily submitted to the jurisdiction of the trial court by seeking affirmative reliefs, thus upholding the trial court's jurisdiction over the petitioner.

The governing laws cited in the Supreme Court ruling includes the Rules of Court, particularly Rule 14 on Summons.

The petition is DENIED.

\ssection{GR210741 October 14, 2020 }

\noindent\textbf{Gesolgon and Santos, Petitioners, v. \\CyberOne PH, Respondents. HERNANDO, J.}\vspace{0.4cm}

This case revolves around the employment status of Gesolgon and Santos. The The main dispute in this case revolves around the employment status of Gesolgon and Santos. The Petitioners alleged that they were illegally dismissed from their positions as Customer Service Representatives and Supervisors by CyberOne AU and CyberOne PH. They also claimed non-payment or underpayment of salaries and 13th-month pay, and sought moral and exemplary damages, as well as attorney's fees. The defendants, on the other hand, denied the existence of an employer-employee relationship with CyberOne PH and argued that Gesolgon and Santos were merely shareholders or incorporators of the company. After filing a case for illegal dismissal against the defendants, Gesolgon and Santos brought their dispute to the Labor Arbiter (LA). The LA dismissed their complaint for lack of merit, stating that there was no employer-employee relationship between the plaintiffs and CyberOne PH. However, the National Labor Relations Commission (NLRC) reversed the LA's decision, ruling in favor of Gesolgon and Santos and declaring them illegally dismissed. The NLRC also found CyberOne AU liable as a foreign corporation doing business in the Philippines. Subsequently, the defendants appealed to the Court of Appeals (CA), which eventually reversed the NLRC's decision and dismissed the complaint. CA stated that CyberOne PH did not have control over the plaintiffs and did not issue the furlough notifications. No evidence was shown to support piercing the corporate veil between CyberOne AU and CyberOne PH. After the plaintiffs' motion for reconsideration was denied, which eventually lead to the SC.

\paragraph{Issues: }
\label{7e3e6950-09fd-11ef-932c-63c852f65e48}


Are the petitioners employees of CyberOne PH and CyberOne AU, and were illegally dismissed? Is the Jurisdiction over CyberOne AU acquired?

\paragraph{Decision:}
\label{7fbb1e40-09fd-11ef-932c-63c852f65e48}


The SC found the petition without merit. Gesolgon and Santos were initially hired by CyberOne AU as home-based Customer Service Representatives, not as employees of CyberOne PH. The Furlough Notifications issued by CyberOne AU indicated the termination of petitioners' services effective April 15, 2011. 

CyberOne AU, being a foreign corporation not doing business in the Philippines, cannot be subject to the jurisdiction of Philippine courts unless it voluntarily appears. The doctrine of piercing the corporate veil does not apply as there was no evidence of fraud, wrongdoing, or control by CyberOne AU over CyberOne PH. Jurisdiction over CyberOne AU cannot be acquired without valid service of summons, and since there was no valid service, no judgment can be issued against it. 

Petitioners' resignation letters as divectors of CyberOne PH and the lack of evidence regarding the control of CyberOne PH over petitioners' work negated their claim of being employees. The SC affirmed the CA’s decision, denying the petition and ruling that petitioners were not employees of CyberOne PH. Therefore, there was no dismissal, much less illegal dismissal, to speak of.

\ssection{GR214270 May 12, 2021}
\label{c3ea15a0-0a0f-11ef-932c-63c852f65e48}


\noindent\textbf{Jay Villanueva Sabado , Petitioner, v. \\Tina Marie L. Sabado, Respondent.}

Tina filed a Petition for Temporary and Permanent Protection Order, Support, and Support Pendente Lite against Jay, alleging psychological and emotional abuse, deprivation of financial support, and public humiliation. The court issued a Temporary Protection Order (TPO) in favor of Tina. Attempts were made to serve Jay with the summons, petition, and TPO, but he was not personally available. Jay's counsel received a copy of the order and petition on his behalf. Jay then filed an Entry of Appearance with Opposition to the Issuance of Permanent Protection Order, but the trial court denied the admission of Jay's opposition for being belatedly filed.The trial court issued a Permanent Protection Order (PPO) in favor of Tina. Jay appealed at Court of Appeals claiming the following:

\begin{enumerate}
  \item Tina's failure to prove the necessity for a protection order; 2) Tina's failure to prove the amount of support needed and Jay's capacity to provide the same. 3) Alleged improper service of summons affecting the court's jurisdiction.
\end{enumerate}

The CA affirmed the findings of the RTC. It held that there was no improper service of summons because Jay's counsel received a copy of the order and petition on his behalf. Jay's opposition was considered belatedly filed, and thus, the RTC correctly denied its admission. Jay presented the issues to the Supreme Court.

\paragraph{Issue:}
\label{8704cf70-09fd-11ef-932c-63c852f65e48}


The validity of the service of summons. Did Jay voluntarily submitted himself to the jurisdiction of the trial court? Jurisdictional implications of Jay's actions regarding the service of summons.

\paragraph{Decision:}
\label{88706d60-09fd-11ef-932c-63c852f65e48}


The SC clarified that summonses and protection orders serve different purposes. Summons notifies the defendant of an action brought against them and acquires jurisdiction over their person, while protection orders are substantive reliefs that prevent further acts of violence.

The Court emphasized that jurisdiction over the person of the respondent in a petition for Temporary Protection Order (TPO) / Permanent Protection Order (PPO) under Republic Act (RA) 9262 can be acquired through any means of serving summons under the Rules of Court.

The Supreme Court denied Jay's petition, affirming the Court of Appeals' decision and upholding the Permanent Protection Order issued by the Regional Trial Court.

\ssection{GR205630 January 12, 2021}
\label{c7908860-0a0f-11ef-932c-63c852f65e48}


\noindent\textbf{Diana Barber et al, Petitioners, v. \\Rolando Chua, Respondent. ZALAMEDA, J.}\vspace{0.4cm}

On August 10, 2007, Rolando Chua filed a complaint with the Municipal Trial Court (MTC) of Cainta, Rizal, seeking ejectment of extended structures encroaching upon the portion of his firewall, along with damages. The complaint was docketed as Civil Case No. MTC-1259. Diana Barber with others, filed a motion to dismiss, arguing that the MTC lacked jurisdiction over the subject matter of the case and Barber's person. They claimed that the dispute was not about physical possession of property but about the removal of structures encroaching upon Chua's property. The MTC dismissed the complaint on August 4, 2009, citing lack of jurisdiction. It ruled that the complaint did not sufficiently allege stealth or tolerance, and that Chua's prayer for the removal of permanent structures on his firewall fell short of the requirements for an ejectment complaint.

Chua appealed the MTC's decision to the RTC. The RTC reversed the MTC's decision, finding that Chua's complaint sufficiently alleged a cause of action for forcible entry. It ruled that a firewall could be the subject of an ejectment case.

Barber brought the case to the Court of Appeals, disputing the RTC's ruling. And, the CA affirmed the decision of the RTC, agreeing that Chua's complaint made out a case for ejectment. It held that the encroachment of structures onto Chua's firewall constituted unlawful dispossession of his property. Additionally, the CA found that the MTC validly acquired jurisdiction over Barber's person through substituted service, as she was a resident defendant temporarily out of the country.

\paragraph{Issue:}
\label{8bdda7b0-09fd-11ef-932c-63c852f65e48}


The Supreme Court was tasked with determining whether the Municipal Trial Court (MTC) had jurisdiction over Diana Barber's person and the subject matter of the complaint.

\paragraph{Decision:}
\label{8d78ba10-09fd-11ef-932c-63c852f65e48}


The Supreme Court ruled that the MTC did have jurisdiction over both Barber's person and the subject matter of the complaint. The Court found that the allegations in the complaint were sufficient to establish a cause of action for forcible entry. Barber, along with her co-defendants, had encroached upon Rolando Chua's property by extending their structures onto his firewall without his consent, thus depriving him of possession of part of his property. The Court also held that even though the complaint did not specifically mention dispossession of land or a building, the remedy of ejectment was still applicable because the owner of a property has rights not only to the surface but also to everything underneath and the airspace above it. Additionally, the Court upheld the validity of the substituted service of summons on Barber's aunt, considering Barber's status as a Philippine resident temporarily out of the country.

The ruling references Section 1, Rule 70 of the Rules of Court, which outlines the requirements for actions for forcible entry. It also cites case law, such as Philippine Long Distance Telephone Company v. \\Citi Appliance M.C. Corporation, to support the application of ejectment as a remedy for dispossession affecting a property owner's rights. And relies on Section 7, Rule 14 of the Rules of Court regarding the service of summons.

The petition lacks merit.

\ssection{GR244681 March 29, 2023 }
\label{caf7cf40-0a0f-11ef-932c-63c852f65e48}


\noindent\textbf{Vicente C. Go, Petitioner, v. \\Court of Appeals, Respondent. ZALAMEDA, J.}\vspace{0.4cm}

The dispute focuses on the validity of the cancellation of encumbrances in favor of Vicente C. Go by the RTC QC, which was done in favor of Spouses Colet. Vicente C. Go filed a complaint against Setcom Inc. and others for recovery of a sum of money, breach of contract, and damages. The RTC-Manila rendered a decision in favor of Vicente C. Go, ordering the defendants to pay him a sum of money, damages, and attorney's fees. An execution sale was conducted, and Vicente C. Go emerged as the highest bidder for a property, which was levied upon pursuant to the writ of execution.He then discovered that Spouses Colet filed a quieting of title case against him, claiming ownership of the same property. Spouses claimed that simultaneous to the execution of the deed of sale, they acquired the owner's duplicate copy of the title, and that they immediately took physical possession of the subject property. Petitioner, Vicente Go, failed to file an answer after the service of summons, the QC RTC declared him in default, and proceeded to try the case ex-parte. The RTC ruled in favor of Spouses Colet. Petitioner then filed for Annulment of Judgment with the Court of Appeals, questioning the RTC QC jurisdiction to interfere with the execution of the decision of the RTC-Manila. He argued that the RTC QC erred in ordering the service of summons by publication on the supposed reason that personal service of summons cannot be made. The CA dismissed the petition for being procedurally and substantially defective. They noted the lack of supporting documents and evidence regarding improper service of summons. The CA also found discrepancies in Vicente C. Go's addresses and denied his motion for reconsideration. Afterwards, Vicente proceeded to defend his interest at the Supreme Court.

\paragraph{Issue:}
\label{90340980-09fd-11ef-932c-63c852f65e48}


The proper service of summons on the petitioner. Does the petitioner's interest in the subject property enjoy preference over the prior unregistered sale to Spouses Colet?

\paragraph{Decision:}
\label{919d0f60-09fd-11ef-932c-63c852f65e48}


The SC ruled that there was valid service of summons by publication after diligent efforts were made by the sheriff to serve the summons personally but to no avail. The ruling cited precedents such as Titan Dragon Properties Corp. v. \\Veloso-Galenzoga, Sagana v. \\Francisco, and Carson Realty \& Management Corp. v. \\Red Robin Security Agency to support this determination. The ruling cited various methods of service of summons, including personal service, substituted service, and service by publication.

In regards to the subject property, the Court ruled that Spouses Colet's interest is superior to that of the petitioner. It cited Miranda v. \\Spouses Mallari to support the principle. That a registered levy on attachment or execution takes precedence over a prior unregistered sale only if ownership has not vested in the prior buyer before the levy. Since Spouses Colet acquired ownership of the property six years before the levy made by the petitioner, their interest prevails. The ruling affirmed the decisions of the lower courts and denied the petitioner's appeal.

\ssection{GR246088 April 28, 2021 }
\label{cea96590-0a0f-11ef-932c-63c852f65e48}


\noindent\textbf{Titan Dragon Properties Corporation, Petitioner, v. \\Marlina Veloso-Galenzoga, Respondent. ZALAMEDA, J. }

The dispute revolves around the ownership and possession of a 70,364-square meter parcel of land in Quezon City, Philippines. Respondent Marlina, alleges that she purchased the property from the petitioner Titan Corp, through a Deed of Absolute Sale in 1997 but claims that the petitioner failed to deliver possession of the property and pay the necessary taxes. Attempts were made to serve summons to the petitioner, but there were issues with the service process. The RTC rendered a default judgment in favor of the respondent, ordering the petitioner to pay taxes and deliver possession of the property. Petitioner disputes the validity of the sale and challenges the service of summons in the subsequent legal proceedings. The petitioner appealed to the CA, to present their challenge. The CA dismissed the petitioner's petition for certiorari, affirming the RTC's decision. The court ruled that service of summons by publication was properly done and upheld the RTC's rulings in both the mandamus and specific performance cases. The CA also denied petitioner's motion for reconsideration, leading the case to the Supreme Court.

\paragraph{Issue:}
\label{94f0a730-09fd-11ef-932c-63c852f65e48}


Validity of the service of summons by publication. Propriety of the expansion of the writ of execution issued in the specific performance case.

\paragraph{Decision:}
\label{96a1d9a0-09fd-11ef-932c-63c852f65e48}


Personal service of summons is the preferred mode, but other modes, including service by publication, are allowed under certain conditions, requiring diligent efforts to locate the defendant. The failure to diligently attempt personal or substituted service before resorting to service by publication renders the service invalid.

In this case however, the sheriff served the summons at least three times on different dates at the addresses stated in the petitioner's own complaint for sum of money, and Certificate of Sale. Thus, the petition was denied. The Resolutions by the Court of Appeals are AFFIRMED.

\ssection{GR223610 July 24, 2017 }
\label{d1d74cf0-0a0f-11ef-932c-63c852f65e48}


\noindent\textbf{Uy, Petitioners v. \\Del Castillo, Respondents. Perlas-Bernabe, J.}\vspace{0.4cm}

Crispulo Del Castillo filed an action against Jaime Uy, later amended to include Jaime's children, over ownership of Lot 791. After proceedings, the RTC ruled in favor of the respondednts, Crispulo's heirs, declaring them as the true owners of Lot 791 and ordering petitioners to pay damages and attorney's fees. Respondents then filed a motion for the issuance of a writ of execution to enforce the RTC decision, which the court granted. THe Petitioners then appealed at CA, questioning the validity of the writ of execution, along with the proper computation of attorney's fees, and also raised jurisdictional issues regarding summons served to them. CA affirmed RTC's orders, finding no merit in petitioners' claims. It ruled that the writ of execution was valid as it made express reference to the RTC decision without adding anything else. CA also found that petitioners were properly served with summons and dismissed their jurisdictional arguments, and were liable for damages and attorney's fees. Upon denial of their motion for reconsideration, the petitioners presented the issues to the Supreme Court.

\paragraph{Issue:}
\label{98bf8980-09fd-11ef-932c-63c852f65e48}


 The correctness of the Orders of the Court of Appeals the Regional Trial Court, which includes the pegged the attorney's fees at 3,387,970. Are the Uy siblings properly served with summons, and voluntarily submitted to the jurisdiction of the RTC?

\paragraph{Decision:}
\label{9a28b670-09fd-11ef-932c-63c852f65e48}


The SC found that the CA correctly upheld the RTC orders. It ruled that the Uy siblings were properly served with summons and voluntarily submitted to the jurisdiction of the RTC. The SC emphasized that judicial admissions made by parties are conclusive and binding, and petitioners cannot now assert otherwise. Furthermore, the SC stated that active participation in the case, including appealing adverse rulings, constitutes an invocation of the court's jurisdiction.

The SC did acknowledg that while the Uy siblings may be held answerable to the monetary awards in the Quieting of Title Case, their liability cannot exceed the value of their inheritance. The SC modified the CA decision to limit the Uy siblings' monetary liability. The SC partly granted the petition, affirming the CA decision with modification to limit the Uy siblings' monetary liability to the value of their inheritance from Jaime Uy.

\ssection{GR236924 March 29, 2023 }
\label{d48ea4c0-0a0f-11ef-932c-63c852f65e48}


\noindent\textbf{Diversified Plastic Film System, Petitioner, v. \\Philippine Investment One, Respondent. GAERLAN, J.}\vspace{0.4cm}

iversified obtained a loan from All Asia Capital and Trust Corporation, secured by a Mortgage Trust Indenture (MTI). All Asia transferred its rights, titles, and interests in the MTI to Development Bank of the Philippines (DBP) through a dation in payment. DBP then assigned the loan to PI-One, which in turn demanded payment from Diversified. These subsytitutions lead to the filing of a Petition for the extra-judicial foreclosure of Diversified's mortgaged properties.

Diversified filed a Complaint for injunction to oppose the foreclosure, leading to the issuance of a TRO by the RTC of Mariveles, Bataan. PI-One challenged that TRO before the CA, which ruled in their favor - annulling the TRO issued by the RTC Bataan. Then PI-One filed a Petition for Appointment as Trustee before the RTC of Makati City, Branch 143, seeking confirmation as trustee under the MTI. The RTC granted PI-One's petition and appointed it as trustee under the MTI. The RTC found that the position of trustee had long been vacant, and PI-One inherited the rights of All Asia and DBP.

Thus appointment lead to an appeal at CA. Diversified challenged the jurisdiction of the RTC to appoint PI-One as trustee, the proper service of summons, qualifications of PI-One as trustee. The CA denied Diversified's appeal and affirmed the RTC's decision appointing PI-One as trustee. The CA found that PI-One acquired the rights of All Asia and DBP and thus had the authority to act as trustee under the MTI. The CA also ruled that the RTC had jurisdiction to appoint PI-One as trustee and that the service of summons issue had become moot. Diversified then reached escalated the casew to the Supreme Court.

\paragraph{Issue:}
\label{9da496c0-09fd-11ef-932c-63c852f65e48}


Along with the improper service of summons to them, Diversified argued that the trial court did not have jurisdiction to appoint a trustee, it violates Section 12 of RA 9182, the Special Purpose Vehicle Act. Cited provision requires prior notice to borrowers and persons holding prior encumbrances upon the assets being transferred, as well as a prior certification of eligibility.

\paragraph{Decision:}
\label{9f7e38c0-09fd-11ef-932c-63c852f65e48}


The SC agreed with the CA that the trial court had jurisdiction to appoint a trustee under the MTI. However, the SC found that the trial court failed to acquire jurisdiction over Diversified due to improper service of summons. The assignment between DBP and PI-One was invalid because it did not comply with the requirements of Section 12 of RA 9182. Without proper notice to borrowers and prior certification of eligibility, the transfer of rights under the MTI was deemed invalid.

Even assuming the assignment was valid, the SC agreed with Diversified that PI-One could not serve as trustee under the MTI because it did not meet the qualifications required by the MTI. Specifically, PI-One was not an institution authorized to engage in trust business in Metro Manila.

Diversified's petition was granted and the decision of the CA is reversed , dismissing PI-One's petition for appointment as trustee.

\chapter{RULE 17 DISMISSAL OF ACTION }
\label{efbe90a0-0a11-11ef-932c-63c852f65e48}


\ssection{GR192903 February 01, 2023 }
\label{d7c16e20-0a0f-11ef-932c-63c852f65e48}


\noindent\textbf{Quiambao, Petitioner, v. \\Bonifacio, Respondents. GAERLAN, J.}\vspace{0.4cm}

This is a dispute revolving around the legality of Pacifica, Inc.'s Annual Stockholders' Meeting (ASM) and the election of its Board of Directors. Respondents alleged that the ASM was conducted in violation of Pacifica's by-laws and the Corporation Code. They sought to enjoin the ASM and nullify the election of the new set of Board of Directors.

Bonifacio C. Sumbilla and Aderito Z. Yujuico, members of the Board of Directors of Pacifica, filed several separate complaints against Cesar T. Quiambao, Owen Casi Cruz, and Anthony K. Quiambao, and Pacifica. The cases are the following. A Pasig Case, filed on August 21, 2007, seeking to enjoin Pacifica's ASM. However, ASM proceeded as scheduled due to failure to serve summons and notices properly. An amended complaint was filed to declare the ASM illegal and nullify the election of the new Board of Directors. And the Manila and Makati cases where Respondents sought clarification from the SEC regarding Pacifica's principal place of business. They filed complaints in Manila and Makati on September 7, 2007, seeking similar relief as in the Pasig Case. Respondents stated that they filed three cases due to doubts about Pacifica's principal place of business. Such manifestation wa included in the Verification and Certification Against Forum Shopping attached to their complaints. Upon clarification from SEC, that Pacifica's principal place of business is Makati City. The respondents immediately withdrew complaints in Pasig and Manila Cases. The Makati Case proceeded after proper service of summons. Then the Respondents filed a motion for judgment by default, which was granted by the RTC. Petitioners then filed for a Certiorari before the CA questioning the validity of service of summons upon petitioners, and an allegation of forum shopping by respondents. CA partially granted the petition for certiorari, reversing and setting aside the RTC's order granting judgment by default. CA found that summons were improperly served upon petitioners; however concluded that the filing of three separate cases was justified due to confusion about Pacifica's principal place of business and did not constitute willful or deliberate forum shopping. The petitioners then presented to the Supreme Court, issues challenging the CA's decision.

\paragraph{Issues: }
\label{a30c4180-09fd-11ef-932c-63c852f65e48}


Is there an error in declaring that the respondents were not guilty of forum shopping?

\paragraph{Decision }
\label{a4854cf0-09fd-11ef-932c-63c852f65e48}


The SC cited previous cases to establish the elements of forum shopping. These include identity of parties, identity of rights asserted and relief prayed for, and the identity of the two preceding particulars such that any judgment rendered in one action will amount to res judicata in the other. The SC then examined the circumstances of the case and concluded that respondents did not commit forum shopping.

It noted that respondents filed three similar cases in different courts due to uncertainty about the proper venue of their action, as indicated by conflicting information in Pacifica's corporate records. Moreover, the SC highlighted that respondents withdrew the Pasig and Manila Cases immediately upon receipt of clarification from the SEC regarding Pacifica's principal place of business. The SC emphasized that the withdrawal of the cases negated any deliberate intent on respondents' part to engage in forum shopping.

The Petition was denoed. And the decision and resolution of the CA are affirmed.

\ssection{GR259282 August 30, 2023}
\label{dab24c80-0a0f-11ef-932c-63c852f65e48}


\noindent\textbf{Spouses Antonio, Petitioners, v. \\Bank of the Philippine Islands. INTING, J.}\vspace{0.4cm}

Spouses Antonio and Monette Prieto executed real estate mortgages over their properties to secure loans from Far East Bank and Trust Company (FEBTC), which later became Bank of the Philippine Islands (BPI) as its successor-in-interest. BPI filed a complaint against Spouses Antonio and Monette Prieto for the alleged deficiency in their loan obligations, claiming that they failed to pay the amounts due. BPI initiated foreclosure proceedings on the mortgaged properties and conducted public auctions to satisfy the outstanding loan obligations. The Regional Trial Court then issued summons to Spouses Antonio and Monette Prieto, but the summons got lost. Subsequently, an alias summons was issued and personally served through Monnette Prieto. On November 11, 2005, BPI was substituted by Philippine Asset Investments(PAI),and declared Spouses Prieto in default. Due to difficulty in collating evidence, upon motion of (PAI) the case was temporariliy archived. On July 9, 2015, PAI was further substituted by Philippine Investment One (PIO). RTC initially dismissed the case for lack of interest to prosecute, however, upon motion for reconsideration from PIO it was reversed and the court admitted their Formal Offer of Evidence. The RTC ruled in favor of BPI and its successor-in-interest entities. The court found that Spouses Prieto had been extended loans secured by real estate mortgages. It is determined that the Prietos defaulted on their loan obligations.

Petitioners received a copy of the Decision on October 18, 2017 and without a motion being filed by both parties, the decision became final and executory. Still, the petitioners filed a Petition for Certiorari before the CA arguing a violation of Section 34,45 Rule 132 of the Rules of Court; and error in awarding fees amounting to l3,268,303 to the successor entity of BPI. CA dismissed the Petition for Certiorari filed by Spouses Prieto, affirmed the RTC decision, then denied their motion for reconsideration. Thus, the Spouses presented the issue to the Supreme Court.

\paragraph{Issue:}
\label{a8b5b030-09fd-11ef-932c-63c852f65e48}


CA's dismissal of the petition for certiorari, outright on procedural grounds. The considered the subject documents in violation of Section 34, Rule 132 of the Rules of Court. And the awarded 13,268,303 despite the insufficiency of the evidence on record.

\paragraph{Decision:}
\label{aa387fa0-09fd-11ef-932c-63c852f65e48}


On the procedural issue, the Supreme Court found that the CA did not err in dismissing the petition because of procedural lapses on the part of the petitioners. While a motion for reconsideration of the RTC decision should have been filed before resorting to a petition for certiorari, the Court nonetheless decided to rule on the substantive issues raised in the petition due to the merits of the case and the interest of justice.

On the substantive issue, the Supreme Court held that the RTC erred in considering the subject documents without formal offer, violating Section 34 of Rule 132 of the Rules of Court. The exception allowing the consideration of evidence not formally offered did not apply in this case since the subject documents were not duly identified by testimony duly recorded and incorporated into the records of the case.

Also the Court found that the evidence presented by Philippine Investment One was insufficient to justify the deficiency claim of P13,268,303. There is failure to establish the material facts necessary to support their claim for deficiency, including the amount of the obligation prior to foreclosure and the proceeds of the foreclosure. Moreover, the delay in prosecuting the action for an unreasonable length of time prejudiced the petitioners.

The Supreme Court granted the petition, setting aside the CA decision and the RTC rulings. The Court dismissed the complaint for deficiency claim (Civil Case Number 02-683) and ordered it to be dismissed with prejudice due to Philippine Investment One's failure to prosecute the action for an unreasonable length of time. The dismissal with prejudice was deemed necessary to uphold the petitioners' right to speedy disposition of their case and prevent further litigation on the matter.

\ssection{GR234999 August 04, 2021 }
\label{dd90b450-0a0f-11ef-932c-63c852f65e48}


\noindent\textbf{Heirs of Sanchez, Petitioners, v. \\Abrantes, Respondents. INTING, J.}\vspace{0.4cm}

The disputes revolve around the ownership and status of a parcel of land in Poblacion, Agusan del Norte. The plaintiffs in both complaints seek the nullification of a sale executed by Horacio Abrantes in favor of Bartolome Sanchez covering the disputed property. On March 19, 2002, Horacio Abrantes filed the First Complaint against the heirs of Bartolome J. Sanchez. Before the RTC Branch could act on the motion to dismiss filed by Bartolome's heirs, Horacio died on April 27, 2003. Horacio's counsel moved for the dismissal of the First Complaint due to lack of interest from Horacio's heirs. On August 13, 2004, the RTC Branch 5 issued the First Dismissal Order for the First Complaint. And on October 18, 2010, the Entry of Judgment made the First Dismissal Order final and executory. More than four years after the death of Horacio Abrantes, his heirs filed the second Complaint (Civil Case Number 5806) pertaining to the same property; for Declaration of Nullity of Sale, Reconveyance, and Damages against the defendants before RTC Branch 3. The RTC Branch 3 dismissed the Second Complaint on October 20, 2014, citing res judicata relative to the First Dismissal Order of the RTC Branch 5. The Abrantes party(respondents) appealed the dismissal of the Second Complaint at the CA, alleging grave abuse of discretion on the part of the RTC Branch 3. The CA affirmed the dismissal of the Second Complaint but based its decision on litis pendentia rather than res judicata.

The CA held that the dismissal of the First Complaint was a patent nullity due to procedural irregularities, thus rendering it void and without legal effect. Consequently, the CA considered the First Complaint as pending, which barred the filing of the Second Complaint. The motion for reconsideration filed by the Sanchez party(petitioners) was denied by the CA, thus they presented the issue to the SC to be resolved.

\paragraph{Issue:}
\label{acc8ee80-09fd-11ef-932c-63c852f65e48}


The correctness of ruling that the First Dismissal Order was a nullity. And did the CA gravely erred in dismissing the Second Complaint on the ground of litis pendentia

\paragraph{Decision:}
\label{ae9d8770-09fd-11ef-932c-63c852f65e48}


The SC discusses the principles of substitution in case of the death of a party, emphasizing that noncompliance with this rule results in a denial of due process for the heirs affected by the decision. While violations of due process may warrant nullification of proceedings and judgments, the violation is personal to the party asserting the defense.

The SC analyzes whether the dismissal of the Frist Complaint was an adjudication on the merits or without prejudice. Section 3, Rule 17 of the Rules of Court governs dismissals due to the fault of the plaintiff, which include failure to prosecute. The dismissal cannot be characterized as failure to prosecute since it did not proceed from any instance outlined in the rule. Thew claim of lack of interest from Horacio's heirs was hearsay, and the dismissal was sought by the plaintiff's counsel, not the defendant. Therefore, the dismissal was not an adjudication on the merits and was without prejudice.

The SC denies the petition, sets aside the CA's decision, and reinstates Civil Case Number 5806, remanding it to the Regional Trial Court for further proceedings.

\ssection{GR232189 March 7, 2018 }
\label{e1f965a0-0a0f-11ef-932c-63c852f65e48}


\noindent\textbf{Alex Blay, Petitioner v. \\Cynthia Bana, Respondent. PERLAS-BERNABE, J.}\vspace{0.4cm}

On September 17, 2014, the Petitioner filed a Petition for Declaration of Nullity of Marriage before the RTC. Respondent filed her Answer with Compulsory Counterclaim on December 5, 2014. In her Answer with Compulsory Counterclaim, she raised counterclaims against the petitioner. Petitioner later lost interest in the case and filed a Motion to Withdraw his petition. Thus, the Respondent filed a comment/opposition to the Motion to Withdraw, invoking Section 2, Rule 17 of the Rules of Court, and prayed that her counterclaims be declared as remaining for the court's independent adjudication. Petitioner filed a reply, arguing that respondent's counterclaims are barred from being prosecuted in the same action due to her failure to file a manifestation within fifteen days from notice of the Motion to Withdraw. The RTC granted petitioner’s Motion to Withdraw petition and also declared respondent's counterclaim "as remaining for independent adjudication." Petitioner's motion for reconsideration was denied leading him to an appeal at CA. He sought to to set aside the RTC Orders to the extent that they allowed the counterclaim to remain for independent adjudication before the same trial court. The CA dismissed the petition for lack of merit. Holding that under Section 2, Rule 17 of the Rules of Court, if a counterclaim has been filed by the defendant before the service upon him of the petitioner’s motion for dismissal, the dismissal shall be limited to the complaint. The CA's decision lead the petitioner to present issues to the supreme Court.

\paragraph{Issue:}
\label{b26b3140-09fd-11ef-932c-63c852f65e48}


Did the Court of Appeals err in upholding the Regional Trial Court orders, declaring the respondent's counterclaim for independent adjudication before the same trial court?

\paragraph{Decision:}
\label{b3dce9b0-09fd-11ef-932c-63c852f65e48}


The SC emphasized that if a counterclaim has been pleaded by the defendant before the service upon him of the plaintiff's motion for dismissal, the dismissal shall be limited to the complaint. However, the defendant must file a manifestation within fifteen days from notice of the plaintiff's motion for dismissal if they desire to prosecute their counterclaim in the same action. Failure to do so would result in the counterclaim being prosecuted only in a separate action. The SC pointed out the error in CA's interpretation by ignoring the prescribed period, thus erroneously sustaining the RTC Orders.

Petition is GRANTED. The CA rulings are reversed and set aside the RTC Orders. And SC granted the petitioner's Motion to Withdraw the Petition for Declaration of Nullity of Marriage but emphasized that the dismissal was without prejudice to the prosecution of respondent's counterclaim in a separate action.

\chapter{RULE 18 PRE-TRIAL }
\label{050b4b10-0a12-11ef-932c-63c852f65e48}


\ssection{GR238892 September 04, 2019 }
\label{e8a1f4d0-0a0f-11ef-932c-63c852f65e48}


\noindent\textbf{Spouses Su, Petitioners, v. \\Eda Bontilao, Pablita Bontilao, and Maricel Dayandayan, Respondents. PERLAS-BERNABE, J.}\vspace{0.4cm}

The dispute revolves around the possession of a parcel of land in Lapu-Lapu City. Spouses Su (petitioners) are registered owners of the subject property covered by TCT Number 29490. Petitioners then filed a complaint for unlawful detainer, damages, and attorney's fees against respondents before the MTCC Lapu-Lapu City on March 1, 2012. Petitioners then informed respondents of their need for the property and requested them to vacate, but respondents refused. Respondents claimed to be the legitimate heirs of the original owner of the property, Mariano Ybañez, and asserted ownership over it. After failed attempts at an amicable settlement, petitioners filed the complaint for unlawful detainer. 

The MTCC initially dismissed the case for failure of petitioners and their counsel to appear at the preliminary conference. After a motion for reconsideration, the MTCC granted the motion and reset the preliminary conference. The MTCC eventually ruled in favor of petitioners, finding that respondents occupied the property by mere tolerance of petitioners and ordered them to vacate. Respondents then appealed the MTCC's decision to the RTC. The RTC affirmed the MTCC's decision, stating that a motion for reconsideration was not prohibited in this case and that pre-trial briefs were not mandatory. It upheld the MTCC's finding that respondents occupied the property by mere tolerance and were thus bound to vacate.

Accordingly, respondents filed an appeal before the CA. The CA reversed the RTC's decision, dismissing the complaint altogether. It held that the MTCC's dismissal of the case for failure of petitioners and their counsel to appear at the preliminary conference should be affirmed. The CA found that petitioners failed to provide a satisfactory explanation for their absence, and the appearance of parties and counsel during the preliminary conference is mandatory in ejectment cases. The CA also rejected petitioners' motion for reconsideration, thus elevating the issue to the Supreme Court.

\paragraph{Issue:}
\label{b6948fa0-09fd-11ef-932c-63c852f65e48}


The correctness of the CAs decision to reverse and set aside the lower courts' decisions based on purely procedural considerations. Are the possession of the respondents over the subject property originally lawful, and if so, did it turned unlawful upon the expiration of their right to possess?

\paragraph{Decision:}
\label{b88fea70-09fd-11ef-932c-63c852f65e48}


The Revised Rules on Summary Procedure provide that failure of the plaintiff to appear in the preliminary conference shall be a cause for dismissal of the complaint. However, this rule also states that the non-appearance of a party may be excused if a valid cause is shown or if a representative appears on their behalf with full written authority. In this case, petitioners had executed a Special Power of Attorney (SPA) authorizing their former counsel to represent them during the preliminary conference. The existence of this SPA was sufficient written authorization that excused petitioners' non-appearance.

The motion for reconsideration filed by petitioners' counsel was not a prohibited motion, as it sought reconsideration of an order of dismissal on the ground of non-appearance at the preliminary conference, not a judgment rendered on the merits. The lack of filing of a pre-trial brief by petitioners was not a ground for dismissal, as cases governed by the Rules on Summary Procedure may be resolved based on pleadings, affidavits, and position papers.

Regarding the issue of possession, unlawful detainer requires the plaintiff to establish that the possession was initially lawful but turned unlawful upon the expiration of the right to possess. The court found that petitioners failed to adduce evidence to establish that respondents' occupation of the property was by mere tolerance. There was a lack of evidence showing how and when respondents entered the property and how and when permission to occupy was given by petitioners. The mere possession of a Torrens title does not allow unbridled authority to immediately eject the current possessor without proving the essential requisites of unlawful detainer.

The petition is DENIED.

\ssection{GR232682 September 13, 2021 }
\label{ec054410-0a0f-11ef-932c-63c852f65e48}


\noindent\textbf{Patricio G. Gemina, et al, Petitioners, v. \\Heirs of Espejo, Respondents. HERNANDO, J.}\vspace{0.4cm}

The dispute is about the ownership and possession of a property located at Batasan Hills, Quezon City. To start, Gemina claimed to have purchased, owned, occupied with his family, and possessed the subject property since 1978. Heirs of Espejo asserted co-ownership of the subject property and demanded Gemina and his family to vacate, eventually resorting to legal action. The Heirs of Espejo then filed an action for recovery of possession against Gemina in the MTC of Quezon City, which was later withdrawn, as they filed the complaint for recovery of possession in the RTC. On the scheduled pre-trial, Gemina was present but his counsel failed to attend. The counsel is again not present in the in the succeeding reschedules. Gemina presented various documents to support his claim of ownership and possession of the subject property since 1978.

The trial court did allow the heirs of Espejo to present their evidence ex parte in its November 26, 2012 Order. Heirs of Espejo presented documents such as the deed of absolute sale, transfer of rights, and tax declarations to establish their ownership and possession of the subject property. Gemina's counsel filed a Withdrawal of Counsel with Attached Motion for Reconsideration citing health reasons. Gemina learned about it only when he followed up the case. Gemina filed a Motion for Reconsideration, now aided by by a new counsel from PAO, on the grounds that he did not receive the notice of withdrawal. Thius was denied by the court. In its September 3, 2013 Decision, the RTC ruled in favor of the heirs of Espejo, ordering Gemina to vacate the subject property and pay reasonable compensation.

Gemina appealed the decision of the RTC to the CA. The CA affirmed the ruling of the RTC but modified the rate of interest and deleted the award of attorney's fees. They held that Gemina could no longer question the propriety of the RTC's order allowing the heirs of Espejo to present evidence ex parte, as Gemina had already filed a motion for reconsideration which lacked the requisite notice of hearing. The CA also found the documentary evidence submitted by the heirs of Espejo to satisfactorily establish their better right of possession over the subject property. The court then modified the legal rate of interest and deleted the award of attorney's fees in its ruling. After Gemina's move for reconsideration was denied, the case is then elevated to the Supreme Court

\paragraph{Issue:}
\label{bbf4c050-09fd-11ef-932c-63c852f65e48}


The CA's decision in affirming the trial court's order, allowing the respondents to present their evidence ex parte due to the absence of the petitioner's counsel during pre-trial. Did the decision violate the petitioner's right to due process?

\paragraph{Decision:}
\label{be60b740-09fd-11ef-932c-63c852f65e48}


The absence of the defendant's counsel during pre-trial, despite the presence of the party-defendant, does not automatically result in the plaintiff's ex parte presentation of evidence. Sections 4 and 5, Rule 18 of the Revised Rules of Court mandate the appearance of parties and their counsels during pre-trial, with consequences for their failure to appear.

In this case, the absence of Gemina's counsel during pre-trial should not have resulted in the ex parte presentation of evidence by the respondents, especially since Gemina himself attended the pre-trial. The lack of notice of hearing in Gemina's motion for reconsideration was cured by the opportunity given to the adverse party to file opposing pleadings, thereby satisfying the requirements of due process.

The Petition is GRANTED, and the case REMANDED to the Regional Trial Court, Quezon City for further proceedings in accordance with this Decision.

\ssection{GR248519 March 17, 2021 }
\label{ef9129f0-0a0f-11ef-932c-63c852f65e48}


\noindent\textbf{St. Francis Plaza Corporation, Petitioner v. \\Emilio Solco, et al, Respondents. INTING, J.}\vspace{0.4cm}

This complex intra-corporate controversy is composed of three consolidated petitions. All parties filed criminal charges against each other. And the adversaries to each case are as follows. St. Francis Plaza Corporation (SFPC) against Solcos( Emilio, Francis, Lily, and Benz). One by Francis Solco filed against Emilio Solco. Then, Lily and Benz Solco against RTC QC Judge Malabaguio and Emilio Solco.

Central to their dispute is a Compromised Agreement approved by RTC QC. The parties entered into a Compromise Agreement, which included provisions for the termination of criminal cases, settlement of claims over shares of stock and real properties, and other matters. The RTC approved the Compromise Agreement. Emilio moved for execution of the Judgment on the Compromise Agreement, however, the Francis Group opposed the motion, alleging breaches of the Compromise Agreement by Emilio. The RTC ordered simultaneous compliance with the terms of the Compromise Agreement by both parties. And the court refused to annul the Compromise Agreement, holding that it had the effect of res judicata.

Benz and Lily then filed a Petition for Certiorari in CA. The court denied the consolidated petitions, stating that without vitiation of consent or economic damage, an allegation of prejudice or inequity is not sufficient to nullify the Compromise Agreement. The CA held that if Emilio breached the agreement, the aggrieved parties could enforce it with the assistance of the trial court through a writ of execution. The CA denied the subsequent motion for reconsideration filed by the Francis Group. It cannot rescind the Compromise Agreement on account of Emilio's non-performance of his covenants therein. The controversy then reached the Supreme court. 

\paragraph{Issue:}
\label{c0f0ff10-09fd-11ef-932c-63c852f65e48}


Is SFPC was an indispensable party to the Compromise Agreement? Is the Compromise Agreement void for being contrary to law and public policy? Are the consent of the parties in the Compromise Agreement was vitiated by fraud and mistake? Did the parties involved, validly exercised the option of rescinding the Compromise Agreement due to Emilio's failure to comply with its terms? 

Considering the contentions of all parties, the complexity can be reduced to whether the Compromise Agreement should be nullified.

\paragraph{Decision:}
\label{c2ad06f0-09fd-11ef-932c-63c852f65e48}


The absence of an indispensable party renders all subsequent acts of the court null and void. And SFPC was deemed to have participated in the Compromise Agreement as it was duly represented through its president, Francis, who was authorized by a Board Resolution.

The Compromise Agreement was not void for being contrary to law and public policy because the dismissal of criminal cases was not solely anchored on the Compromise Agreement. Emilio's failure to comply with his obligations under the Compromise Agreement constituted a major breach, justifying rescission.

The Francis Group validly regarded the unimplemented portions of the Compromise Agreement as rescinded due to Emilio's material breach and noncompliance. The sale of certain properties was upheld despite the rescission of the Compromise Agreement's unimplemented portions due to the presence of a separability clause.

The petition is PARTIALLY GRANTED. The Decision and Resolution of the Court of Appeals are modified in that the Comprehensive Compromise Agreement dated May 4, 2013 is CANCELLED in so far as the unimplemented portions thereof are concerned.

The assailed Decision and Resolution are AFFIRMED in all other respects

\ssection{GR219431 August 24, 2020 }
\label{f59fea20-0a0f-11ef-932c-63c852f65e48}


\noindent\textbf{Spouses Garcia, PETITIONERS, v. \\Spouses Soriano, RESPONDENTS. INTING, J.}\vspace{0.4cm}

The dispute revolves around a compromise agreement reached between the respondents (Soriano spouses) and the petitioners (Garcia spouses) regarding the ownership of real property. The petitioners failed to fulfill their obligations under the compromise agreement, leading to a motion for execution by the respondents. The compromise agreement was reached on October 29, 2005. The compromise agreement allowed the Garcia spouses a grace period to repurchase/redeem the subject property, but they failed to do so within the stipulated time frame. The Soriano spouses moved for execution of the judgment based on the compromise agreement on September 9, 2008. 

The RTC granted the motion for execution, leading to the issuance of a writ of execution against the Garcia spouses. The RTC denied the Garcia spouses' motion to quash the writ of execution. They appealed at the CA the validity of the RTC's decision to issue the writ of execution . The CA upheld the validity of the RTC's decision, the appellant court held that a compromise agreement approved by a court becomes final and executory and cannot be modified or amended. Since the Garcia spouses failed to fulfill their obligations under the compromise agreement within the stipulated time frame, the Soriano spouses were entitled to seek execution of the judgment. The CA denied the Garcia spouses' motion for reconsideration, affirming its earlier decision. 

\paragraph{Issue:}
\label{c5d4d3d0-09fd-11ef-932c-63c852f65e48}


Correctness of RTC's issuance of the writ of execution to enforce the subject judgment based on compromise agreement. Did the proper party litigants validly entered into a new or modified compromise agreement which superseded the judgment based on compromise agreement? 

\paragraph{Decision:}
\label{c7e2cc40-09fd-11ef-932c-63c852f65e48}


The SC affirmed the ruling of the CA but for different reasons. The SC noted that the execution proceedings were unnecessarily drawn-out due to the RTC erroneously permitting the petitioners to resort to improper remedies. The SC cited Section 9, Rule 15 of the Revised Rules of Civil Procedure, known as the Omnibus Motion Rule, which requires all available objections to be raised in a single motion to avoid multiple objections.

On the validly entered into a new or modified compromise agreement. The SC stated that even if it were to disregard the procedural infirmities, the petition would still fail. It discussed the validity of a new or modified compromise agreement entered into after a final judgment has attained finality

On the writ of execution to enforce the subject judgment. .Since the petitioners failed to perform valid consignation of payment, they would have still been in default under the terms of the new or modified compromise agreement, justifying the issuance of the writ of execution.

Their petition is DENIED. The CA Decision and Resolution are AFFIRMED.

\chapter{RULE 19 INTERVENTION }
\label{226c80c0-0a12-11ef-932c-63c852f65e48}


\ssection{GR213960 October 07, 2020 }
\label{fbd5e160-0a0f-11ef-932c-63c852f65e48}


\noindent\textbf{Philippine Reclamation Authority , Petitioner, v. \\Ria S. Rubin, Respondent. LAZARO-JAVIER, J.}\vspace{0.4cm}

The disputes primarily revolve around the ownership and possession of reclaimed land identified as Lot Nos. 1 and 2, located along the Manila Cavite Coastal Road, Las Piñas City. Various parties claim ownership through different means such as special land patents, miscellaneous sales applications, and subsequent transfers of titles. Presidential Decree Number 1085 in 1977, allowed the transfer of reclaimed land in Manila Bay to the ownership and administration of the Public Estates Authority ( PRA), which then submitted a survey plan for a special land patent on reclaimed land in Las Piñas City in 1988. PRA then entered into a Memorandum of Agreement with MERALCO in 1993, granting permission to construct and maintain a substation on a portion of the reclaimed land. Issues arose regarding the approval of survey plans and the subsequent issuance of miscellaneous sales patents to individuals over portions of the reclaimed land. Leading Ria S. Rubin to initiate legal action against MERALCO, for accion reinvindicatoria. 

The RTC was presented with disputes over the ownership and possession of reclaimed land, with PRA and Ria S. Rubin as parties. PRA sought to intervene in the case filed by Rubin against MERALCO, asserting its ownership rights over the disputed lots. However, the RTC denied PRA's motion to intervene. PRA appealed the RTC's decision, arguing that it had a legal interest in the case and should be allowed to intervene to defend its ownership rights over the disputed lots. The Court of Appeals affirmed the RTC's decision, stating that PRA had not shown a legal interest that would be directly affected by the judgment in the case. Further noted by the court is the fact that PRA had filed a separate case regarding the same subject matter, recognizing the jurisdiction of another branch of the RTC. This escalated the issue to the Supreme Court. 

\paragraph{Issue:}
\label{cc0acb10-09fd-11ef-932c-63c852f65e48}


The main issue to be resolved by the Supreme Court is whether petitioner's motion to intervene and admit answer-in-intervention is proper.

\paragraph{Decision:}
\label{ce125ae0-09fd-11ef-932c-63c852f65e48}


The Supreme Court reiterates that intervention is a remedy by which a third party, not originally impleaded in the proceedings, becomes a litigant to protect or preserve a right or interest that may be affected by such proceedings. This is governed by Rule 19 of the Rules of Court. The Court explains that the interest must be of a direct and immediate character, such that the intervenor will gain or lose by the direct legal operation of the judgment. The legal interest must be actual, material, and more than mere curiosity or sentimentality. And emphasized that permission to intervene is subject to the discretion of the court, considering whether the intervention will unduly delay or prejudice the adjudication of the rights of the original parties. 

The petitioner has a legal interest in the subject matter of the case, as its asserted rights and interests would be directly affected by the outcome of the litigation. However, the Court also considers whether the petitioner's rights may be protected in a separate proceeding, particularly the reversion case filed by the petitioner.The mentioned separate case filed by the petitioner in another branch, had already been resolved. The reversion case was ruled in favor of the petitioner , although the decision had not yet attained finality. Therefore, the SC concludes that both the trial court and the Court of Appeals correctly denied the petitioner's motion to intervene.

The petition is DENIED. The CA decision and resolution are AFFIRMED.

\ssection{GR230103 August 27, 2020 }
\label{00b23c10-0a10-11ef-932c-63c852f65e48}


\noindent\textbf{Martin Tirol, Petitioner, v. \\Sol Nolasco, Respondent. CAGUIOA, J.}\vspace{0.4cm}

The main dispute revolves around whether respondent Sol, as the widow of Roberto Jr., has the legal right to intervene in the probate proceedings for the estates of Gloria Tirol and Roberto Sr. She claims that as the surviving spouse of Roberto Jr., she is entitled to a portion of his share in his parents' estate. Petitioner Martin, the son of Roberto Jr., opposes her intervention, arguing that she has no legal interest in the probate proceedings. Respondent Sol files a motion for intervention, claiming a legal interest as the surviving spouse of Roberto Jr. RTC-101 granted Sol's motion for intervention in the intestate estate proceedings, and RTC-218 denied her motion for intervention in the probate proceedings. This issue reached and was resolved by the CA by granting respondent Sol's certiorari petition. 

The CA stated that Sol should be allowed to intervene in the probate proceedings. As the widow of Roberto Jr, she has an interest or claim in her husband's estate, which includes his share of his parents' estate. The CA annulled the RTC's resolution denying respondent Sol's motion for intervention and ordered RTC-218 to grant her motion and denied petitioner Tirol's motion for reconsideration. This now brings the case to the Supreme Court. 

\paragraph{Issue:}
\label{d0a69a50-09fd-11ef-932c-63c852f65e48}


Can Sol's alleged rights and interests over the estate of Roberto Jr. be fully protected by the intestate proceeding, and can her intervention in the probate proceeding cause undue delay and prejudice to the party of Tirol?

\paragraph{Decision:}
\label{d2542340-09fd-11ef-932c-63c852f65e48}


The SC ruled that respondent Sol's right or interest in Roberto Jr.'s estate can be fully protected in the Special Proceeding which involves the settlement of Roberto Jr.'s intestate estate. Therefore, her intervention in the probate proceeding for Gloria and Roberto Sr.'s estates, is unnecessary. Section 1, Rule 19 of the Amended Rules of Civil Procedure provides the criteria for intervention. Also cited is Ongco v. \\Dalisay, where it is elucidated that intervention as a remedy rather than a right, subject to the court's discretion and considering both legal interest and potential delay or prejudice to the original parties. 

Also, in emphasizing that only the court first taking cognizance of the settlement of a decedent's estate has jurisdiction to resolve the controversies, the SC denied Sol's intervention in the probate proceeding due to jurisdictional exclusivity and the potential enlargement of issues. Section 1, Rule 73 of the Rules of Court grants exclusive jurisdiction to the court first taking cognizance of the settlement of a decedent's estate. Section 1, Rule 90 of the Rules of Court specifies the distribution of residue and resolution of controversies over lawful heirs. 

Intervention is not intended to inject an independent controversy into a suit, especially if it enlarges issues or delays the proceedings.

The SC reversed the CA's decision and denied respondent Sol's motion for intervention in the Probate proceeding. The CA erred in allowing the intervention without considering jurisdictional issues and potential delay.

\chapter{RULE 23. SUBPOENA }
\label{ce9030e0-0a12-11ef-932c-63c852f65e48}


\ssection{GR240773 February 05, 2020 }
\label{041abb70-0a10-11ef-932c-63c852f65e48}


\noindent\textbf{Anselmo Anselmo, Heirs of Palomo, PETITIONERS, v. \\Sucere Foods Corporation, RESPONDENT. INTING, J.}\vspace{0.4cm}

This case revolves around the ownership history of the lots in question. Allegations of fraudulent acquisition of titles by the respondent. Plaintiffs filed a Complaint for Quieting of Title, Recovery of Possession, and Damages against the respondent and the Register of Deeds, Guiguinto, Bulacan. Respondent filed notices to take depositions, which were denied by the RTC. The RTC denied respondent's notices to take depositions, citing lack of merit and failure to comply with procedural requirements. This prompts Respondent to file additional notices to take depositions, which were again denied by the RTC. Respondent filed a petition for certiorari before the Court of Appeals (CA) to set aside the RTC's orders.

This resulted to a CA Appeal wherein it is questioned whether the RTC properly denied respondent's notices to take depositions, and whether the RTC's orders were in accordance with the Rules of Court.The CA granted respondent's petition for certiorari, ordering the RTC to allow the taking of depositions upon oral examination of certain individuals involved in the case. The CA ruled that the RTC's denial of respondent's notices to take depositions was not justified and that the respondent had complied with legal requirements. The CA emphasized the importance of promoting just, speedy, and inexpensive disposition of actions and proceedings. And this decision of the CA lead to an escalation tof the case to the Supreme Court.

\paragraph{Issue:}
\label{d4fe0d90-09fd-11ef-932c-63c852f65e48}


Clarity on the requirement to state the purpose for taking deposition in the notice under Rule 23 of the Rules of Court. Propriety of CA's decsision to sett aside the RTC order, denying respondent's notice to take deposition.

\paragraph{Decision:}
\label{d6bcd490-09fd-11ef-932c-63c852f65e48}


The SC reiterated the provision of Section 1, Rule 23 of the Rules of Court, which allows depositions to be taken without leave of court after an answer has been served. This provision facilitates the gathering of testimonies before trial for the just, speedy, and inexpensive disposition of actions. They cited Section 1, Rule 23 of the Rules of Court.

The SC rejected the argument made by petitioners that it is necessary to state the specific purpose or purposes of the deposition to ensure relevance and absence of privilege. The SC pointed out that Rule 23 does not mandate such a requirement. It emphasized that the notice need only include the time, place, and the names or descriptions of persons to be examined. This is in Section 15, Rule 23 of the Rules of Court. There is emphasis on the importance of allowing utmost freedom in the use of depositions, in line with the rules and jurisprudence. It encouraged the utilization of modes of discovery to gather information for the efficient resolution of cases.

The Court, clarified that depositions may be taken before any judge, notary public, or any person authorized to administer oaths, as stipulated in Section 10, Rule 23 of the Rules of Court. The trial courts cannot exclusively arrogate such duties to themselves.

The Court DENIES the petition and AFFIRMS the Decision and the Resolution of the Court of Appeals.

\ssection{GR240053 October 09, 2019 }
\label{07bc7340-0a10-11ef-932c-63c852f65e48}


\noindent\textbf{People, PETITIONER, v. \\Mary Jane Veloso and Julius L. Lacanilao, RESPONDENTS. HERNANDO, J.}\vspace{0.4cm}

The main dispute revolves around the recruitment and exploitation of Mary Jane Veloso by Cristina and Julius, leading to her conviction in Indonesia, and the consideration of depositions that were not part of a previous decision.

Julius and a Cristina, offered Mary Jane a job as a domestic helper in Malaysia. Mary Jane left the Philippines for Malaysia on April 21, 2010, upon arrival informed that the job intended for her was no longer available. Cristina sent Mary Jane to Indonesia for a holiday, where she was apprehended for allegedly carrying drugs. Mary Jane was convicted of drug trafficking in Indonesia and sentenced to death. Cristina and Julius were arrested in the Philippines and charged with human trafficking, illegal recruitment, and estafa. And Mary Jane provided a sworn statement implicating Cristina and Julius in her recruitment and exploitation.

Before RTC, the prosecution motioned to take Mary Jane's testimony through written interrogatories, subject to certain conditions outlined in its resolution dated August 16, 2016. And uit was granted.

Cristina and Julius appealed the RTC's ruling, arguing that the deposition should be made before and not during the trial. They contended that the use of written interrogatories would violate their right to confront the witness. The CA found grave abuse of discretion on the part of the RTC and reversed its decision. The CA held that the conditional examination of witnesses in criminal proceedings is primarily governed by Rule 119 of the Rules on Criminal Procedure, and the use of written interrogatories would violate the accused's right to confront the witness. The CA denied the OSG's motion for reconsideration in its subsequent resolution. Prompting the prosecutors to reach the Supreme Court for a resolution.

\paragraph{Issue:}
\label{da64a6e0-09fd-11ef-932c-63c852f65e48}


Can the SC supplement its final and executory decision from October 9, 2019, regarding the conduct of Mary Jane Veloso's deposition by written interrogatories? Should the conditions set by the Government of Indonesia, as stated in their letter dated December 4, 2020, be integrated into the previously finalized decision?

\paragraph{Decision:}
\label{dbdc52c0-09fd-11ef-932c-63c852f65e48}


The Court explains that judgments nunc pro tunc are meant to correct omissions or errors in recording judicial actions previously taken, not to alter substantive decisions or introduce new conditions. Emphasized is the principle that final and executory judgments are immutable and unalterable, except in limited circumstances such as correction of clerical errors, judgments nunc pro tunc, or void judgments.

The Urgent Omnibus Motion filed by the Office of the Solicitor General (OSG) seeks specific instructions on the conduct of Mary Jane Veloso's deposition, which the Court deems as not falling within the exceptions to the finality of judgments. The decision-making authority of the court is circumscribed by procedural rules and legal principles regarding finality of judgments.

The Court notes that there was no inadvertent omission in the previous decision regarding the conditions for Mary Jane Veloso's deposition. The conditions set by the Indonesian authorities, as stated in their letter dated December 4, 2020, were not part of the considerations at the time of the decision. The Court's decision-making process is based on the evidence and arguments presented before it, and any new developments or information cannot be retroactively applied to alter final judgments.

The Urgent Omnibus Motion is NOTED WITHOUT ACTION.

\ssection{GR197122 June 15, 2016}
\label{0b114390-0a10-11ef-932c-63c852f65e48}


\noindent\textbf{Ingrid Sala Santamaria and Astrid Sala Boza, Petitioners, v. \\Thomas Cleary, Respondent. LEONEN, J.}\vspace{0.4cm}

The main dispute revolves around whether a foreign plaintiff residing abroad, in this case, Thomas Cleary, who filed a civil suit in the Philippines, is allowed to take a deposition abroad for his direct testimony on the ground that he is "out of the Philippines" under Rule 23, Section 4(c)(2) of the Rules of Court.

Thomas Cleary, an American citizen residing in California, filed a Complaint for specific performance and damages against Miranila Land Development Corporation, Manuel S. Go, Ingrid Sala Santamaria, Astrid Sala Boza, and Kathryn Go-Perez before the RTC of Cebu. Cleary elected to file the case in Cebu, despite provisions in the Stock Purchase and Put Agreement allowing him to file it in California or the United States District Court for the Central District of California. Pre-trial proceedings commenced, and Cleary stipulated his intention to testify either on the witness stand or through oral deposition. Cleary filed a motion for court authorization to take a deposition before the Consulate-General of the Philippines in Los Angeles to be used as his direct testimony.

The RTC denied Cleary's motion for court authorization to take a deposition, stating that depositions are not meant to substitute for actual testimony in open court. The RTC held that as the plaintiff, Cleary should appear in court and testify under oath.

This resulted to an appeal in CA to resolve whether Cleary, as a foreign plaintiff residing abroad, is allowed to take a deposition abroad for his direct testimony. The CA granted Cleary's petition for certiorari and reversed the RTC's ruling, and held that Rule 23, Section 1 of the Rules of Court allows the taking of depositions, and Cleary's status as the plaintiff does not preclude him from this right. The CA emphasized that it is immaterial that Cleary is the plaintiff himself and denied reconsideration of its decision. With this decision the Supreme Court was tasked with resolving two main issues.

\paragraph{Issue:}
\label{deebb5a0-09fd-11ef-932c-63c852f65e48}


Can the limitations for the taking of deposition under Rule 23, Section 16 of the Rules of Court apply? Is the taking of deposition under Rule 23, Section 4(c)(2) of the ROC applicable to a non-resident foreigner plaintiff’s direct testimony?

\paragraph{Decision:}
\label{e12a8210-09fd-11ef-932c-63c852f65e48}


The ruling of the Supreme Court addressed these issues by examining the relevant provisions of the Rules of Court and pertinent jurisprudence. 

The Supreme Court emphasized that the Rules of Court allow utmost freedom in taking depositions to gather information for pending cases. Rule 23, Section 1 provides that the testimony of any person may be taken by deposition upon oral examination or written interrogatories at the instance of any party. Rule 23, Section 4 outlines the use of depositions, stating that they may be used for any purpose if certain conditions are met, including if the witness is out of the Philippines. Rule 23, Section 16 allows courts to issue protective orders for parties and deponents. However, such orders require notice and must be for good cause shown.

Also its worth noting that the concept of "good cause" was discussed, emphasizing that it means a substantial reason that affords a legal excuse. The burden is on the party seeking relief to show adequate reasons for the order.

The Supreme Court highlighted the importance of depositions as a method of discovery and presenting testimony, even after the commencement of trial. Rule 23 also addresses objections to the admissibility of depositions during trial, stating that objections may be made for reasons that would require exclusion if the witness were testifying in person.

The Petitions are DENIED for lack of merit.

\ssection{GR159127 March 3, 2008}
\label{0e59dee0-0a10-11ef-932c-63c852f65e48}


\noindent\textbf{Luis, Petitioner v. \\Honorable Rojas \\and Berdex International, Respondents. AUSTRIA-MARTINEZ, J.}\vspace{0.4cm}

Berdex International, Inc. filed a complaint against the petitioner for the recovery of a sum of money. The dispute arose from an alleged transaction where the petitioner received funds from Berdex, partly as advances or loans and partly for the purchase of shares in certain corporations. Berdex claimed that despite the funds being provided, the shares were not transferred to them, and the petitioner failed to repay the amounts owed.

Before the Regional Trial Court, Berdex alleged that the petitioner received funds from them but failed to fulfill their agreement regarding the purchase of shares and repayment of loans. However, the petitioner argued that the funds received were intended for specific purposes, including the purchase of shares and as advances for business operations. The Regional Trial Court granted Berdex's motion to take depositions through written interrogatories and the petitioner's motion for reconsideration was denied leading a the case to the CA.

At the CA The petitioner argue that the RTC erred in granting Berdex's motion for deposition through written interrogatories. The CA dismissed the petitioner's appeal due to non-compliance with procedural requirements. The petition lacked an affidavit of service and contained blurred annexes and the pleadings filed before the Regional Trial Court were not attached.

The Court of Appeals found that non-compliance with the requirements warranted dismissal of the petition. The petitioner's motion for reconsideration was also denied by the Court of Appeals which brings the issue to the Supreme Court.

\paragraph{Issue:}
\label{e61311c0-09fd-11ef-932c-63c852f65e48}


Did the CA commit grave abuse of discretion by dismissing the petition for certiorari despite the uniqueness of the legal issue raised by the petitioner? Is there potential injustice that would result if the deposition by written interrogatories of all witnesses, who are non-resident foreigners, is allowed?

\paragraph{Decision:}
\label{e8edf720-09fd-11ef-932c-63c852f65e48}


The Supreme Court found that the petitioner's resort to a petition for certiorari under Rule 65 was proper, considering the dismissal of the petition by the Court of Appeals. The failure to attach an affidavit of service was not fatal to the petition, as copies of the petition were personally served on the Regional Trial Court (RTC) and the private respondent's counsel, as evidenced by official receiving stamps. The explanation provided by the petitioner for the blurred copies of the annexes attached to the petition was found satisfactory by the Supreme Court.

The Court found that the relevant facts and arguments of the case were adequately presented through the documents attached to the petition, obviating the need to attach all other pleadings filed in the RTC. And the reliance of the Court of Appeals on Administrative Circular Number 3-96 was deemed misplaced, emphasizing the principle of substantial compliance and the importance of ensuring that rules of procedure do not frustrate the ends of justice.

The Court ruled that Section 1, Rule 23 of the Rules of Court allows any person's testimony to be taken by deposition upon oral examination or written interrogatories, irrespective of whether they are a non-resident foreign corporation.

And rejected the petitioner's argument that allowing deposition-taking would prevent the observation of witnesses' demeanor and curtail the right to cross-examine, emphasizing that depositions are consistent with the promotion of just, speedy, and inexpensive disposition of cases. The argument of potential challenges in enforcing claims against a non-resident foreign corporation is deemed conjectural.

The petition is GRANTED. The Resolutions of the Court of Appeals are REVERSED and SET ASIDE.

The Orders issued by the Regional Trial Court of Pasig City stands.

\ssection{GR210789 December 03, 2018 }
\label{11c125c0-0a10-11ef-932c-63c852f65e48}


\noindent\textbf{Roberto C. Martires, Petitioner, v. \\Heirs Avelina S. Somer, Responents. J. REYES, JR., J.}\vspace{0.4cm}

Avelina S. Somera alleged that she was the rightful owner of a parcel of land unlawfully transferred to petitioner Roberto C. Martires. Avelina then instituted a complaint for accion reivindicatoria and accion publiciana against the petitioner and others before the RTC. Avelina filed a motion on June 15, 2007, requesting the RTC to conduct depositions upon oral examination, which was granted on July 5, 2007. Depositions of Avelina and her witnesses were conducted on September 27 and 28, 2007, before the Vice-Consul of the Philippine Consulate in New York City.

Also Avelina filed a motion on February 3, 2011, to mark additional documentary evidence, including transcripts of the depositions. The RTC admitted the transcripts of the depositions as evidence, ruling that there was substantial compliance with the notice requirement. The admission of the transcripts was held despite the petitioner's objections. Petitioner filed a motion for reconsideration, which was denied. 

At the CA, Petitioner questions the admittance of the transcripts of the depositions as evidence. The CA dismissed the petition for lack of merit, affirming the RTC's decision. The CA held that petitioner waived any objections to the notice of deposition by unreasonably delaying filing objections for more than three years after becoming aware of the defect. Thus, Petitioner brought the issues to the Supreme Court.

\paragraph{Issue:}
\label{ec0aa070-09fd-11ef-932c-63c852f65e48}


Whether the Court of Appeals (CA) erred in affirming the admission of the complainant's depositions despite the alleged lack of reasonable prior notice in writing of the actual date and time of the taking of said depositions, as required by Section 15, Rule 23 of the Rules of Court.

\paragraph{Decision:}
\label{eec357d0-09fd-11ef-932c-63c852f65e48}


The SC discusses the purpose of depositions, which serve as a device for narrowing and clarifying the basic issues between parties and ascertaining facts relevant to those issues. It cites Section 1, Rule 23 of the Rules of Court, which allows testimony to be taken by deposition upon oral examination or written interrogatories.

While depositions are generally meant to inform parties of relevant facts, they may be admitted as evidence under certain conditions and for limited purposes, as specified in Section 4(c)(2), Rule 23 of the Rules of Court.

The Court also discussed the requirement of giving reasonable notice in writing for the taking of depositions, as stated in Section 15, Rule 23 of the Rules of Court. It emphasizes that the purpose of notice is to inform the other party about the intended proceedings.

Objections to errors and irregularities in the notice for taking a deposition are waived unless promptly served upon the party giving the notice, as per Section 29(a), Rule 23 of the Rules of Court. Objections to notice should be raised immediately upon receipt, and the possession of deposition transcripts is not a condition precedent for challenging the validity of the notice. 

The petition is DENIED. 

\ssection{GR217360 November 13, 2019 }
\label{159efc30-0a10-11ef-932c-63c852f65e48}


\noindent\textbf{BDO Strategic Holdings, Petitioner, v. \\ Asia Amalgamated Holdings, Respondent. REYES, A., JR., J.}\vspace{0.4cm}

On November 6, 2007, respondent filed a complaint against petitioners. Trial commenced with Mr. Jimmy Gow, majority shareholder of respondent company, as the first witness. On December 10, 2012, BDO Strategic Holdings requested a subpoena duces tecum, which was granted. Respondent intended to file an opposition and motion to quash that subpoena. Pending petitioners' opposition, BDO Strategic Holdings, Inc. filed written interrogatories addressed to respondent. The RTC quashed the subpoena duces tecum and ad testificandum and disallowed the written interrogatories, believing they would make Mr. Gow a witness for the adverse party.

This reached the Court of Appeals, and tasked to resolve whether the RTC erred in quashing the subpoena duces tecum and ad testificandum, and erred in disallowing the written interrogatories. The CA reversed the quashal of the subpoena duces tecum and ad testificandum and upheld the disallowance of the written interrogatories. Petitioners' filed a Motion for Partial Reconsideration regarding the denial of the request for written interrogatories and the CA denied the motion on March 10, 2015. This brings the issues to be resolved by the Supreme Court.

\paragraph{Issue:}
\label{f1367ab0-09fd-11ef-932c-63c852f65e48}


Propriety of CA's decision on affirming the disallowance of the written interrogatories addressed to the respondent.

\paragraph{Decision:}
\label{f3028820-09fd-11ef-932c-63c852f65e48}


The SC acknowledges the importance of depositions in facilitating the disposition of cases but emphasizes that the right to take depositions, whether oral or written interrogatories, has limitations. The Rules of Court expressly provides for limitations when the examination is conducted in bad faith, to annoy, embarrass, or oppress the person subject to the inquiry, or when the inquiry touches upon the irrelevant or encroaches upon the recognized domains of privilege, citing Rules of Court, particularly Section 16 of Rule 23

The court enjoys considerable leeway in matters pertaining to discovery. Section 16 of Rule 23 states that, upon notice and for good cause, the court may order a deposition not to be taken. The court exercises its judicial discretion to determine the matter of good cause. This exercise of discretion will not be set aside unless there is abuse or unless the disposition of matters of discovery was improvident and affected the substantial rights of the parties.

The SC finds no reason to reverse the ruling of the CA, affirming the RTC's decision to disallow the written interrogatories addressed to the respondent. Petitioners failed to establish that the disallowance by the lower court was arbitrary, capricious, or oppressive to warrant a reversal. And the respondent showed good cause for the disallowance. Considering that the case was already in the cross-examination stage, the use of written interrogatories would not serve its purpose anymore and would only cause further delay in the proceedings.

Petition for Review on Certiorari is DENIED.

\ssection{GR132577 August 17, 1999}
\label{18d7e010-0a10-11ef-932c-63c852f65e48}


\noindent\textbf{People, Petitioner, v. \\Hubert Jeffrey P. Webb, Respondent. YNARES-SANTIAGO, J.}\vspace{0.4cm}

The dispute revolves around the admissibility of taking oral depositions of certain witnesses residing in the United States, as requested by the respondent, Hubert Jeffrey P. Webb. This request was made during the proceedings of a criminal case for Rape with Homicide. Respondent Webb filed a Motion To Take Testimony By Oral Deposition, seeking to take the testimonies of several witnesses residing in the United States. The prosecution filed an opposition to the motion, contending that Rule 24, Section 4 of the Rules of Court has no application in criminal cases and that Rule 119, Section 4 only provides for conditional examination of witnesses before trial, not during trial. The trial court denied respondent Webb's motion, citing that it is not allowed by the relevant rules of court.

This is appealed at Court of Appeals, a challenge to the denial of respondent Webb's motion to take oral depositions and whether the trial court's orders constituted grave abuse of discretion. The CA granted the petition and annulled the orders of the trial court. It ordered the deposition of the witnesses to be taken before the proper consular officer of the Republic of the Philippines in Washington D.C. and California. And this CA decision is challenge at the Supreme Court.

\paragraph{Issue:}
\label{f62b8d80-09fd-11ef-932c-63c852f65e48}


Applicability of Rule 23 of the 1997 Rules of Civil Procedure to criminal proceedings. Can depositions be taken before a consular officer of the Philippines where the prospective witnesses reside or are officially stationed?

\paragraph{Decision:}
\label{f7bcdbe0-09fd-11ef-932c-63c852f65e48}


The SC found that the trial judge gravely abused her discretion in denying the motion to take the deposition of the witnesses for the petitioner (respondent in the trial court). While Rule 23, Section 1 is under the classification of Civil Procedure, its application is not limited to civil cases as long as it is not contrary to specific rules provided in criminal procedure. The SC ruled that depositions obtained during trial in a foreign state or country may be taken before a consular officer of the Republic of the Philippines where the deponent resides or is officially stationed. This was considered essential to protect the constitutional right of the accused to present evidence in his defense.

The SC emphasized that denying the deposition-taking would amount to the denial of the constitutional right to present evidence and for the production of evidence on behalf of the accused. Furthermore, the SC noted that the evidence sought through deposition would be merely corroborative or cumulative in nature, and the trial court exercised caution in its discretion, considering the ample evidence already presented by the defense.

The petition is hereby GRANTED. The decision of the Court of Appeals is REVERSED and SET ASIDE. The Regional Trial Court of Parañaque City is ordered to proceed posthaste in the trial of the main case and to render judgment therein accordingly.

\ssection{GR185527 July 18, 2012}
\label{1bb13ed0-0a10-11ef-932c-63c852f65e48}


\noindent\textbf{Harry L. Go, Tonny Ngo, Jerry Ngo, and Jane Go, Petitioners, v. \\People and Highdone Company et al., Respondents. PERLAS-BERNABE, J.}\vspace{0.4cm}

The petitioners are accused of conspiring to defraud Highdone Company Ltd., represented by Li Luen Ping, a frail old businessman from Cambodia, by falsely representing that they owned chattels in a certain property and executing a mortgage deed on those chattels, knowing that the property had already been encumbered and foreclosed by China Bank Corporation. Upon arraignment, the petitioners pleaded not guilty. The prosecution's witness, Li Luen Ping, traveled to the Philippines for a hearing but subsequent trial dates were postponed due to his unavailability. The prosecution filed a motion with the Metropolitan Trial Court (MeTC) to take the oral deposition of Li Luen Ping due to his health condition. MeTC granted the motion despite opposition from the petitioners, who theb filed a Petition for Certiorari before the RTC challenging the MeTC's decision. The RTC granted the petition and declared the MeTC orders null and void. Holding that the provision on the taking of depositions of witnesses in civil cases cannot apply to criminal cases.

This issue on deposition was brought before the CA, which inturn upheld the MeTC's decision to allow the deposition-taking of Li Luen Ping, stating that no grave abuse of discretion can be imputed upon the MeTC for allowing the deposition-taking. No rule of procedure expressly disallows the taking of depositions in criminal cases and that the petitioners would still have the opportunity to cross-examine the witness and make objections during the deposition. And this decision of the appelant court is eventually challenged at the Supreme Court.

\paragraph{Issue:}
\label{fa28abc0-09fd-11ef-932c-63c852f65e48}


Did the MetC by allowing the deposition of the complaining witness in Cambodia, infringe the constitutional rights of the petitioners to a public trial? Did the MeTC commit judicial legislation by applying the rules on deposition-taking in civil cases to criminal cases?

\paragraph{Decision:}
\label{fbfafac0-09fd-11ef-932c-63c852f65e48}


The examination of witnesses in criminal cases must be done orally before a judge in open court to ensure the accused's right to a public trial and to confront witnesses. This is in accordance with Section 14(2), Article III of the Constitution. Rule 119 of the Revised Rules of Criminal Procedure governs the conditional examination of witnesses for the prosecution. Section 15 of Rule 119 provides for the examination of witnesses before the court where the case is pending when it satisfactorily appears that a witness is too sick or infirm to appear at trial or has to leave the Philippines with no definite date of returning.

The CA's view that the petitioners would still have the opportunity to cross-examine the witness and raise objections during the deposition-taking overlooks the significance of face-to-face confrontation in ensuring the reliability of testimony in criminal trials. The right of confrontation is specifically intended to secure the accused in the right to be tried with witnesses testifying in open court. And the CA's reliance on the case of People v. \\Webb, which allowed deposition-taking in certain circumstances, is not applicable in this case where the prosecution seeks to depose a witness against the accused. 

The CA's disregard of the procedure under Rule 119 for taking the deposition of an unavailable prosecution witness constitutes grave abuse of discretion.

The petition is hereby GRANTED. The assailed Decision of the Court of Appeals are REVERSED and SET ASIDE. The Decision of the Regional Trial Court which disallowed the deposition-taking in Cambodia is REINSTATED.

\chapter{RULE 24. DEPOSITIONS AND DISCOVERY }
\label{ea0b6880-0a12-11ef-932c-63c852f65e48}


\ssection{GR209057 March 15, 2017}
\label{1f802120-0a10-11ef-932c-63c852f65e48}


\noindent\textbf{Renato S. Martinez, Petitioner v. \\Jose v. \\Ongsiako, Respondent. SERENO, C J.}\vspace{0.4cm}

On May 17, 2010, the Ongsiako filed a Petition before the RTC of Makati seeking permission to perpetuate his testimony under Rule 24 of the Rules of Civil Procedure. He cited reasons such as his health condition and the need to preserve his testimony for anticipated future suits involving properties in which he had an interest. And the Martinez filed a Comment/Opposition to the Petition, arguing against the proceedings on the grounds that estate proceedings were already pending before another branch of the RTC Makati. The RTC granted the Petition, allowing the perpetuation of the respondent's testimony. Various motions and hearings followed, including a motion for reconsideration by the expected adverse parties, which was denied by the RTC.

On August 18, 2010, the RTC issued an Order declaring that the petitioner and one of the expected adverse parties, respondent's brother Juan Miguel v. \\Ongsiako and Bank of the Philippine Islands, had waived their right to cross-examine the respondent, thus perpetuating his testimony. The RTC deemed the petition closed and terminated.

The decision of the Court of Appeals affirmed the RTC ruling, undermining petitioner's claime that the RTC had deprived him of the right to cross-examine respondent. CA emphasized the broad and liberal treatment of deposition-discovery rules and upheld the finding of waiver on the part of the petitioner and his counsel due to their failure to attend hearings without justification. The CA concluded that the petitioner's objections lacked merit and denied the appeal. And this eventually lead the petitioner to the SC, and assert that the CA erred. 

\paragraph{Issue:}
\label{fe966620-09fd-11ef-932c-63c852f65e48}


Correctness of the affirmed the RTC ruling that declared the petitioner to have waived his right to cross-examination. Is there error in in allowing the perpetuation of the respondent's testimony in a separate proceeding despite the pendency of a related estate case, thus allegedly permitting forum shopping?

\paragraph{Decision:}
\label{ffdd6510-09fd-11ef-932c-63c852f65e48}


The governing laws involved in this case are primarily the Revised Rules on Evidence, which govern the procedures for the perpetuation of testimony and the admissibility of depositions in court proceedings. Additionally, the principles of due process, as enshrined in the Constitution and upheld in various jurisprudence, guided the Court's decision-making process.

The court's rationale for the decision lies in the fundamental principles of due process, particularly the right to cross-examine opposing witnesses. The court found that the petitioner and his counsel were not properly notified of the hearing to cross-examine the respondent, which led to their absence. Despite the trial court's ruling that this absence constituted a waiver of the right to cross-examination, the Supreme Court disagreed. They emphasized that the failure to attend the hearing was not due to any fault on the part of the petitioner or his counsel. The court highlighted discrepancies in the notification process, including delays and incorrect addresses, which prevented the petitioner and his counsel from receiving the notices in a timely manner. As a result, the court concluded that the absence at the hearing did not signify an intention to waive the right to cross-examine the respondent. Therefore, they granted the petition, set aside the decision of the Court of Appeals, and remanded the case to the Regional Trial Court to allow the petitioner to conduct cross-examination.

The Petition for Review is GRANTED.

\chapter{RULE 25. INTERROGATORIES TO PARTIES }
\label{02ecbb60-0a13-11ef-932c-63c852f65e48}


\ssection{GR233395 January 17, 2018}
\label{22575d00-0a10-11ef-932c-63c852f65e48}


\noindent\textbf{Norlina G. Sibayan, Petitioner v. \\Elizabeth O. Alda, Respondent. VELASCO, JR., J.}\vspace{0.4cm}

Elizabeth charged Norlina with unauthorized deduction from her BDO Savings Account and failure to post certain check deposits. Elizabeth filed a complaint against Norlina with the Office of Special Investigation of the Bangko Sentral ng Pilipinas (OSI-BSP).The complaint alleged unauthorized deductions from Elizabeth's BDO Savings Account and failure to post certain check deposits. Norlina defended herself by claiming that the charges were retaliatory due to a previous criminal case filed by BDO against Elizabeth, Ruby, and others.

OGCLS-BSP found a prima facie case against Norlina for Conducting Business in an Unsafe or Unsound Manner under Section 56.2 of Republic Act Number 8791, punishable under Section 37 of Republic Act Number 7653. Subsequently, denied Norlina's motion for the production of bank documents and requests to answer written interrogatories.

This leads the aggrieved to appeal at Court of Appeals, whether the OGCLS-BSP committed grave abuse of discretion in denying Norlina's motions.

The CA upheld the OGCLS-BSP's rulings, stating that Norlina's persistence to utilize modes of discovery would delay the resolution of the case. The court found that the information Norlina sought to elicit through her motions was sufficiently contained in the pleadings and attachments submitted by the parties. The subsequently denied Norlina's motion for reconsideration, thus the issue reached the SC.

\paragraph{Issue:}
\label{03236850-09fe-11ef-932c-63c852f65e48}


Existence of substantial grounds for the denial of Norlina's requests to answer written interrogatories. Is Norlinaentitled to the production of bank documents pursuant to Section 1, Rule 27 of the Rules of Court?

\paragraph{Decision:}
\label{04d97cc0-09fe-11ef-932c-63c852f65e48}


No error in the ruling of the CA, technical rules of procedure and evidence are not strictly adhered to in administrative investigations. Administrative proceedings are summary in nature and not bound by strict adherence to the Rules of Court. OGCLS-BSP did not gravely abuse its discretion in denying Norlina's request for written interrogatories and motion for production of bank documents. The summary nature of administrative proceedings is meant to achieve expeditious and inexpensive determination of cases without strict adherence to technical rules. Technical rules of procedure and evidence applicable to judicial trials are not strictly applied in administrative proceedings. The denial of Norlina's motions did not violate her right to due process. 

Absent any showing that the OGCLS-BSP acted without jurisdiction or in excess thereof, its orders dispensing with the need to resort to modes of discovery may not be corrected by certiorari.

The petition is DENIED. 

\ssection{GR185145 February 5, 2014}
\label{27c01930-0a10-11ef-932c-63c852f65e48}


\noindent\textbf{Spouses Afulugencia, Petitioners, v. \\Metropolitan Bank Trust Co. and Emmanuel L. Ortega, Respondents. DEL CASTILLO, J.}\vspace{0.4cm}

The dispute revolves around a Complaint filed by the petitioners against Metrobank and Emmanuel L. Ortega seeking the nullification of mortgage, foreclosure, auction sale, certificate of sale, and other related documents, along with damages.

Petitioners filed a Complaint before the RTC of Malolos City, docketed as Civil Case Number 336-M-2004. Pre-trial proceedings took place, after which petitioners filed a Motion for Issuance of Subpoena Duces Tecum Ad Testificandum. Petitioners sought to have Metrobank's officers appear and testify as their initial witnesses during a hearing. They wanted these officers to bring documents related to their loan with Metrobank and the foreclosure proceedings concerning their property. The RTC denied petitioners' Motion for Issuance of Subpoena Duces Tecum Ad Testificandum on the grounds of a defective notice of hearing and the requirement for written interrogatories to be served upon adverse parties before compelling them to testify.

Petitioners appealed to the CA, arguing that their motion was not litigated and did not seek relief, but merely aimed for the issuance of a process. The CA dismissed the petition for lack of merit and affirmed the RTC's orders. It held that petitioners' motion was indeed a litigated motion, and a proper notice of hearing was required. Additionally, since petitioners failed to serve written interrogatories upon Metrobank, they could not compel Metrobank's officers to testify in court.Emphasizing that the Rules do not permit burdening the adverse party with courtroom appearances or other cumbersome processes if the party failed to seize the opportunity to elicit relevant facts through available means, such as written interrogatories. Petitioners' Motion for Reconsideration was also denied by the CA, opening the issues to be resolved by the SC.

\paragraph{Issue:}
\label{07a37230-09fe-11ef-932c-63c852f65e48}


Correctness of requiring notice and hearing for a motion for subpoena of respondent bank's officers when such requirements apply only to depositions under Rule 25 of the Rules of Court. Is it an error to hold that the petitioners must first serve written interrogatories to respondent bank's officers before they can be subpoenaed?

\paragraph{Decision:}
\label{0982e090-09fe-11ef-932c-63c852f65e48}


The SC clarified that while the lack of notice of hearing in the motion was cured by the filing of the opposition, the motion sought to compel Metrobank's officers to testify as the calling party's main witnesses, which is not permitted without serving written interrogatories under Rule 25 of the Rules of Court. Rhe purpose of requiring prior written interrogatories is to prevent fishing expeditions, needless delays, and unfair court practices. By compelling the adverse party to take the witness stand without serving written interrogatories, the calling party may weaken its own case and burden the court with unnecessary processes. The petitioners cannot build their case solely from the evidence of their opponent. The burden of proof and evidence falls on the petitioners, and they cannot pressure Metrobank to provide evidence for their case from the start, and allowing petitioners to call Metrobank's officers as their main witnesses and to access Metrobank's documentary evidence at the incipient phase of presenting their evidence-in-chief would go against the principles of justice and fair play. 

The SC concluded that the petitioners' move would have significant consequences. Such as Metrobank unwittingly admitting to allegations in the complaint. Therefore, the SC affirmed the CA's decision to deny the petition. The ruling is based on Rule 25, Section 6 of the Rules of Court, which provides that a party not served with written interrogatories may not be compelled by the adverse party to give testimony in open court or give a deposition pending appeal. Unless allowed by the court for good cause shown and to prevent a failure of justice.

The Petition is DENIED. 

\chapter{RULE 26. ADMISSION BY ADVERSE PARTY}
\label{2560ec20-0a13-11ef-932c-63c852f65e48}


\ssection{GR226130 February 19, 2018}
\label{2b3c2090-0a10-11ef-932c-63c852f65e48}


\noindent\textbf{Lilia S. Duque and heirs of Mateo Duque v. \\Spouses Bartolome, Respondents. VELASCO, JR., J.}\vspace{0.4cm}

The dispute revolves around the authenticity of the Deed of Donation over the subject property in favor of respondent Capacio, and the subsequent sale of a portion of the property to respondents Spouses Yu.

Spouses Duque allegedly executed a Deed of Donation over their property in favor of respondent Capacio on August 28, 1995. Respondent Capacio sold a portion of the property to respondents Spouses Yu. Spouses Duque filed a Verified Complaint before the RTC of Barili, Cebu, seeking the nullity of the Deed of Donation and Deed of Absolute Sale and the cancellation of Tax Declarations.

Before the Regional Trial Court, Spouses Duque claimed that the signature in the Deed of Donation was forged and sought the nullity of the documents related to the transfer of the property. Respondents Capacio and Spouses Yu refuted Spouses Duque's claims, with Capacio admitting the falsification but denying involvement, and Spouses Yu questioning Spouses Duque's personality to challenge the genuineness of the Deed of Absolute Sale. The RTC granted the demurrer to evidence filed by respondents Spouses Yu and dismissed the Complaint of Spouses Duque.

From that decision of the RTC, an issue issue appealed at CA - whether Spouses Duque's non-compliance with the order to comment on the documents filed by respondents Spouses Yu resulted in the implied admission of the authenticity of the Deed of Donation and other documents. The CA affirmed the RTC's dismissal of the Complaint, agreeing that Spouses Duque's failure to comply with the order to comment resulted in the implied admission of the authenticity of the documents filed by respondents Spouses Yu. The CA ruled that Spouses Duque failed to prove their case and their evidence was rendered worthless due to the implied admission of the authenticity of the documents. The aggrieved brought the issues to be resolve by the SC.

\paragraph{Issue:}
\label{0bf73bf0-09fe-11ef-932c-63c852f65e48}


Petitioner's failure to reply to the request for admission is tantamount to an implied admission of the authenticity and genuineness of the documents subject. Does the dismissal of the petitioners' Complaint based on an improper application of the rule on implied admission result in unjust enrichment at the latter's expense?

\paragraph{Decision:}
\label{0e6a5ed0-09fe-11ef-932c-63c852f65e48}


Failure to reply to a request for admission does not automatically result in an implied admission if the matters subject to the request have already been denied or controverted in previous pleadings. This principle is governed by Sections 1 and 2 of Rule 26 of the Rules of Court. Implied admission applies only when the matters have not been previously addressed or when there is no denial or specific response in previous pleadings.

The Supreme Court found that the petitioners' failure to respond to the request for admission was justified because the matters subject to the request had already been denied and controverted in their Verified Complaint. Therefore, there was no implied admission of the authenticity and genuineness of the documents. 

The trial court's grant of the demurrer to evidence was based on the erroneous application of implied admission. Since there was no implied admission, the demurrer to evidence should be denied, and the orders granting it were considered void. In this case, since the Supreme Court denied the demurrer to evidence, it proceeded to rule on the merits of the Complaint solely based on the evidence presented by the petitioners. 

The petition is GRANTED.

\ssection{GR231854 October 06, 2020 }
\label{2e4dcd60-0a10-11ef-932c-63c852f65e48}


\noindent\textbf{People, Petitioners, v. \\Leila L. Ang, Rosalinda Driz, Joey Ang, Anson Ang, and Vladimir Nieto, Respondents. CARANDANG, J.}\vspace{0.4cm}

The case involves allegations of Falsification of Public Documents, Malversation of Public Funds, and Violation of the Anti-Graft and Corrupt Practices Act against the respondents. The respondents were accused of defrauding and swindling the Development Bank of the Philippines (DBP) by falsely crediting cash deposits and concealing cash shortages at the DBP-Lucena City Branch.

On April 4, 2005, the Deputy Ombudsman for Luzon issued a Resolution finding probable cause to indict the respondents for various offenses related to the alleged fraud at the DBP-Lucena City Branch. Three separate Informations were filed by the OMB-Luzon before the RTC of Lucena, Branch 53. The cases were first handled by the Office of the City Prosecutor of Lucena City (OCP-Lucena). Requests for admission and motions were filed by both the prosecution and the defense regarding certain matters relevant to the case. The RTC considered requests for admission filed by both parties and ruled on their admissibility. The court denied the People's Requests for Admission and maintained its ruling that the implied admissions made by the respondents are regarded as judicial admissions in all related cases. 

At Sandiganbayan, the People filed a Petition for Certiorari (Rule 65), questioning the RTC's rulings regarding the implied admissions and the consolidation of the cases. The Sandiganbayan dismissed the petition, ruling that the RTC did not commit palpable errors in its decisions. The Sandiganbayan also noted infirmities in the petition itself, including deficiencies in verification and questions regarding the authority of the special prosecutor to file the petition. The People's motion for reconsideration was also denied, bringing the issues to the Supreme Court.

\paragraph{Issue:}
\label{11c1c730-09fe-11ef-932c-63c852f65e48}


Applicability of Request for Admission under Rule 26 of the Rules of Civil Procedure is in criminal proceedings. Is allowing the Request for Admission and deeming the facts stated therein as implied admissions by the People, correct?

\paragraph{Decision:}
\label{133aab90-09fe-11ef-932c-63c852f65e48}


The SC determined that a Request for Admission under Rule 26 of the Rules of Civil Procedure is not applicable in criminal proceedings. This is because criminal proceedings present inherent limitations for the use of Rule 26 as a mode of discovery, particularly considering the accused's constitutional rights, such as the right against self-incrimination. Allowing requests for admission in criminal cases would effectively compel the accused to testify against themselves, which is prohibited.

The SC further ruled that the trial court erred in allowing the Request for Admission filed by the accused, as it disregarded the rules and established jurisprudence. The joint orders of the trial court deeming the facts stated in the Request for Admission as implied admissions by the People were declared void.

The Decisions of the Sandiganbayan are REVERSED and SET ASIDE. The Joint Orders of the Regional Trial Court are declared VOID. 

\ssection{GR231854 October 06, 2020 - Inting J. Opinion}
\label{3171f0c0-0a10-11ef-932c-63c852f65e48}


Grounded in legal reasoning that Rule 26 requests for admission are not suitable for criminal cases. Justice Inting's concurrence showed the foillowing points:

1. **Request for Admission Inapplicable to Criminal Cases: Rule 26, allowing requests for admission, pertains only to civil actions. Applying it to criminal cases would violate the accused's constitutional rights, particularly the right against self-incrimination.

2. **Nature of Criminal Proceedings and Rights of the Accused: Criminal cases involve the accused's right against self-incrimination. Serving a request for admission would compel the accused to testify against themselves, violating this fundamental right.

3. Alternative Mechanisms for Admission of Facts: Rule 118 provides adequate mechanisms for obtaining admissions in criminal cases, such as pre-trial conferences, without infringing on the rights of the accused.

4. Consistency with American Jurisprudence: Refers to American jurisprudence, including a ruling from the Supreme Court of Indiana, which supports the notion that requests for admission are unnecessary in criminal cases due to existing mechanisms for obtaining admissions and the prosecutor's obligation to disclose exculpatory evidence.

\chapter{RULE 27. PRODUCTION OR INSPECTION\\ OF DOCUMENTS OR THINGS}
\label{3e5096e0-0a13-11ef-932c-63c852f65e48}


\ssection{GR204700 April 10, 2013}
\label{34346c20-0a10-11ef-932c-63c852f65e48}


\noindent\textbf{Eagleridge Development Corporation et al, Petitioners, v. \\Export and Industry Bank et al Respondent. LEONEN, J.}\vspace{0.4cm}

Petitioners Eagleridge Development Corporation, Marcelo N. Naval, and Crispin I. Oben filed a motion for production seeking the Loan Sale and Purchase Agreement (LSPA) dated April 7, 2006, along with its annexes and attachments, from respondent Cameron Granville 3 Asset Management, Inc. (Respondent). They wanted to inspect or photocopy the document. Respondent opposed the motion for production, arguing that it was filed out of time and that production of the LSPA would violate the parol evidence rule. Petitioners filed a rejoinder to respond to respondent's arguments. The case underwent pretrial proceedings from September 2005 to 2011, during which the trial court allegedly denied petitioners' motion for production as it was filed only during the trial proper.

In their motion for reconsideration before the Supreme Court, respondent raised the following arguements:

1)The motion for production was filed out of time; 2) Production of the LSPA would violate the parol evidence rule; 3) The LSPA is a privileged and confidential document.

Petitioners reasoned that their motion for production was not filed out of time and argued that the LSPA was relevant to the disposition of the case. They also disputed respondent's assertion that the LSPA was privileged and confidential.

\paragraph{Issue:}
\label{16b04a50-09fe-11ef-932c-63c852f65e48}


The timeliness of the motion for production . Is the LSPA is a privileged and confidential document, and its production of would violate the parol evidence rule?

Applicability of Article 1634 of the Civil Code applies to the case. Do petitioners have the right to obtain information about the transfer price of their loan obligation? And does parol evidence rule applies to petitioners?

\paragraph{Decision:}
\label{18c93f40-09fe-11ef-932c-63c852f65e48}


The SC stated that the motion for production can be availed of even beyond pre-trial upon a showing of good cause, as per Rule 27 of the Rules of Court. Article 1634 of the Civil Code, which governs the assignment of credits in litigation, is applicable to the case.

Petitioners have the right to extinguish their debt under Article 1634, and the 30-day period to exercise this right has not yet lapsed. The parol evidence rule does not apply to petitioners as they were not parties to the deed of assignment.

Alternative defenses are allowed under the Rules of Court, allowing petitioners to invoke Article 1634 while challenging the validity of the deed of assignment. 

The LSPA is not privileged and confidential in nature. The court found that the LSPA was relevant and material to the issue of the validity of the deed of assignment, which was raised by petitioners in the trial court. Additionally, the court held that the parol evidence rule did not apply to petitioners, who were not parties to the deed of assignment, and that the LSPA was not privileged and confidential.

SC DENIED the MR and affirmed its April 10, 2013 decision - which reversed and set aside the Court of Appeals' resolution.

\ssection{GR179786 24 July, 2013}
\label{39ec80d0-0a10-11ef-932c-63c852f65e48}


\noindent\textbf{Josielene Lara Chan, Petitioner, v. \\Johnny T. Chan, Respondent. Abad, J.}\vspace{0.4cm}

On February 6, 2006, Josielene filed a petition for the declaration of nullity of her marriage to Johnny at the Regional Trial Court (RTC) of Makati City. During the pre-trial conference, Josielene pre-marked a Philhealth Claim Form attached to Johnny's answer, indicating his forcible confinement at a rehabilitation unit of a hospital due to methamphetamine and alcohol abuse. On August 22, 2006, Josielene filed a request for the issuance of a subpoena duces tecum to obtain Johnny's medical records from his hospital confinement. Johnny opposed the motion, citing physician-patient privilege. The RTC denied Josielene's motion to submit the medical records as evidence and also denied her motion for reconsideration, citing physician-patient privilege.

This prompted Josielene to file a special civil action of certiorari before the Court of Appeals, alleging grave abuse of discretion by the RTC.

The CA denied Josielene's petition, ruling that allowing the production of medical records would violate physician-patient privilege. Although Johnny could waive the privilege, he did not do so in this case, and the Philhealth form attached to his answer was for a limited purpose. The denial presented issues to be resolved by the Supreme Court.

\paragraph{Issue:}
\label{1b9aaec0-09fe-11ef-932c-63c852f65e48}


Is there an error in ruling that the issuance of a subpoena duces tecum covering Johnny’s hospital records, on the ground that these are covered by the privileged character of the physician-patient communication?

\paragraph{Decision:}
\label{268392d0-0a16-11ef-932c-63c852f65e48}


The request for subpoena duces tecum covering hospital records is premature as objections to evidence must be made after the offer of such evidence for admission in court. The offer could be made part of the physician’s testimony or as independent evidence. Section 36, Rule 132 of the Rules of Court states that objections to evidence must be made after the offer of such evidence for admission in court.

Although Josielene's motion for the issuance of a subpoena duces tecum can be treated as a motion for production of documents, the documents to be disclosed must not be privileged. Section 1, Rule 27 of the Rules of Civil Procedure provides for the motion for production or inspection of documents.

SC denied the petition and affirmed the decision of the CA.

\ssection{GR204089 Digest July 29, 2015}
\label{93d651c0-0f88-11ef-8297-edd1c0998860}


\noindent\textbf{Borgoña et al, Petitioners v. \\Abra Valley Colleges, Respondent. BERSAMIN, J.}\vspace{0.4cm}

The dispute revolves around the petitioners' request to inspect the corporate books and records of Abra Valley Colleges, Inc. The petitioners, who are siblings, are seeking access to these documents as they claim to be bona fide stockholders of Abra Valley. The respondent, Francis Borgoña, who is also a sibling and the president of Abra Valley, contests their status as stockholders and their right to inspect the corporate records.

The petitioners filed a complaint against Abra Valley in the RTC, seeking to inspect its corporate books and records, and financial statements. The RTC rendered judgment in favor of the petitioners, ordering Abra Valley to allow them to inspect its corporate books and records, furnish them with financial statements, and pay attorney’s fees. Subsequently, the RTC denied Abra Valley’s motion for reconsideration.

Subsequently an appeal is made at CA, focused on the dismissal of the case by the RTC and the petitioners' failure to present sufficient evidence, particularly stock certificates, to support their claim as stockholders of Abra Valley.

On June 6, 2012, the CA denied the petitioners' appeal and affirmed the RTC's dismissal of the case. The appellant court upheld the ruling that the petitioners failed to present adequate evidence, such as formal stock certificates, to prove their ownership of Abra Valley stocks.

\paragraph{Issue:}
\label{341ac960-0f91-11ef-a737-5534cab7361b}


Whether the RTC properly dismissed the Special Civil Action Case on the ground of the petitioners’ failure to comply with the order issued by the RTC on March 8, 2010, to produce stock certificates.

Whether the petitioners were bona fide stockholders of Abra Valley Colleges, Inc., and thus entitled to exercise their rights as shareholders, including the right to inspect corporate books, demand financial statements, and participate in stockholders' meetings.

\paragraph{Decision:}
\label{364fd1d0-0f91-11ef-a737-5534cab7361b}


While possession of stock certificates is prima facie evidence of stock ownership, it is not the sole determining factor. Other evidence, such as official receipts of payments for shares, corporate documents showing issuance of shares, and minutes of meetings indicating participation as stockholders, can also establish stock ownership.

The respondents allowed the petitioners to become members of the Board of Directors, implying acknowledgment of their status as stockholders. The doctrine of estoppel applies, preventing the respondents from denying the petitioners' status as stockholders. And the petitioners were entitled to demand the production of the STB of Abra Valley to establish the validity of the transfer of shares. The failure to produce the STB cannot be a basis for dismissing the case, especially when the respondents also have access to the same documents. 

SC reversed the dismissal of the case by the lower courts, declared the petitioners as stockholders of Abra Valley, and ordered the reinstatement of the case for further proceedings.

\chapter{RULE 28 PHYSICAL AND MENTAL EXAMINATION\\ OF PERSONS}
\label{5ffd2a80-0a16-11ef-932c-63c852f65e48}


\ssection{GR130487 June 19, 2000 }
\label{319868e0-0a2e-11ef-a1a5-03b0bde1fccf}


\noindent\textbf{People, Plaintiff-appellee v. \\ROBERTO ESTRADA, Accused-appellant. PUNO, J.}\vspace{0.4cm}

The accused-appellant, Roberto Estrada y Lopez, was charged with the crime of murder for killing Rogelio P. Mararac, a security guard at the St. John's Cathedral in Dagupan City. Before the arraignment, accused-appellant's counsel filed a motion to suspend arraignment and to commit him to a psychiatric ward, alleging mental incapacity. The Presiding Judge assessed that the accused could answer intelligently, based on how he answered questions intelligently. This leads the Urgent Motion to Suspend Arraignment and to Commit Accused to Psychiatric Ward, to being denied. At the trial, after the prosecution rested its case, the accused-appellant's counsel filed a "Demurrer to Evidence." Among other reasons, they claimed that the prosecution did not have sufficient evidence to show that the accused-appellant was of sound mind. The prosecution, in their comment, stated that the mental condition of accused-appellant to stand trial had already been determined and unless a competent government agency certifies otherwise, the trial should proceed. The accused was not able to take the witness stand. Instead, his counsel presented Dr. Gawidan of Baguio General Hospital, testifying that accused-appellant had been confined at hospital suffering from Schizophrenic Psychosis. 

Ruling in favor of the People, accused-appellant was found guilty of murder. And the appeal to the RTC affirmed the trial court's conclusion, despite the accused-appellants arguemnt that the CLEAR AND CONVINCING EVIDENCE ON RECORD supports his plea of insanity. Thus prompting them to bring the issue before the Supreme COurt. 

\paragraph{Issue:}
\label{4602fc10-0f91-11ef-a737-5534cab7361b}


Whether there is an error in finding the accused-appellant guilty despite the CLEAR AND CONVINCING EVIDENCE ON RECORD, SUPPORTING HIS PLEA OF INSANITY.

\paragraph{Decision:}
\label{488bf0e0-0f91-11ef-a737-5534cab7361b}


The rules of criminal procedure allow for the suspension of arraignment if the accused appears to be suffering from an unsound mental condition that renders them unable to fully understand the charges and plead intelligently. This decision lies within the discretion of the trial court.

The accused's competency to stand trial is separate from the question of criminal responsibility at the time of the crime. A proper mental examination is crucial for determining both the accused's competency to stand trial and their sanity at the time of the offense. Neglecting to conduct such an examination violates the accused's right to due process and fair trial, necessitating a retrial with the proper evaluation of the accused's mental condition.

The Supreme Court vacated the decision of the trial court and remanded the case for further proceedings. 

\chapter{RULE 29 REFUSAL TO COMPLY \\WITH MODES OF DISCOVERY}
\label{6e276ad0-0a16-11ef-932c-63c852f65e48}


\ssection{GR182738 February 24, 2014 }
\label{ccb31830-0a2d-11ef-a1a5-03b0bde1fccf}


\noindent\textbf{CAPITOL HILLS GOLF AND COUNTRY CLUB INC. AND PABLO ROMAN JR., Petitioner v. \\MANUEL SANCHEZ, Respondent. PERALTA, J.}\vspace{0.4cm}

Sanchez filed a Motion for Production and Inspection of Documents; CHG, the party that is supposed to provide the information did not respond with an objection to produce and make available for inspection and photocopying by the plaintiff the following documents:

\begin{itemize}
  \item  List of stockholders of record
  \item Proxies, validated or not, received by them
  \item Specimen signatures of all stockholders
  \item Recording of a certain stockholders’ meeting
\end{itemize}

CHG filed a motion for reconsideration of the aforementioned order, which was denied. They also filed a motion for deferment of the implementation of the order. The inspection of documents faced delays and challenges. Upon motion, the court released the assailed September 3 Resolution, which state = "Failure of the defendants to comply with all the requirements of the order dated September 10, 2002 will result in this court citing all the defendants in contempt of court."

The petitioners appealed to the CA via a petition for certiorari, challenging the court orders. The petitioner cited Dee v. \\SEC, that the court has no authority to punish for disobedience of an order issued without authority. However, the CA denied the petition and the subsequent motion for reconsideration. Saying that the Dee case is inapplicable, since the resolution merely warned petitioners that they would be cited for contempt. The appelant court's decision lead the appellant to file a petition for review on certiorari under Rule 45.

\paragraph{Issue:}
\label{ee6ee210-0f96-11ef-a737-5534cab7361b}


Whether the Regional Trial Court (RTC) acted properly in threatening to penalize the petitioners for non-compliance with its order regarding the production and inspection of documents.

\paragraph{Decision:}
\label{ec7e83c0-0f96-11ef-a737-5534cab7361b}


The SC stated that a person who disobeys a lawful court order or commits improper conduct that obstructs the administration of justice may be punished for indirect contempt. They clarified that the procedures for initiating contempt proceedings, either through a court's initiative or through a verified petition. The court must follow due process, including providing the respondent with an opportunity to respond to the charges and conducting a hearing. The high Court provided an outline fior the proper remedy for appealing a judgment or final order in contempt proceedings, which involves filing an appeal under Rule 41 and posting a bond for suspension of the judgment's execution.

Still, the SC denied the petition and affirmed the decision of the Court of Appeals, upholding the RTC's resolution regarding the threatened sanctions for non-compliance with its order. The proceedings for indirect contempt have not been initiated. The assailed resolution is not yet a "judgment or final order of a court in a case of indirect contempt" as contemplated under the Rules. 

SC affirmed the RTC's authority to impose sanctions for non-compliance with its orders and clarifies the procedures and remedies available in contempt proceedings.

\chapter{RULE 30 TRIAL}
\label{759e4680-0a16-11ef-932c-63c852f65e48}


\ssection{GR227457 June 22, 2020 }
\label{77a828d0-0a2d-11ef-a1a5-03b0bde1fccf}


\noindent\textbf{HELEN L. SAY, GILDA SAY and DANNY L. SAY, Petitioner v. \\GABRIEL DIZON, Respondent. }\vspace{0.4cm}

Respondent filed a complaint for Declaration of Nullity of the Deed of Absolute Sale against Robert Dizon and petitioners before the RTC, which was then dismissed. On the ground of forum shopping because respondent had filed a similar complaint involving the same subject matter. After the dismissal order attained finality, petitioners filed an Ex-Parte Motion for Leave of Court to Set Defendants' Counterclaim for Hearing.

The RTC set hearing and petitioners filed their Judicial Affidavits beyond the five-day period prescribed by the Judicial Affidavit Rule, nevertheless it was admitted, citing the principle that technicalities must give way to substantial justice. Imposing a fine of P2,500.00 on petitioners for their late submission was deemed proper.

Respondent filed a motion for reconsideration, which was denied, promppting them to file a petition for certiorari under Rule 65. The Court of Appeals found that the RTC gravely abused its discretion in admitting the belatedly filed Judicial Affidavits of petitioners without proof of compliance with the conditions laid down under Section 10 (a) of the JAR. CA set aside the RTC's orders and held that petitioners failed to show valid reasons for the delay and failed to prove that the late submission would not unduly prejudice the opposing party. This escalated the case tp the Supreme Court.

\paragraph{Issue:}
\label{bb610560-0f9b-11ef-892d-31fd918e73ca}


Whether the RTC erred in admitting the belatedly filed Judicial Affidavits of petitioners without proof of compliance with the conditions laid down under Section 10 (a) of the JAR.

Whether the CA erred in finding grave abuse of discretion on the part of the RTC.

\paragraph{Decision:}
\label{b82e6310-0f9b-11ef-892d-31fd918e73ca}


True, the petitioners failed to comply strictly with the JAR, but the admission of their Judicial Affidavits would not unduly prejudice respondent, as the latter still had the opportunity to present rebuttal evidence.

The Supreme Court reiterated the principle that the admission of Judicial Affidavits beyond the prescribed period may be allowed once certain conditions are met, including valid reasons for the delay and absence of undue prejudice to the opposing party. The petitioners' delay in filing the Judicial Affidavits was an honest procedural mistake, and they submitted the affidavits before the scheduled hearing, demonstrating no deliberate intention to flout the rules. CA erred in finding grave abuse of discretion on the part of the RTC.

The Supreme Court granted the petition, reversing the CA's decision and reinstating the RTC's orders.

\ssection{GR241348 July 5, 2022 }
\label{79f80c40-0a2d-11ef-a1a5-03b0bde1fccf}


\noindent\textbf{LORETO A. CANAVERAS and OFELIA CANAVERAS, Petitioner v. \\JUDGE JOCELYN P. GAMBOA- DELOS SANTOS and RODEL MARIANO, Respondent. }\vspace{0.4cm}

The main dispute revolves around the criminal case for Falsification of Public Documents by a Private Individual filed against Loreto A. Cañaveras and Ofelia B. Cañaveras. The criminal was filed before MTCC San Fernando, Pampanga. The scheduled hearings took place, and during which the defense counsel was absent on certain occasions. The presiding Judge Gamboa-Delos Santos issued orders regarding the absence, ruling that the absence of counsel despite notice is construed as waiver to cross examine a witness. The motion for reconsideration by the defense counsel due to medical reasons, which was denied. As te trial proceed objections are raised by the defense counsel regarding the admissibility of evidence. However, the judge ruled that prosecutors are allowed to utilize the affidavit that was used before the Office of the City Prosecutor, which is provided under the Revised Guidelines for Continuous Trial of Criminal Cases. Defense argued that the rule does not apply when a private prosecutor handling the case. RTC ruled against the defense counsel's motions for reconsideration and allowed the trial to proceed.

Due to the importance of the issues raised in this case, particularly concerning the petitioners' liberty, the SC decides to relax the rule on hierarchy of courts and allows the petitioners' direct recourse to the SC.

\paragraph{Issue:}
\label{82874220-1146-11ef-a21b-81e54ec76734}


Whether the second sentence of Section 10(b) of the Judicial Affidavit Rule is unconstitutional. 

Whether the public respondent (Judge Gamboa-Delos Santos) committed grave abuse of discretion amounting to lack or excess of jurisdiction in issuing the questioned order dated June 6, 2018. 

Whether the public respondent committed grave abuse of discretion amounting to lack or excess of jurisdiction for allowing the presentation of a witness despite non-compliance with the Judicial Affidavit Rule. 

Whether the petitioners are entitled to the issuance of a Temporary Restraining Order (TRO) or a Writ of Preliminary Injunction.

\paragraph{Decision:}
\label{8005a050-1146-11ef-a21b-81e54ec76734}


The SC reminded trial judges to exercise discretion judiciously and to be mindful of the importance of procedural rules while also ensuring fairness and due process. It emphasizes that procedural rules should be interpreted liberally to promote the objective of achieving a just, speedy, and inexpensive determination of cases.

On the argument that the second sentence of Section 10(b) of the JAR is unconstitutional as it deprives them of their constitutional right to confront and cross-examine witnesses. The SC found that this issue is not raised before the trial court and is therefore not the "lis mota" or primary issue of the case. The Court declined to rule on this.

On the grave abuse of discretion by deeming their right to cross-examine, as it is waived due to the absence of their counsel. SC ruled that the trial judge acted within her discretion. Relying on the Revised Guidelines for Continuous Trial of Criminal Cases, the judge's decision was based on the lack of valid grounds for postponement and failure to comply with procedural requirements. While the trial judge's decision was upheld, the SC emphasizes the paramount importance of the petitioners' right to cross-examine witnesses.

The petition is **PARTLY GRANTED**. The Orders dated issued in the criminal case are **SET ASIDE** insofar as to have waived right to cross-examine.

The trial court is **DIRECTED** to proceed with the cross-examination and to continue the criminal proceedings with dispatch.

\ssection{GR182805 April 22, 2015 }
\label{7c3b93a0-0a2d-11ef-a1a5-03b0bde1fccf}


\noindent\textbf{HEIRS OF MABBORANG, Petitioner v. \\MABBORANG, Respondent. }\vspace{0.4cm}

Plaintiffs claimed entitlement to a share in several parcels of land left behind by Severino Mabborang and Maria Magabung, alleging that they are surviving heirs of their father, Rufino Mabborang, who they claim is one of the children of the spouses. Respondents countered that Rufino was not a child of the spouses but a grandson, being the son of their daughter, Sofronia Mabborang. They argued that the plaintiffs' claims are invalid as Sofronia had already received her share of the estate and subsequently sold it. Plaintiffs filed an action for Judicial Partition of Realty with Damages before the RTC of Tuguegarao, Cagayan. The RTC dismissed the case, finding that Rufino was not a child of the spouses, and therefore, the plaintiffs were not entitled to a judicial partition.

The issue on entitlement of the plaintiffs to judicial partition is appealed at CA. On which, the CA reversed the RTC's decision, stating that there was insufficient evidence to support the claim that the estate had been partitioned and that Sofronia had already sold her share. CA found that the plaintiffs, were entitled to a share in the estate of Severino Mabborang and Maria Magabung. The case was remanded to the RTC for the determination of the exact shares of the plaintiffs. CA affirmed this decision upon reconsideration, stating that the documents alleged to show Sofronia's disposition of her shares were not properly presented as evidence. This ruling was disputed and elevated to the SC. 

\paragraph{Issue:}
\label{9d632520-1149-11ef-a21b-81e54ec76734}


Whether the Court of Appeals (CA) decided the matter in accordance with the law despite the alleged waiver of evidence presentation by the plaintiffs.

Whether the CA departed from the accepted and usual course of judicial proceedings by affirming the judgment of the lower court.

\paragraph{Decision:}
\label{9b1a94b0-1149-11ef-a21b-81e54ec76734}


The SC emphasized that evidence not formally offered cannot be considered or assigned any evidentiary weight or value. It must be formally offered to be considered by the court. Section 34, Rule 132 ensures that the trial judge knows the purpose for which the evidence is presented and allows opposing parties to examine and object to its admissibility.

None of the conditions to relax formal offer of evidence were met in this case. Along with the requisite that the evidence must be duly identified by testimony and incorporated into the records of the case. The records of the case lacked documentary or testimonial evidence to support the petitioner's allegations regarding the disposition of Sofronia Mabborang's share in the subject properties.

The petitioners failed to formally offer the subject documents in evidence during the trial on the merits, and their failure to do so casts doubt on the authenticity and reliability of the documents. Therefore, the SC found no error in the refusal by the lower courts to give any probative value to the subject documents.

The petition is DENIED. The decision and resolution of the Court Appeals are AFFIRMED.

\chapter{RULE 31 CONSOLIDATION OR SEVERANCE}
\label{80ccabf0-0a16-11ef-932c-63c852f65e48}


\ssection{GR183211 June 5, 2009}
\label{1cbd0350-0a2d-11ef-a1a5-03b0bde1fccf}


\noindent\textbf{PNB, Petitioner v. \\GOTESCO TYAN MING \\DEVELOPMENT, INC. , Respondent. NACHURA, J.}\vspace{0.4cm}

The dispute revolves around the foreclosure proceedings filed by PNB against GOTESCO due to the latter's failure to fully pay the credit facility extended by PNB and other banks. Foreclosure proceedings were initiated by PNB, and the property in question was auctioned, with PNB emerging as the highest bidder. After the expiration of the redemption period without GOTESCO exercising its right of redemption, PNB consolidated the title in its name. GOTESCO filed a motion to consolidate the case involving the issuance of a writ of possession with its case for annulment of foreclosure proceedings, specific performance, and damages against PNB, which the RTC granted.

The RTC denied PNB's motion for reconsideration, anf emphasized that consolidation of cases involving the same parties and subject matter is discretionary with the trial court and becomes a matter of duty if two or more cases are tried before the same judge or filed with different branches of the same court and one of the cases has not been partially tried. And so, the RTC deemed it proper to treat the case for annulment of foreclosure proceedings as an opposition to the case for the issuance of a writ of possession, requiring PNB to present evidence first.

PNB filed a petition for certiorari with the CA to challenge the decisions, which is dismissed. The CA rejected PNB's argument that a petition for issuance of a writ of possession cannot be consolidated with an ordinary civil action. The RTC cited the express mandate of Section 1, Rule 31 in granting the motion for consolidation. The RTC did not commit grave abuse of discretion in consolidating the cases and requiring PNB to present evidence first. Additionally, PNB's motion for reconsideration is denied thus elevating the case to the Supreme Court.

\paragraph{Issue:}
\label{4d1595c0-1154-11ef-b8ef-a5537d38c9c8}

\begin{itemize}
  \item Whether the consolidation of the issuance of a writ of possession with the case for annulment of foreclosure prejudiced PNB's right to a writ of possession.
  \item Whether the consolidation of the cases was proper under the Rules of Civil Procedure.
  \item Whether the RTC and the CA erred in granting the motion for consolidation.
\end{itemize}


\paragraph{Decision:}
\label{4aa61c60-1154-11ef-b8ef-a5537d38c9c8}


The SC cited the requisites for consolidation laid down in Teston v. \\Development Bank of the Philippines, which include that the cases arise from the same act, event, or transaction, involve the same or like issues, and depend largely on the same evidence. Consolidation aims to avoid multiplicity of suits, guard against oppression or abuse, prevent delays, and simplify the work of the trial court.

While consolidation may be discretionary, it should not prejudice the substantial rights of any party, and consolidation should be denied if it would result in complications, delay, or prejudice to any party. The consolidation of PNB's petition with GOTESCO's complaint did not serve the purposes of consolidation. Instead, it delayed the issuance of the writ of possession and prejudiced PNB's right to take immediate possession of the property. The expiration of the redemption period, the purchaser's right to possession of the foreclosed property becomes absolute, and a pending action for annulment of foreclosure does not stay the issuance of the writ of possession. Thus the Court concluded that the consolidation of the cases in this instance was not proper and constituted an abuse of discretion by the RTC. 

SC ruled in favor of PNB, setting aside the CA's decision and the RTC's orders for consolidation. It directed that PNB's petition for issuance of a writ of possession and GOTESCO's complaint proceed and be heard independently in accordance with the rules and resolved promptly.

\ssection{GR199501 March 6, 2013}
\label{1eb26ab0-0a2d-11ef-a1a5-03b0bde1fccf}


\noindent\textbf{REPUBLIC}\noindent\textbf{, Petitioner}\noindent\textbf{ v. \\HEIRS OF ENRIQUE ORIBELLO, JR. and REGISTER OF DEEDS OF OLONGAPO CITY, Respondent. CARPIO, J.}\vspace{0.4cm}

In 1968, Valentin Fernandez occupied the parcel of land in Nagbaculao, Kalaklan, Olongapo City, by virtue of a Residential Permit issued by the government. Upon Valentin's death, his son, Odillon Fernandez, continued to occupy the property together with spouses Ruperto and Matilde Apog. Odillon Fernandez then sold the property to Mrs. Florentina Balcita, who later sold it to Oribello. Oribello filed a Miscellaneous Sales Application with the DENR, but it was denied as the land remained classified as forest land.

The property was declared open for disposition under the Public Land Act.

On March 27, 1990, the Director of Lands issued an Order for the issuance of a sales patent and OCT in favor of Oribello, which Matilde Apog and Aliseo San Juan protested against, alleging fraud and misrepresentation. This led to a complaint for reversion and cancellation of title filed by the Office of the Solicitor General before the Regional Trial Court of Olongapo. The RTC dismissed the consolidated cases without prejudice for non-substitution of the deceased plaintiff (Oribello) and his counsel. However, upon motion for reconsideration, the RTC allowed the continuation of the presentation of petitioner’s evidence.

Petitioner appealed to the Court of Appeals which was the denied, diting that the remedy of appeal was no longer available to petitioner as it had lost its right to participate in the proceedings by failing to question the trial court’s order, which declared it to have abandoned the case.

\paragraph{Issue:}
\label{2f4833f0-1158-11ef-b8ef-a5537d38c9c8}


Wether Interlocutory orders are not subject of appeal.

Wheter the consolidated cases, without any order of severance, cannot be subject of multiple appeals.

Wheter there can be no private ownership over an unclassified public forest.

\paragraph{Decision:}
\label{2ccc8590-1158-11ef-b8ef-a5537d38c9c8}


The Supreme Court agreed with petitioner that the 12 September 1997 Order was interlocutory in nature and therefore not appealable. They clarified that while the cases were consolidated for joint trial, each case retained its separate and distinct character, and therefore, severance was not necessary to appeal a judgment rendered in one case.

On consolidation of cases and multiple appeals. The Court clarified that while the cases were consolidated for joint trial, each retained its separate and distinct character. Therefore, the Court held that severance was not necessary to appeal a judgment rendered in one case.

However, the Supreme Court did not rule on the nature of the property, as fraud is a question of fact. The case was remanded to the trial court for further proceedings to determine whether fraud and misrepresentation occurred in the issuance of the sales patent.

The Court GRANTS the petition IN PART and SETS ASIDE the assailed Decision and Resolution of the Court of Appeals. The reversion case is remanded to the trial court for further proceedings. 

\chapter{RULE 32 TRIAL BY COMMISSIONER}
\label{8a735ff0-0a16-11ef-932c-63c852f65e48}


\ssection{GR122216 Digest March 28, 2001}
\label{b147b480-0a2c-11ef-a1a5-03b0bde1fccf}


\noindent\textbf{ALJEM’S CORPORATION, PACIFICO DIZON JR., Petitioner v. \\HON HILARIO MAPAYO and RUDY CHUA, Respondent. }\vspace{0.4cm}\noindent\textbf{MENDOZA, J.}\vspace{0.4cm}

The main issue in this case revolves around the confirmation and adoption by RTC of the report of a commissioner on the examination of Aljem's accounting records. Private respondent sued petitioner for a sum of money and damages, alleging that despite confirmed earnings from the logging operations of their joint venture, petitioner refused to pay him his share. The case was filed with the Regional Trial Court, Branch 8, Davao City. Private respondent alleged that the joint venture earned an income of P3,659,710 from January to August 1990, which petitioner failed to share with him. Petitioner disputed the amount claimed by private respondent, stating that the correct amount was only P2,089,141. A commissioner was appointed by the trial court to conduct an audit of petitioner's accounting records. interviewed petitioner's representative and private respondent and filed her report in court, which was adopted by the trial court. Petitioner objected to the commissioner's report, alleging discrepancies and procedural irregularities. The RTC confirmed and adopted the commissioner's report upon the appeal of the Petitioner. Then at then CA, the decision of the trial court to confirm and adopt. This brings the Petitioner at the Supreme Court.

\paragraph{Issue:}
\label{8a6cf520-115a-11ef-b8ef-a5537d38c9c8}


 Whether the commissioner should have conducted a formal hearing as directed by the order of the trial court.

Whether the lack of a formal hearing and swearing of witnesses before the commissioner deprived petitioner of due process.

Whether petitioner waived its right to object to the proceedings before the commissioner.

4. Whether the objections raised by petitioner were timely and properly made.

\paragraph{Decision:}
\label{8888a4c0-115a-11ef-b8ef-a5537d38c9c8}


The Supreme Court agrees with petitioner's contention that a formal hearing should have been conducted by the commissioner. Rule 33 of the 1964 Rules of Court, under which the case was decided, clearly indicates the necessity for a formal hearing and swearing of witnesses, as mandated by the trial court's order of reference.

The lack of a formal hearing deprived petitioner of the opportunity to object to the procedure followed by the commissioner and to challenge the disallowance of certain items in the computation of the corporation's assets. This constitutes a violation of due process. Petitioner did not waive its right to object to the proceedings before the commissioner, as the lack of a formal hearing prevented it from raising objections during the proceedings. The objections raised by petitioner were properly made during the hearing conducted by the trial court on the commissioner's report. The lack of a formal hearing before the commissioner rendered the proceedings null and void.

WHEREFORE, the ORDERS of the Court of Appeals are REVERSED, and this case is REMANDED to the trial court for further proceedings in accordance with law.

\chapter{RULE 33 DEMURRER TO EVIDENCE}
\label{939b4700-0a16-11ef-932c-63c852f65e48}


\ssection{GR187448 January 9, 2017 }
\label{2a073950-0a2c-11ef-a1a5-03b0bde1fccf}


\noindent\textbf{REPUBLIC, Petitioner v. \\DE BORJA, Respondent. CAGUIOA, J.}\vspace{0.4cm}

The dispute revolves around a Complaint filed by the petitioner Republic against the respondent for "Accounting, Reconveyance, Forfeiture, Restitution, and Damages". The complaint is aimed for recover ill-gotten assets allegedly amassed during the administration of the late President Ferdinand E. Marcos. Geronimo Z. Velasco, one of the defendants, was alleged to have diverted government funds during his tenure as President and Chairman of the Board of Directors of the Philippine National Oil Company (PNOC). Velasco's nephew, respondent De Borja, was implicated in the collection of address commissions allegedly diverted to a corporation controlled by Velasco. The Republic claimed that De Borja acted as Velasco's dummy, nominee, and/or agent for certain corporations.De Borja was alleged to have collected these address commissions on behalf of Velasco and to have acted as Velasco's dummy or agent for corporations controlled by Velasco. The Republic, presented Epifanio F. Verano as a witness to support its claims. The Sandiganbayan found that the evidence presented by the petitioner was insufficient to support a claim for damages against De Borja. The trial court granted De Borja's Demurrer to Evidence, stating a failure to establish the liability based on the testimony of Verano and an affidavit from another witness.

The petitioner appealed SB's decision at the CA, claiming that it erred in granting De Borja's Demurrer to Evidence and in denying its Motion for Reconsideration. The Court of Appeals affirmed the Sandiganbayan's decision, finding that the evidence presented was insufficient to establish De Borja's liability. This prompted the petitioner to escalate the issue to the Supreme Court.

\paragraph{Issue:}
\label{c30bd770-120e-11ef-aa24-9916ea601717}


Whether or not the SB committed reversible error in granting respondent De Borja's Demurrer to Evidence.

\paragraph{Decision:}
\label{c1365420-120e-11ef-aa24-9916ea601717}


The SC discussed the nature of a demurrer to evidence, which is a motion to dismiss on the ground of insufficiency of evidence. The Court also emphasized that factual questions are not within the scope of a petition for review under Rule 45. The failure of the petitioner Republic to strictly comply with procedural rules renders its petition dismissible, but the Court decides to dispense with these lapses in the broader interest of justice.

The burden of proof in civil cases lies on the plaintiff to establish their case by a preponderance of evidence. The demurrer to evidence is filed prior to the defendant's presentation of evidence, and its purpose is to expedite the termination of the case without the need for the defendant's evidence. The note that Reyes' affidavit was rendered inadmissible due to his non-appearance before the SB, making it hearsay evidence. As for the testimony on envelopes allegedly containing commissions, it is deemed to be lacking sufficient substance. Verano did not know the contents of the envelopes, never confirmed De Borja's receipt of them, and there was no direct evidence linking the latter to any wrongdoing.

The SC agrees with the SB's observation that Verano's testimony was speculative, conjectural, and inconclusive, failing to prove De Borja's involvement. The Court affirms the SB's decision to grant the Demurrer to Evidence.

\ssection{GR243163 July 10, 2019 }
\label{2c220300-0a2c-11ef-a1a5-03b0bde1fccf}


\noindent\textbf{INTERNATIONAL EXCHANGE BANK (Now UNION BANK), Petitioner v. \\JOSE CO LEE and ANGELA T. LEE, Respondent. LEONEN, J.}\vspace{0.4cm}

The dispute revolves around allegations of fraud and the subsequent legal actions taken by UnionBank (formerly iBank) against Christina T. Lee, Jeffrey R. Esquivel, Violeta T. Lee, Karin Tse Go, Jose Co Lee, and Angela T. Lee. The bank accused them of fraudulently transferring funds from the Forward Foreign Exchange Placement Accounts of iBank's clients to their own accounts. The Regional Trial Court (RTC) granted a Demurrer to Evidence filed by Jose Co Lee and Angela Lee, dismissing the complaint against them for insufficiency of evidence. The RTC reasoned that the evidence presented by UnionBank primarily implicated Christina, failing to sufficiently implicate Jose and Angela in the alleged fraud. Additionally, the RTC noted that UnionBank failed to provide a demand letter to Jose and Angela for the return of the funds allegedly credited to their account. UnionBank sought a Motion for Partial Reconsideration of the RTC's decision, which was denied. Subsequently, UnionBank filed a Notice of Appeal, followed by a Motion for Leave to Allow Notice of Appeal. However, due to delays in the RTC's response, UnionBank withdrew its Notice of Appeal.

UnionBank then filed a Petition for Certiorari with the Court of Appeals, contesting the RTC's decision. However, the Court of Appeals dismissed the petition, stating that UnionBank should have appealed the RTC's decision rather than seeking certiorari. The Court of Appeals also upheld the RTC's decision, finding no grave abuse of discretion in granting the demurrer to evidence. This prompted an appeal to the Supreme Court, with an arguement that certiorari was the appropriate remedy due to the RTC's delay in acting on its Motion for Leave to file its Notice of Appeal. UnionBank also contended that the demurrer to evidence was improperly granted, alleging sufficient evidence against Jose and Angela. 

\paragraph{Issue:}
\label{3efde540-1213-11ef-aa24-9916ea601717}


Whether filing a petition for certiorari under Rule 65 was the appropriate remedy to contest the Regional Trial Court's grant of the demurrer to evidence.

Whether the petitioner had provided sufficient evidence to warrant relief against the respondents in its complaint for sum of money and damages.

\paragraph{Decision:}
\label{4093ee90-1213-11ef-aa24-9916ea601717}


The Supreme Court ruled that filing a petition for certiorari under Rule 65 was indeed the proper remedy in this case, despite the alternative option of filing an appeal under Rule 41. This decision was based on the fact that the dismissal resulting from the grant of the demurrer to evidence was only as to some of the respondents, while the complaint remained pending against others. Therefore, the exception provided in Rule 41 applied, allowing for the use of a special civil action under Rule 65.

Regarding the sufficiency of evidence, the Supreme Court found that while the Court of Appeals had determined the petitioner's evidence to be insufficient to establish a claim against one of the respondents, there were grounds to dispute this conclusion. Specifically, evidence presented against one of the respondents, Jose Co Lee, was deemed by the Supreme Court to be sufficient to maintain a claim against him. However, there was no similar evidence against the other respondent, Angela, and therefore the grant of the demurrer to evidence as to her was affirmed. The Court cited Rule 33 which pertains to demurrer to evidence, outlining the procedure for a defendant to challenge the sufficiency of the evidence presented by the plaintiff.

Supreme Court partially granted the petition for review, reversing the decision of the Court of Appeals as it pertained to respondent Jose Co Lee and ordering him to return the amounts fraudulently transferred into his account, while affirming the decision regarding respondent.

\ssection{GR176694 July 18, 2014 }
\label{2e6deed0-0a2c-11ef-a1a5-03b0bde1fccf}


\noindent\textbf{GMA NETWORK INC, Petitioner v. \\CENTRAL CATV, INC., Respondent. BRION, J.}\vspace{0.4cm}

The petitioner filed a complaint with the National Telecommunications Commission (NTC) against the respondent, alleging that the respondent's solicitation and showing of advertisements in its CATV system infringed upon the television and broadcast markets, as prohibited by EO 205. The respondent argued that EO 436, issued by former President Fidel v. \\Ramos, allowed CATV providers to carry advertisements with consent from their program providers. The respondent filed a motion to dismiss by demurrer to evidence claiming that the evidence presented failed to show how the respondent’s acts of soliciting and/or showing advertisements infringed upon the television and broadcast market. NTC granted the respondent's demurrer to evidence and dismissed the complaint. Ruling that EO Number 436 clarified the term "infringement" in EO 205 and allowed CATV operators to show advertisements with consent from program providers.

At the Court of Appeals the petitioners raised a procedural issue- NTC's grant of the demurrer to evidence based on insufficiency of the complaint; and a substantive issue - Interpretation of EO Number 205 and 436, regarding the prohibition of infringement on television and broadcast markets. CA upheld the NTC ruling, stating that the NTC did not err in considering the respondent's evidence and that EO 436 merely clarified EO 205. The CA affirmed the NTC's dismissal of the complaint. The Petitioner escalated the issues to the Supreme Court.

\paragraph{Issue:}
\label{8858f840-1215-11ef-aa24-9916ea601717}


Whether the CA erred in affirming the NTC order granting the respondent's motion to dismiss by demurrer to evidence.

Whether the respondent is prohibited from showing advertisements under Section 2 of EO 205, in relation to paragraph 2, Section 3 of EO Number 436.

\paragraph{Decision:}
\label{868103f0-1215-11ef-aa24-9916ea601717}


The SC denied the petition, stating that the NTC's grant of the demurrer to evidence was proper. It clarified that a demurrer to evidence in administrative proceedings before the NTC is allowable, and the NTC correctly considered both the insufficiency of the allegations in the complaint and the insufficiency of the complainants' evidence. However, the SC found that the NTC erred in considering the respondent's evidence, specifically the certifications attached to the demurrer to evidence, which violated the petitioner's due process rights. The SC cited Rule 33 of the Rules of Court, which prohibits the consideration of the defendant's evidence in resolving a motion to dismiss based on a demurrer to evidence.

On EO 205 and 436, the court clarified the nature of the two EOs. The Court emphasized that EO Number 205 grants legislative authority to the NTC to issue implementing rules and regulations, which are binding and have the force of law, and pointed out that the NTC and the CA erroneously treated EO436 as a statute, leading to incorrect interpretations of the law. EO Number 436 did not amend MC 4-08-88, as erroneously concluded.

The governing rules in the case are the implementing rules and regulations embodied in NTC Memorandum Circular 4-08-88, particularly the "must-carry rule," which requires CATV operators to carry local TV broadcast signals without alteration or deletion. The act of showing advertisements does not constitute an infringement of the "television and broadcast markets" under Section 2 of EO 205.

SC affirmed the decision of the CA and upheld the respondent's right to show advertisements under the applicable laws and regulations.

\chapter{RULE 34 JUDGMENT ON THE PLEADINGS}
\label{9a8e4cb0-0a16-11ef-932c-63c852f65e48}


\ssection{GR229070 November 10, 2020 }
\label{46f86490-0a2b-11ef-a1a5-03b0bde1fccf}


\noindent\textbf{EUFEMIA ABAD et al, Petitioner v. \\ HEIRS OF ABAD, Respondent. CAGUIOA, J.}\vspace{0.4cm}

The dispute revolves around the ownership and possession of a parcel of land, which is part of a larger property covered by an Original Certificate of Title. After the death of the original owners, Spouses Miguel Abad and Agueda de Leon, the land was inherited by their three children: Dionisio, Isabel, and Enrique. 

A civil case(Civil Case Number 0591) for annulment of deed with damages was filed at RTC Branch 21 in Santiago City. This case was dismissed on the manifestation Dionisio and Isabel, a compromise agreement had been forged between them and Enrique. However, despite the agreement, the land was not physically partitioned, and TCT remained in Enrique's name. Subsequently, Isabel passed away, and her son, Jose Eusebio, inherited the contested lot through a deed of donation. 

Dispute arises again over the possession and ownership of the land, when Eufemia Abad the heir of Enrique, refused to give the title of the subject lot unless the loan incurred from the processing fee for the segregation was paid. In their answer, the Enrique heirs admitted that the TCT was registered in the name of Enrique and averred that the subject lot is exclusively owned by them through hereditary succession. They denied the rest of the allegations in the complaint for want of knowledge sufficient to form a belief with respect to the truth or falsity thereof. On January 26, 2016, respondents filed a motion for judgment on the pleadings, the counsel for petitioners interposed no opposition to the motion. The RTC ruled in favor of the respondents, ordering the Heirs of Enrique Abad and Eufemia Abad to comply with the partition agreement and honor the deed of donation in favor of Jose Eusebio. The trial court found that res judicata applied due to earlier proceedings(Civil Case Number 0591). And Despite the petitioners' motion for reconsideration, the RTC upheld its decision. Subsequently, the petitioners filed a petition seeking to overturn the RTC's ruling.

\paragraph{Issue:}
\label{954ee500-121a-11ef-aa24-9916ea601717}


Whether the RTC err in granting the respondents' motion for judgment on the pleadings.

Whether the doctrine of res judicata applies in this case.

Whether the petitioners' answer raised genuine issues and thus precluded judgment on the pleadings.

\paragraph{Decision:}
\label{976086f0-121a-11ef-aa24-9916ea601717}


The Supreme Court clarified that judgment on the pleadings is proper when an answer fails to tender an issue or admits the material allegations of the adverse party's pleading. However, in this case, the petitioners' answer did raise factual issues by denying the material allegations of the complaint. Therefore, the RTC's grant of judgment on the pleadings was deemed improper. Citing Rule 34 of the Rules of Civil Procedure, which provides guidelines for judgment on the pleadings.

It is also determined that the doctrine of res judicata did not apply in this case. Res judicata requires that the former judgment or order must be final, rendered by a court having jurisdiction over the subject matter and the parties, and must be a judgment on the merits. Since there was no final judgment on the merits in the prior case, res judicata was incorrectly applied by the RTC. Section 47, Rule 39 of the Rules outlines the effects of judgments or final orders, including the doctrine of res judicata.

Also emphasised, is that the petitioners' denial of the genuineness and authenticity of the documents in the complaint was sufficient to tender factual issues. The respondents were required to introduce evidence to establish the authenticity of the documents.

Supreme Court granted the petition, reversing the RTC's resolutions and directing the RTC to hear and decide the case on its merits with dispatch. The decision underscores the importance of giving parties the opportunity to establish their case on the merits rather than deciding based solely on technicalities.

\ssection{GR219509 January 18, 2017}
\label{4bd3add0-0a2b-11ef-a1a5-03b0bde1fccf}


\noindent\textbf{ILOILO JAR CORP., Petitioner v.\\ COMGLASCORP, Respondent. MENDOZA, J.}\vspace{0.4cm}

The dispute between Iloilo Jar Corporation (Iloilo Jar) and Comglasco Corporation/Aguila Glass (Comglasco) revolves around a lease contract for a portion of a warehouse building in Iloilo City. Comglasco requested the pre-termination of the lease contract due to economic difficulties, which Iloilo Jar rejected. Subsequently, Comglasco removed its stock, merchandise, and equipment from the leased premises and stopped paying rent. Iloilo Jar filed a civil action for breach of contract and damages against Comglasco before the RTC. The trial court granted Iloilo Jar's motion for judgment on the pleadings, finding that Comglasco's answer admitted the material allegations of the complaint and that its affirmative defense under Article 1267 of the Civil Code was inapplicable to lease contracts. The RTC ordered Comglasco to pay unpaid rentals, attorney's fees, litigation expenses, and exemplary damages. 

Comglasco appealed to the CA, arguing that judgment on the pleadings was improper because its answer tendered an issue and raised affirmative defenses. The CA reversed the RTC's decision, ruling that judgment on the pleadings was improper as Comglasco's answer raised factual issues and tendered an issue. The CA remanded the case to the RTC for further proceedings. Iloilo Jar then filed a petition, seeking the reversal of the CA's decision and the reinstatement of the RTC's ruling. The issues raised in the petition include whether a defense raised in the answer, which is not applicable to the case, can be considered as tendering an issue, and whether judgment on the pleadings is appropriate when the defense raised in the answer is not applicable to the cause of action stated in the complaint.

\paragraph{Issue:}
\label{11e2ea10-121d-11ef-aa24-9916ea601717}


Whether or not a judgment on the pleadings is appropriate and valid when the defense interposed by the defendant in the answer is not applicable as a defense to the cause of action as stated in the complaint.

\paragraph{Decision:}
\label{100e7830-121d-11ef-aa24-9916ea601717}


While acknowledging the importance of strict compliance with procedural rules, the Court recognized that in exceptional cases where substantial justice is at stake, procedural rules may be relaxed. In this case, there were procedural lapses, but the Court decided to entertain the appeal due to the merits of the case.

The Court clarified the distinction between judgment on the pleadings and summary judgment. It noted that judgment on the pleadings is appropriate when the answer fails to tender any issue, whereas summary judgment may be granted if there are no genuine issues of fact raised. In this case, the Court found that while Comglasco's answer raised an affirmative defense, there were no genuine issues of fact that required a full-blown trial.

The Court reversed and set aside the Decision and Resolution of the Court of Appeals. The Order of the Regional Trial Court,was affirmed with modification in that the award of exemplary damages and litigation expenses was deleted. 

\ssection{GR201427 March 18, 2015}
\label{5091d220-0a2b-11ef-a1a5-03b0bde1fccf}


\noindent\textbf{ADOLFO, Petitioner v. \\ADOLFO, Respondent. DEL CASTILLO, J.}\vspace{0.4cm}

Fe inherited the subject property which she then sold to her brother, who then mortgaged it to DB, ending foreclosure. DBP then sold the property to the Garcias, who then sold it back to Fe. At this moment of acquring the property back, Fe is has been married to Teofilio. This case involves two cases where the disputes revolves around the ownership of the mentioned property.

The most recent is filed on 2004, Civil Case Number MAN-4821, a petition for judicial separation of property against ny Teofilo B. Adolfo against his now estranged wife Fe. Teofilo claimed that they acquired the subject property during their marriage but Fe denied his co-ownership, asserting it as her paraphernal property inherited from her mother. 

Concurrently, there is a pending case (MAN-2683), filed in 1996, wherein FE’s sister Florencia and her husband Juanito Gingoyon (the Gingoyons) filed a case for partition with damages against her. arising from the non fulfilment the sale of the subject property. Fe claimed when the property was sold back to her, it is constituted conjugal property and since she did not sign the deed of sale, the sale was null and void. Back at MAN-4821, Teofilio submitted as part of his evidence the Fe's Answer to Gingoyon to support his conjugal property claim. The RTC granted Teofilo's motion for judgment on the pleadings, ruling that the subject property was conjugal and ordering its partition between the spouses. 

Fe filed her Appellant’s Brief at CA, where she argued that the trial court erred, since the issue of whether it constitutes conjugal or paraphernal property was still pending. The CA reversed the decision for MAN-4821, ruling that the trial court cannot treat the motion as one for summary judgment because the issue of property ownership was still pending in MAN-2683. Teofilio assailed this deciosn at the Supreme Court.

\paragraph{Issue:}
\label{54e43e40-122c-11ef-aa24-9916ea601717}


Whether there should be legal consequences of Fe's failure to reply.

Whether there are no issues left to resolve on MAN-4921, making judgment based on the pleadings proper.

\paragraph{Decision:}
\label{56d2c7d0-122c-11ef-aa24-9916ea601717}


They elaborated on the distinction between judgment on the pleadings and summary judgment. Judgment on the pleadings is appropriate when there is no ostensible issue raised by the answer. However, summary judgment is granted when there is no genuine issue as to any material fact and the moving party is entitled to judgment as a matter of law. In rendering the summary judgment, the trial court relied on the respondent's failure to reply to the petitioner's request for admission and other supporting evidence. However, the Supreme Court found that the trial court erred in not considering the pending appeal in a related case, which raised the issue of whether the subject property is conjugal or paraphernal. Therefore, it was premature to render judgment on the petitioner's motion for judgment on the pleadings. The Court also noted that the petitioner cannot repudiate or question the final and executory ruling of the CA in the related case, which declared the subject property as the respondent's paraphernal property.

Petition is hereby GRANTED. The Resolutions of the Regional Trial Court of Santiago City are REVERSED and SET ASIDE. The motion for judgment on the pleadings filed by the defendants therein is DENIED. The Regional Trial Court is divected to hear and decide the case on the merits with dispatch.

\chapter{RULE 35 SUMMARY JUDGMENTS}
\label{a0dc86e0-0a16-11ef-932c-63c852f65e48}


\ssection{GR215175 April 25, 2022 }
\label{f8183e50-0a29-11ef-a1a5-03b0bde1fccf}


\noindent\textbf{ALJEM’S CREDIT INVESTORS CORP., Petitioner v. \\SPS. CATALINA and PORFERIO BAUTISTA, Respondent. HERNANDO, J.}\vspace{0.4cm}

The petitioner alleged that a parcel of land owned by the spouses Bautista was mortgaged to it as security for a loan. When the spouses Bautista failed to pay the loan, the petitioner foreclosed the mortgage, leading to the consolidation of title to the property in the petitioner's name. Subsequently, the spouses Bautista offered to repurchase the property, and they entered into a Contract to Sell. However, the spouses Bautista failed to comply with the terms of the contract, resulting in its cancellation. Despite entering into another Contract to Sell, the spouses Bautista still failed to comply with the agreement. Spouses Bautista raised various defenses, including the assertion that the mortgage contract is void due to the lack of Porferio Bautista's conformity, the presence of pactum commissorium in the contract to sell, and contesting the imposition of high interest rates. The Regional Trial Court (RTC) denied the petitioner's Motion for Summary Judgment, holding that genuine issues of fact exist that necessitate a full-blown trial to resolve. Aggrieved by the RTC's decision, the petitioner filed a Petition for Certiorari and Prohibition before the Court of Appeals (CA). However, the CA affirmed the RTC's ruling, stating that the defenses raised by the spouses Bautista are triable issues, including the validity of the mortgage contract and the allegation of forgery. Despite the denial of its motion for reconsideration by the CA, the petitioner elevated the case to the Supreme Court, arguing that the CA erred in its ruling.

\paragraph{Issue:}
\label{6e5d35e0-122f-11ef-aa24-9916ea601717}


Whether the RTC properly denied the petitioner's Motion for Summary Judgment.

\paragraph{Decision:}
\label{70023350-122f-11ef-aa24-9916ea601717}


The Supreme Court affirmed the ruling of the CA, which upheld the RTC's denial of the Motion for Summary Judgment. The Court emphasized that the CA did not rule on the validity of the mortgage contract but merely discussed the legal basis for voiding contracts without spousal consent, which was relevant to the denial of the motion. The CA also did not address certain issues raised by the petitioner, such as accion publiciana and rescission of the contract to sell, because the focus was on the propriety of denying the motion. The requirements for granting summary judgment, which include the absence of genuine issues of fact. It emphasized that genuine issues of fact require presentation of evidence for resolution. The Court also clarified that the mere absence of specific denials in the answer does not automatically warrant summary judgment, as long as the defendant's denials are specific and comply with procedural rules.

The Court also explained that issues such as whether a contract constitutes an equitable mortgage or pactum commissorium, as well as allegations of forgery, are questions of fact 

Supreme Court affirmed the CA's decision and ordered the RTC to proceed with the trial promptly.

\ssection{GR224076 April 25, 2022 }
\label{fb229820-0a29-11ef-a1a5-03b0bde1fccf}


\noindent\textbf{REPUBLIC, Petitioner v. \\SUSAN DATUIN, et al., Respondent. LAZARO-JAVIER, J.}\vspace{0.4cm}

he dispute revolves around the cancellation and reversion of certain land titles issued to private individuals and corporations by the government. The petitioner, alleged that the lots covered by the titles were inalienable and should be reverted to the government based on a previous court decision and fraudulent transfers. The respondents, including Susan Datuin, Evelyn Dayot, and several corporations, countered that the titles were legally issued to them and the lands were already classified as alienable and disposable before their acquisition. The case was raffled to the RTC where the motion for summary judgment was filed by respondents. Due to the petitioner's failure to respond, respondent argued that the petitioner was deemed to have admitted all the allegations in the request for admission, as well as the authenticity of relevant documents, such as the DENR Certificate of Verification dated February 20, 2013. As such, there are no more dispute remaining that could be addressed through a trial. The RTC initially denied a motion for summary judgment, stating that conflicting claims and genuine issues of fact required a full trial. However, upon reconsideration, the RTC granted the motion for summary judgment deeming the petitioner to have admitted the material facts and authenticity of documents due to their failure to respond to a request for admission. The RTC concluded that no genuine issue existed and dismissed the complaint. The petitioner appealed to the CA via a petition for certiorari, alleging grave abuse of discretion by the RTC. However, the CA dismissed the petition, stating that a summary judgment can only be corrected through appeal or direct review, not through a petition for certiorari under Rule 65. This prompted the Republic to reach the SC for a decision on the issues.

\paragraph{Issue:}
\label{186e9910-1236-11ef-aa24-9916ea601717}


Whether the Court of Appeals correctly dismissed the petition for certiorari for being an improper remedy against the trial court's summary judgment in favor of the respondents.

Whether the trial court correctly deemed the Republic to have admitted the matters raised in respondents' request for admission and subsequently rendered summary judgment against it.

\paragraph{Decision:}
\label{1b0b15e0-1236-11ef-aa24-9916ea601717}


The Supreme Court clarified that while summary judgment cases are typically appealed through ordinary appeal to the Court of Appeals, certiorari can be an appropriate remedy if the trial court's actions amount to grave abuse of discretion or excess of jurisdiction. In this case, the trial court's violation of due process justified the use of certiorari. The Court determined that the trial court erred in deeming the Republic to have admitted all matters raised in the request for admission and subsequently granting summary judgment. It found that genuine issues remained unresolved, particularly regarding the classification of the land and the validity of the titles issued.

The trial court is also criticised for issuing simultaneous orders granting the respondents' motion for reconsideration, motion for summary judgment, and rendering summary judgment itself. It deemed this action as depriving the Republic of the opportunity to seek reconsideration before judgment was rendered, thus constituting grave abuse of discretion.

Supreme Court granted the petition, reversed the Court of Appeals' rulings, nullified the trial court's orders, and directed the trial court to reopen the case, conduct pre-trial and trial proper, and resolve the case on its merits with utmost dispatch.

\ssection{GR170658 June 22, 2011}
\label{fce61a10-0a29-11ef-a1a5-03b0bde1fccf}


\noindent\textbf{ANICETO CALUBAQUIB et al, Petitioner v. \\REPUBLIC, Respondent. DEL CASTILLO, J.}\vspace{0.4cm}

The main dispute revolves around the ownership and possession of a 5-hectare portion of land within a larger landholding that was declared a military reservation site through Proclamation Number 80. Petitioners claim to have occupied and cultivated this portion since the early 1900s, while the respondent asserts ownership over the entire landholding as per the OCT issued in their name. 

Respondent obtained an Original Certificate of Title (OCT) over the entire landholding. In 1995, respondent filed a complaint for recovery of possession against petitioners, alleging that they unlawfully occupied a 5-hectare portion of the landholding since 1992. Petitioners denied the allegation and asserted their ownership and possession of the disputed portion since the early 1900s. The RTC conducted pre-trial conferences and noted admissions of fact. Despite suggestions for a summary judgment, no motion was filed by either part. subsequently the trial court issued an order for summary judgment due to the failure of the petitioners to raise any issue. The RTC rendered a decision in 2004, dismissing petitioners' claim of possession of the subject property, stating that their right was not recognized by the government and there was no evidence of a recognized private right over the subject property.

Issues were appealed at Court of Appeals. The CA affirmed the RTC decision, stating that petitioners failed to establish their right to the subject property prior to the issuance of Proclamation Number 80. They also failed to provide clear and absolute evidence of compliance with the requirements for confirmation of an imperfect title. Therefore, the CA upheld the dismissal of petitioners' claim. Petitioners argue that they acquired ownership of the subject property through extraordinary acquisitive prescription by occupying it since the early 1900s. They request a remand to the trial court for the reception of evidence to support their claims.

\paragraph{Issue:}
\label{fc1f2180-1235-11ef-aa24-9916ea601717}


Whether the trial court's decision to render a summary judgment was proper.

Whether the trial court violated procedural requirements in rendering the summary judgment.

Whether the trial court's decision deprived the petitioners of their right to due process by not allowing them the opportunity to present evidence and prove their defense.

\paragraph{Decision:}
\label{fe948e50-1235-11ef-aa24-9916ea601717}


Summary judgments are proper when there is no genuine issue as to any material fact and one party is entitled to judgment as a matter of law. This is outlined in the Rules of Court and reiterated in the case of Viajar v. \\Estenzo.

The determination of the propriety of rendering summary judgments hinges on the existence of genuine issues of fact. The court must carefully examine the affidavits, depositions, and admissions submitted by the parties to ascertain whether such issues are genuine. The filing of a motion and the conduct of a hearing on the motion for summary judgment are crucial procedural requirements. Failure to comply with these requirements warrants the setting aside of the summary judgment.

In this case, the trial court proceeded to render summary judgment without either party filing a motion for it. Moreover, the respondent objected to the summary judgment and insisted on the existence of genuine issues of fact. Thus, it was improper for the trial court to persist in rendering summary judgment. The trial court's decision deprived the petitioners of their right to due process by making conclusions based on assumptions without giving them the opportunity to present evidence and prove their defense. This violated the petitioners' right to a trial where they can fully present their case.

SC granted the petition, set aside the summary judgment, and remanded the case to the trial court for trial, emphasizing the need for the presiding judge to proceed with dispatch.

\chapter{RULE 36 JUDGMENTS, FINAL ORDERS\\ AND ENTRY THEREOF }
\label{a9953160-0a16-11ef-932c-63c852f65e48}


\ssection{GR235586 January 22, 2020}
\label{a8b39b70-0a4c-11ef-a1a5-03b0bde1fccf}


\noindent\textbf{SPS. MILA YAP-SUMNDAD and ATTY. DALIGDIG SUMNDAD et al}\noindent\textbf{, Petitioner }\noindent\textbf{ v. \\FRIDAY’S HOLDINGS INC., Respondent. PERALTA, C. J.}\vspace{0.4cm}

The dispute in this case revolves around a forcible entry case filed by the respondent, who sought a decision in their favor, declaring them as the actual prior possessor and owner of the subject property and entitled to its continuous, exclusive, peaceful, and actual possession. The Municipal Circuit Trial Court (MCTC) ruled in favor of the respondent, finding them to be in better right to possession of the subject property prior to February 15, 2014. The MCTC directed the defendants to restore the respondent in peaceful possession of the property and to pay reasonable compensation for the use and occupation of the premises.

On appeal, the Regional Trial Court (RTC) affirmed the MCTC's decision with modification. The RTC directed the defendants to pay the respondent for reasonable compensation for lost profits equal to the reasonable daily rental income of the rooms involved in the forcible entry for a specified period.

The petitioners, dissatisfied with the RTC's decision, filed a Petition for Review with the Court of Appeals (CA). However, the CA dismissed the petition due to several procedural lapses and infirmities, including the failure to file the mandatory Certificate of Non-Forum Shopping, failure to indicate material dates in the petition, failure to pay corresponding lawful fees for injunctive relief, and failure to attach all relevant pleadings and documents. The CA also noted deficiencies in the verification and notarial certificate. Despite the petitioners' motion for reconsideration, the CA upheld its decision to dismiss the petition.

The case has now been elevated to the Supreme Court through a Petition for Review on Certiorari, challenging the CA's resolution denying the motion for reconsideration.

\paragraph{Issue:}
\label{59fbc1a0-1236-11ef-aa24-9916ea601717}


Whether the Court of Appeals erred in denying the petitioners' Motion for Reconsideration for being filed beyond the reglementary period.

\paragraph{Decision:}
\label{86019160-122e-11ef-aa24-9916ea601717}


The Court held that the Court of Appeals did not err in denying the petitioners' motion for reconsideration for being belatedly filed. The motion for reconsideration was filed 20 days beyond the fifteen-day reglementary period prescribed by Section 1, Rule 52 of the Rules of Court. The purpose of filing a motion for reconsideration within the period to appeal is to allow an inferior court to correct itself before review by a higher court. However, if the motion for reconsideration is filed beyond such period, it forecloses the right to appeal.

The Court emphasized that the relaxation of procedural rules should remain as the exception, and rules must be complied with for the orderly administration of justice. The Court refused to admit the motion for reconsideration despite the petitioners' admission that they received a copy of the Court of Appeals' resolution late, emphasizing that it is the counsel's duty to adopt and strictly maintain a system to ensure timely filing and service of pleadings. The Court rejected the argument for leniency based on the end view of giving substantial justice, stating that procedural rules cannot simply be set aside based on their non-observance potentially prejudicing a party's substantive rights.

Since the petitioners' Motion for Reconsideration was belatedly filed, the Court of Appeals' resolution became final and executory by operation of law. Consequently, the petitioners lost the right to seek reconsideration and to appeal to the Supreme Court.

Therefore, the Court denied the petition, affirming the resolutions of the Court of Appeals.

\ssection{GR232202 February 28, 2018}
\label{94987c90-1236-11ef-aa24-9916ea601717}


\noindent\textbf{DANIEL A. VILLAREAL JR, Petitioner v. \\METROPOLITAN WATERWORKS AND SEWERAGE SYSTEM, Respondent. TIJAM, J.}\vspace{0.4cm}

The case involved Metropolitan Waterworks and Sewerage System (MWSS) and Orlando Villareal (Orlando). MWSS filed a case for unlawful detainer against Orlando and others concerning premises located at Balara Filters, Quezon City. The MeTC initially dismissed MWSS's case for being prematurely filed and for lack of cause of action. On MWSS's appeal to the RTC, the court rendered a decision wherein it reversed the MeTC's judgment. MeTC issued a Writ of Execution, for the satisfaction of the RTC. Orlando then filed a Petition for Certiorari under Rule 65 with the RTC, which affirmed the Writ of Execution dated October 26, 2015, as well as the Sheriff's Notice to Vacate and Pay. The RTC found no merit in the Orlando's argument that the five-year period under Section 6, Rule 39 of the Rules of Court had been violated. The RTC determined that the execution of the judgment was not time-barred, as the delay was caused by Orlando's filing of a Comment/Opposition, which interrupted the running of the five-year period. Therefore, the RTC upheld the validity of the Writ of Execution and the Sheriff's Notice. This issue on Rule 39 was presented before the Supreme Court.

\paragraph{Issue:}
\label{9d655c40-123f-11ef-aa24-9916ea601717}


Whether the Regional Trial Court (RTC) erred in dismissing the petition based on an erroneous application of Rule 39, Section 6 of the Rules of Court and apparent ignorance of applicable jurisprudence.

\paragraph{Decision:}
\label{9ab9eb50-123f-11ef-aa24-9916ea601717}


The Court held that the RTC's dismissal was erroneous because it failed to correctly apply Rule 39, Section 6 of the Rules of Court and relevant jurisprudence. The execution by motion is only available if the enforcement of the judgment was sought within five years from the date of its entry. After the lapse of such time, execution may only be enforced by action. 

In this case, the RTC decision became final and executory on December 15, 2002, and MWSS filed a Motion for Issuance of Writ of Execution on May 17, 2004, within the five-year period. However, the MeTC issued the actual writ of execution more than 12 years after the judgment became final, which was beyond the five-year prescriptive period. Therefore, the writ of execution issued by the MeTC on October 26, 2015, was declared null and void.

The Court also rejected MWSS's argument that Orlando's filing of a Comment/Opposition to the Motion for Issuance of Writ of Execution caused the delay. It ruled that there was no valid reason to extend the five-year period for execution, as there was no evidence that the delay was caused by the judgment debtor's actions.

Supreme Court reversed the RTC decision and set aside the Writ of Execution, thereby granting relief to the petitioner.

\ssection{GR221978 Digest April 04, 2022}
\label{293df3b0-0a1d-11ef-932c-63c852f65e48}


\noindent\textbf{DMCI PROJECT DEVELOPERS, INC.}\noindent\textbf{, Petitioner }\noindent\textbf{ v. \\NELIA BERNADOS et al, Respondent. LOPEZ, J., J.}\vspace{0.4cm}

The dispute in this case revolves around the ownership of a parcel of land located in Taguig and was the subject of a labor case filed by Nelia Bernadas and others against Liberty Transport Corp, wherein rhe NLRC ruled in favor of Bernadas. Leading to the annotation of a Notice of Levy on the TCT, and an auction sale where Bernadas et al. emerged as the winning bidders. Subsequently, Bernadas et al. executed a Deed of Sale and/or Certificate of Redemption of Real Property, ceding the ownership of the subject lot to DMCI after receiving a sum of money. They also executed a Release and Quitclaim discharging the original owners and DMCI from liabilities arising from the labor case. However, the Bernadas party later filed a motion before the NLRC to nullify the Deed of Sale and/or Certificate of Redemption of Real Property and Release and Quitclaim, alleging that these documents were spurious and falsified. The NLRC, on January 4, 2011, granted the Bernadas party's motion, nullifying the said documents. DMCI appealed, but the NLRC affirmed its earlier decision. DMCI's subsequent motion for reconsideration was denied, and the NLRC's resolution became final and executory. 

Meanwhile, DMCI filed a complaint before the RTC for quieting of title, asserting its ownership over the subject lot. The RTC granted a temporary restraining order against Bernadas et al., enjoining them from implementing the levy on execution. 

DMCI challenged the NLRC ruling before the CA, arguing that the NLRC's Entry of Judgment dated July 19, 2011, had been recalled and reinstated, thus rendering the LRA's decision moot. . Petitioner argues that the Entry of Judgment does not dispense with the requirement of a writ of execution. The CA denied DMCI's petition, affirming the LRA's ruling. The CA held that the NLRC's subsequent Entry of Judgment issued on May 16, 2012, rendered the issue moot and academic. The CA further invoked the doctrine of immutability of judgments, stating that decisions that have acquired finality cannot be modified or altered. DMCI's motion for reconsideration was subsequently denied by the CA, prompting them to seek for a relief from the Supreme Court.

\paragraph{Issue:}
\label{adb4a2a0-1243-11ef-aa24-9916ea601717}


Whether the CA erred in upholding the Resolution and the Order of the LRA, on the ground that the Entry of Judgment rendered the issue moot and academic. 

\paragraph{Decision:}
\label{af957090-1243-11ef-aa24-9916ea601717}


The court emphasized the principle of the immutability of judgments. Once a decision attains finality, it becomes immutable and unalterable, except for certain exceptions like clerical errors or void judgments. The court found no such exceptions applicable in this case.

The petitioner argued that the January 4, 2011, Order should not be implemented or registered without a writ of execution. However, the court held that while a writ of execution is typically necessary for enforcing NLRC decisions, in this case, one had not yet been issued. The LRA's Consulta Number 5208 did not dispense with the requirement of the writ; it merely declared the property as registrable. The court clarified that the Registrar of Deeds, not the LRA, is responsible for the actual registration of instruments. The LRA's role is to assist different agencies in land registration proceedings. Once a decision becomes final and executory, the Registrar of Deeds has a ministerial duty to register the instrument.

Also, the court noted that the petitioner failed to file a timely appeal to the CA within the prescribed period. Consequently, the judgment became final and executory. Despite procedural errors, the CA did not dismiss the petition outright. However, the court highlighted that the CA could have done so, given the petitioner's failure to comply with procedural requirements.

SC denied the petitioner's appeal, affirming the CA's decision and the LRA's Resolution and Order. 

\ssection{GR220606 January 11, 2021}
\label{fe583390-0a1c-11ef-932c-63c852f65e48}


\noindent\textbf{MIGUEL GOCOLAY, Petitioner  v. \\MICHAEL BENJO GOCOLAY, Respondent. LEONEN, J.}\vspace{0.4cm}

The dispute revolves around whether Priscilla's criminal act, specifically her conviction for making false entries in Michael's birth certificate, constitutes a supervening event that modifies a final and executory judgment regarding the order to conduct DNA testing to establish Michael's paternity. 

The Regional Trial Court initially granted Miguel's motion to dismiss or recall the orders allowing DNA testing, citing Priscilla's conviction as a basis for discrediting Michael's birth certificate. However, the Court of Appeals reversed this decision, ruling that Priscilla's conviction did not invalidate the entire birth certificate but only the falsified entries related to her marital status with Miguel. The Court of Appeals also emphasized that Priscilla's testimony regarding her relationship with Miguel, along with the birth certificate, constituted sufficient prima facie evidence to support the DNA testing order. In response, Miguel argues that Priscilla's conviction should be considered a supervening event that affects the validity of the birth certificate as a whole, thereby undermining Michael's prima facie case for paternity.

The Court of Appeals upheld its decision, maintaining that Priscilla's conviction did not alter the validity of the birth certificate's other entries, particularly those identifying Miguel as Michael's father. Moreover, the court found that Priscilla's testimony provided additional support for Michael's claim of paternity.

\paragraph{Issue:}
\label{05a8d9d0-1245-11ef-aa24-9916ea601717}


Whether Priscilla Castor's conviction for making false entries in Michael Gocolay's Certificate of Live Birth constitutes a supervening event that warrants setting aside the final and executor RTC Orders for DNA testing of Michael and Miguel Gocolay.

\paragraph{Decision:}
\label{080a2260-1245-11ef-aa24-9916ea601717}


First, the conviction occurred before the RTC's orders became final and executory. 

Second, the conviction did not materially change the parties' situation or affect the execution of the judgment. The false entry in Michael's birth certificate, which claimed that Priscilla and Miguel were married, did not defeat Michael's claim for paternity and filiation. While it affected Michael's status, it did not negate his claim to be recognized as Miguel's nonmarital son. Moreover, Priscilla's testimony about her relationship with Miguel, which resulted in Michael's birth, provided additional support for Michael's claim.

Supreme Court affirmed the Court of Appeals' decision, denying the Petition and upheld the Decision and Resolution of the Court of Appeals.

\ssection{GR215615 December 09, 2020}
\label{38b57480-0a1d-11ef-932c-63c852f65e48}


\noindent\textbf{LILIA M. TANINGCO et al., Petitioners v. \\REYNALDO FERNANDEZ et al, Respondent. HERNANDO, J.}\vspace{0.4cm}

The dispute in this case revolves around a Complaint for Quieting of Title and/or Recovery of Possession and Ownership filed by the respondents against the petitioners. The Municipal Trial Court (MTC) ruled in favor of the respondents, ordering the petitioners to vacate a specific portion of the subject lot. The MTC denied the petitioners' Motion to Quash the Writ of Execution, stating that it was their ministerial duty to grant the writ due to the finality of the appellate court's decision. Petitioners also argued that there was no valid At the pendency of the aforementioned case the defendant Jose P. Taningco died, hence the also argued that there was no valid substitution of the defendant. Notwithstanding that the absence of a proper substitution will not nullify the trial court's jurisdiction, the MTC ruled that there is a proper Notice of Death and Substitution.

The petitioners approached the RTC for relief, which then dismissed the Petition for Certiorari, and denied the prayer for preliminary injunction and temporary restraining order (TRO). The RTC also denied their motion for reconsideration. At the CA, upon the petitioners appeal, the ruling of the lower court is affirmed. The appellant court found that the RTC did not gravely abuse its discretion and that there was no ground for the mandatory disqualification of the RTC judge. The CA reiterated its stance that notice to counsel is notice to the client and emphasized the importance of counsel's responsibility in handling cases diligently. Hence, petitioners filed a Petition for Review on Certiorari at the Supreme Court.

\paragraph{Issue:}
\label{54ba3750-1248-11ef-aa24-9916ea601717}


Whether CA committed grave error by not furnishing the petitioners with a copy of its Decision, and in not adequately addressing the principal issues raised in the petition.

Whether CA committed grave abuse of discretion in not ruling that there was no valid substitution of the deceased defendant, and that the MTC Decision along with its writ of execution and demolition, are void ab initio.

\paragraph{Decision:}
\label{521c8200-1248-11ef-aa24-9916ea601717}


The Supreme Court ruled against the petitioners, stating that their petition lacked merit. The Court emphasized that notice to counsel constitutes notice to parties, and in this case, the petitioners' counsel of record, was duly served with a copy of the Court of Appeals' Decision. Despite the petitioners' claim of non-receipt, the Court found no compelling evidence to refute the presumption of regularity in the service of notice.

The principle of the immutability of final judgments, emphasizing that once a decision attains finality, it cannot be modified except under exceptional circumstances such as clerical errors or void judgments. Since the petitioners failed to file a motion for reconsideration within the prescribed period, the Court of Appeals' Decision became final and immutable.

In conclusion, the Supreme Court denied the petition for review on certiorari and affirmed the resolutions of the Court of Appeals.

\ssection{GR178083 March 13, 2018}
\label{3ba03860-0a1d-11ef-932c-63c852f65e48}


\noindent\textbf{FLIGHT ATTENDANTS AND STEWARDS ASSOCIATION OF THE PHILIPPINE AIRLINES, INC,, Petitioners v. \\PHILIPPINE AIRLINE, Respondent. BERSAMIN, J.}\vspace{0.4cm}

This case revolves around Philippine Airlines (PAL) and the Flight Attendants' and Stewards' Association of the Philippines (FASAP). The dispute primarily concerns PAL's retrenchment of around 1,400 cabin crew members.

In the Labor proceedings, FASAP filed a complaint alleging that PAL's retrenchment was illegal. Scrutinizing if PAL's retrenchment was justified by severe financial losses and PAL followed fair criteria in selecting employees for retrenchment. The Labor Arbiter, National Labor Relations Commission (NLRC), and Court of Appeals (CA) ruled in favor of PAL.

The Supreme Court (SC) reversed the CA decision and found PAL guilty of unlawful retrenchment. PAL filed a Motion for Reconsideration, which was denied. PAL filed a Second Motion for Reconsideration, which was initially challenged by FASAP as a prohibited pleading. The high Court did grant PAL's motion for leave to file a second motion for reconsideration, stating that it could be allowed in the higher interest of justice. SC recalled previous resolutions and allowed PAL's Second Motion for Reconsideration, stating that the July 22, 2008 decision did not yet attain finality. The Supreme Court (SC), is primarily considering the validity of PAL's retrenchment, but there is one main procedural issues that needs to be resolved- the recall of a resolution which the petitioner claims a lacking legal basis.

\paragraph{Issue:}
\label{7eb66ab0-125c-11ef-aa24-9916ea601717}


The jurisdictional implications stemming from the Court en banc recalling a previous resolution issued by a division due to concerns over jurisdiction and interpretation of the IRSC. Does the the harmless error rule apply to such action?

\paragraph{Decision:}
\label{7c8a14d0-125c-11ef-aa24-9916ea601717}


On Judicial Notice and Previous Rulings. The SC acknowledges previous rulings regarding PAL's financial condition and the suspension of monetary claims during rehabilitation proceedings. Judicial notice is taken of these findings.

On the Validity of the October 4, 2011 Resolution. The resolution recalls the September 7, 2011 resolution of the Second Division. The petitioner challenges its validity, claiming it lacks legal basis. However, the Court explains that the recall was not a decision on the merits but an exercise of the Court's inherent power to recall orders and resolutions before finality. The Court explains that the requirement to state legal and factual bases for decisions, as per Section 14, Article VIII of the 1987 Constitution, applies only to decisions adjudicating on the merits of a case, not to recall orders.

The Court clarifies that the recall aimed to resolve jurisdictional conflicts arising from retirements and inhibitions of justices involved in the case. It was intended to ensure that the appropriate legal competence reviewed the pending motion for reconsideration.

The Court dismisses the application of the harmless error rule, stating it applies to errors during trial proceedings and does not affect the merits of the case. In this context, the error in raffling the case to the Second Division did affect the jurisdictional implications and merits of the case.

The Court confirms the recall of the September 7, 2011 resolution, asserting that the Court en banc assumed jurisdiction over the resolution of the merits of the case. The recall was made before finality, and no due process issue arose.

\ssection{GR173148 April 6, 2015}
\label{3e0acfc0-0a1d-11ef-932c-63c852f65e48}


\noindent\textbf{ELSA DEGAYO, Petitioner v. \\CECILIA MAGBANUA-DINGLASAN, Respondent. BRION, J.}\vspace{0.4cm}

The dispute revolves around the ownership of a parcel of land and whether a disputed area adjacent to it is an accretion to Lot Number 861 or an abandoned riverbed that belongs to the respondents.

Respondents filed a complaint against tenants of Lot Number 861 for ownership and damages. Degayo initiated a separate case against the respondents for declaration of ownership with damages. RTC Branch 27 ruled in favor of the respondents in Civil Case Number 16047 and RTC Branch 22 ruled in favor of Degayo in Civil Case Number 18328.

At Court of Appeals it is appealed whether the disputed property is an abandoned riverbed or an accretion to Lot Number 861. And whether the RTC Branch 27 decision in Civil Case Number 16047 is conclusive upon Degayo. CA reversed the RTC Branch 22 decision, declaring the disputed property as abandoned riverbeds belonging to the respondents. CA considered the RTC Branch 27 decision in Civil Case Number 16047 conclusive to the title of the thing, being an aspect of the rule on conclusiveness of judgment. Degayo's disagrees with the CA's declaration

\paragraph{Issue:}
\label{25e76b40-1258-11ef-aa24-9916ea601717}


Whether the Decision in Civil Case Number 16047 constitutes res judicata.

Whether there is an identity of parties between Civil Case Number 16047 and the present case.

Whether the principle of res judicata applies to the present case.

\paragraph{Decision:}
\label{23a8c5e0-1258-11ef-aa24-9916ea601717}


The Decision in Civil Case Number 16047 constitutes res judicata, meaning that it is conclusive and binding on the parties involved in subsequent cases involving the same subject matter. Res judicata prevents relitigation of the same issue, promotes judicial efficiency, conserves judicial resources, and provides stability to judicial decisions. The doctrine is codified in Section 47 of Rule 39 of the Rules of Court, which comprehends two distinct concepts: bar by former judgment and conclusiveness of judgment.

Conclusiveness of judgment applies when a fact or question has been judicially determined in a former suit by a court of competent jurisdiction and continues to bind the parties while the judgment remains standing. The Supreme Court finds that there is an identity of parties between Civil Case Number 16047 and the present case, as there exists a community of interest between Degayo and her tenants, who were respondents in Civil Case Number 16047. Even though Degayo was not formally made a party in Civil Case Number 16047, she had the fullest opportunity to assert her claim and participate in the proceedings, thus being bound by the judgment.

The Supreme Court rejects Degayo's objection to the Court of Appeals taking judicial notice of Civil Case Number 16047, as Degayo herself referred to it in her pleadings and appellee's briefs.

The CA's decision to dismiss the petition is affirmed, as res judicata applies, and there is no need to relitigate issues already decided in Civil Case Number 16047.

\ssection{GR167237 April 23, 2010}
\label{4fb499e0-0a1d-11ef-932c-63c852f65e48}


\noindent\textbf{ASSOCIATED ANGLO-AMERICAN TOBACCO CORP., Petitioner v. \\FLORANTE DY, Respondent. DEL CASTILLO, J.}\vspace{0.4cm}

The dispute revolves around the extrajudicial foreclosure of a mortgage bond executed by Paul Pelaez, Jr. in favor of Associated Anglo-American Tobacco Corporation, his employer, and the subsequent legal actions taken by the Pelaez spouses to stop the foreclosure and seek damages.

Paul Pelaez, Jr. executed a mortgage bond over his family's house and lot in favor of Associated Anglo-American Tobacco Corporation. The Corporation initiated extrajudicial foreclosure proceedings against Pelaez due to alleged default in remitting sales proceeds. Pelaez and his wife filed a Complaint before the RTC against the Corporation, Florante C. Dy, and Sheriff Virgilio S. Villar to stop the extrajudicial sale.

The RTC issued a restraining order and subsequently a writ of preliminary injunction to halt the extrajudicial sale. After a hearing, the RTC rendered a Decision in favor of the Pelaez spouses, ordering the defendants to pay damages, release the mortgage, and make the injunction permanent. The trial courtlater amended its Decision to increase the damages awarded to the Pelaez spouses. The RTC also denied the defendants' motion for reconsideration.

The defendants appealed to the Court of Appeals, challenging the RTC's Decision and subsequent orders. he CA upheld the RTC's Decision and amendments, finding no reason to disturb them. The CA dismissed the defendants' petition for certiorari for lack of merit, affirming the RTC's rulings.

\paragraph{Issue:}
\label{8c0267b0-1256-11ef-aa24-9916ea601717}


Whether the Court of Appeals (CA) committed grave abuse of discretion in holding that the trial court's decision was final and executory despite its modification by a subsequent order upon which an appeal was duly perfected.

Whether the CA erred in dismissing the petition for certiorari and prohibition filed by the petitioners and in finding that the proper remedy was an ordinary appeal.

Whether the CA erred in validating the trial court's order to release the mortgage and declaring the injunction permanent despite the perfection of an appeal.

\paragraph{Decision:}
\label{89eaab40-1256-11ef-aa24-9916ea601717}


The SC acknowledges that while certiorari is ordinarily not available when the appeal period has lapsed, exceptions exist when substantial justice demands it. In this case, the CA's dismissal of the petitioners' certiorari petition despite the timely filing of a notice of appeal amounted to an oppressive exercise of judicial authority, justifying the invocation of certiorari.

The SC clarifies that the February 7, 2001 order of the trial court, which amended certain aspects of its earlier decision, constituted a substantial amendment rather than a mere supplement. As such, an appeal from the amended decision must be deemed an appeal from the entirety of the integrated amended decision. The trial court erred in granting partial execution of its decision after the appeal had been duly perfected, as the motion for partial execution was filed beyond the court's jurisdiction and lacked sufficient grounds. Moreover, the subsequent orders implementing the partial execution were also null and void.

The SC grants the petition, reverses the CA's decision, declares the trial court's orders null and void, and orders the transmission of case records to the Court of Appeals for the appeal to proceed.

\chapter{RULE 37 NEW TRIAL OR RECONSIDERATION}
\label{c0145150-0a16-11ef-932c-63c852f65e48}


\ssection{GR241841 November 28, 2022}
\label{d6423660-124d-11ef-aa24-9916ea601717}


\noindent\textbf{HEIRS OF DIONISIO DELOY and PRAXEDES MARTONITO DELOY, Petitioners,v. VERNA R. BASA-JOAQUIN, Respondents. INTING, J.}\vspace{0.4cm} 

Dionisio and Isabel Deloy were registered owners of Lot Number 4012 which the sold into portions of 13 lots. This is to be part of subdivision plan that was approved by the Court of First Instance of Trece Martires on February 15, 1966. Dionisio passed away in 1985, and in 1989, Praxedes the spouse of Dionisio discovered the issuance of titles in the name of the Province of Cavite. Praxedes and other heirs filed a case for Annulment of Torrens Title and Deed of Conveyance against the Province of Cavite. The RTC ruled in favor of Praxedes and the heirs, declaring certain titles by defendants Province of Cavite and the Register of deeds for the Province of Cavite as null and void. The CA affirmed the RTC's decision.

Declaring the titles issued in their names as valid and subsisting. It ordered the cancellation of certain titles issued in the names of the heirs of Dionisio Deloy and Praxedes Martonito Deloy. The RTC also directed the Register of Deeds to cancel certain titles and tax declarations and awarded attorney's fees to the respondents. 

The Heirs Del Rosario and the Heirs of Maxima, alleged that the reconstitution affected the certificates of title issued in their respective names, which are purchases made by their parents from Dionisio in 1967. Prompting the filing of their respective petitions for quieting of title. Due to the Heirs of Deloy's repeated failure to appear in court they were precluded from presenting their evidence. The heirs filed for a Motion for New Trial which the RTC denied, paving the way for CA to decide on the issue. The CA affirmed the RTC's decision, denying the appeal filed by the Heirs of Deloy. The CA upheld the validity of the titles issued in the respondents' names and found no grounds to grant a new trial. This decision lead the isssues to be presented before the Supreme Court.

\paragraph{Issue:}
\label{e066a5e0-124d-11ef-aa24-9916ea601717}


Whether the CA erred in upholding the RTC's denial of petitioners motion for a new trial.

Whether the CA erred in affirming the RTC's ruling to grant respondents' petitions to quiet title.

\paragraph{Decision:}
\label{e24291d0-124d-11ef-aa24-9916ea601717}


Grounds for a motion for new trial include fraud, accident, mistake, or excusable negligence. The Heirs of Spouses Deloy's failure to receive notices of hearings was not considered as extrinsic or collateral fraud, accident, or mistake. The lack of proof of receipt of orders was not enough basis to overturn the default declaration.

The failure to appear in court without justification precluded the Heirs of Spouses Deloy from being heard, thus the motion for new trial was deemed bereft of merit.

On the calidity of Petition to Quiet Title. Requisites for an action to quiet title include legal or equitable title to the property and invalidity of the deed, claim, or encumbrance. Respondents were the registered owners of the disputed lots but their titles were derived from a void reconstituted title. Lack of evidence regarding the validity of the sale of the lots rendered the petitions for quieting of title without basis. Thus, the petitions for quieting of title were dismissed due to the inability to prove valid acquisition of ownership over the lots.

The petition was granted. The decision and resolution of the Court of Appeals were reversed and set aside. The petitions for quieting of title or removal of the cloud thereof and damages filed by the respondents were dismissed.

\ssection{GR180817 June 23, 2009}
\label{c73c25d0-0a1a-11ef-932c-63c852f65e48}


\noindent\textbf{MULTI-TRANS AGENCY PHILS. INC., Petitioner v. \\ORIENTAL ASSURANCE CORP, Respondent. CHICO-NAZARIO, J.}\vspace{0.4cm}

The dispute arose from a complaint for a sum of money filed by Oriental Assurance Corporation against Multi-Trans Agency Phils., Inc. and Neptune Orient Lines, Ltd. The complaint alleged negligence on the part of the defendants in handling and unloading a shipment, resulting in the non-delivery of one box of the shipment.

Oriental Assurance Corporation filed a complaint against Multi-Trans and Neptune before the RTC of Manila on July 22, 1997. The complaint alleged negligence on the part of the defendants in handling and unloading a shipment, which resulted in the non-delivery of one box of the shipment. Oriental made demands for payment, which were refused by Multi-Trans and Neptune, leading to the filing of the complaint. The RTC found Multi-Trans and Neptune solidarily liable to Oriental and ordered them to pay ₱256,937.03 with legal interest, attorney's fees of ₱30,000, and costs.

Multi-Trans and Neptune appealed the decision of the RTC. Multi-Trans filed a motion for new trial and to admit attached answer and Neptune filed a motion for reconsideration of the decision.

The Court of Appeals denied the appeal of Multi-Trans and affirmed the RTC's decision holding Multi-Trans liable. The appelant court granted the appeal of Oriental, modifying the RTC's decision to hold only Multi-Trans liable, not Neptune. Both Multi-Trans and Neptune filed motions for reconsideration, which were denied by the Court of Appeals.

\paragraph{Issue:}
\label{fa5d3fb0-1255-11ef-aa24-9916ea601717}


Whether the negligence of petitioner's former counsel amounted to such gross negligence that it deprived petitioner of its day in court, thus violating its right to due process.

Whether petitioner should be granted a new trial based on the gross negligence of its former counsel.

Whether petitioner should be held liable for damages despite its assertion that it was not the agent of the vessel "Tokyo Bay."

\paragraph{Decision:}
\label{f81ec160-1255-11ef-aa24-9916ea601717}


The Supreme Court acknowledges that while the negligence of counsel generally binds the client, exceptions exist, particularly in cases of gross negligence that result in the deprivation of due process for the client. (Citing Section 1 of Rule 37 of the 1997 Revised Rules of Civil Procedure). The SC finds the negligence of petitioner's former counsel to be gross, resulting in petitioner's deprivation of its day in court and violation of due process. The SC emphasizes that a client may reasonably expect that their counsel will protect their interests during the trial, and the negligence of counsel may warrant relief for the client. The trial courts should be liberal in setting aside orders of default and granting motions for new trial if the defendant appears to have a meritorious defense, as parties must be given every opportunity to present their side. 

SC orders that the case be remanded to the Regional Trial Court for a new trial, with instructions to admit petitioner's Answer and to receive petitioner's evidence. 

The former counsel for petitioner is required to show cause why they should not be held administratively liable for their acts and omissions. 

\chapter{RULE 38 RELIEF FROM JUDGMENTS, ORDERS,\\ OR OTHER PROCEEDINGS}
\label{ca1448e0-0a16-11ef-932c-63c852f65e48}


\ssection{GR209057 March 15, 2017}
\label{4585f890-0a1a-11ef-932c-63c852f65e48}


\noindent\textbf{RENATO S. MARTINEZ, Petitioner v. \\JOSE MARIA v. \\ONGSIAKO, Respondent. SERENO, C J.}\vspace{0.4cm}

On May 17, 2010, the Ongsiako filed a Petition before the RTC of Makati seeking permission to perpetuate his testimony under Rule 24 of the Rules of Civil Procedure. He cited reasons such as his health condition and the need to preserve his testimony for anticipated future suits involving properties in which he had an interest. And the Martinez filed a Comment/Opposition to the Petition, arguing against the proceedings on the grounds that estate proceedings were already pending before another branch of the RTC Makati. The RTC granted the Petition, allowing the perpetuation of the respondent's testimony. Various motions and hearings followed, including a motion for reconsideration by the expected adverse parties, which was denied by the RTC.

On August 18, 2010, the RTC issued an Order declaring that the petitioner and one of the expected adverse parties, respondent's brother Juan Miguel v. \\Ongsiako and Bank of the Philippine Islands, had waived their right to cross-examine the respondent, thus perpetuating his testimony. The RTC deemed the petition closed and terminated.

The decision of the Court of Appeals affirmed the RTC ruling, undermining petitioner's claime that the RTC had deprived him of the right to cross-examine respondent. CA emphasized the broad and liberal treatment of deposition-discovery rules and upheld the finding of waiver on the part of the petitioner and his counsel due to their failure to attend hearings without justification. The CA concluded that the petitioner's objections lacked merit and denied the appeal. And this eventually lead the petitioner to the SC, and assert that the CA erred. 

\paragraph{Issue:}
\label{704c1c80-124e-11ef-aa24-9916ea601717}


Whether the Court of Appeals (CA) correctly affirmed the Regional Trial Court's (RTC) ruling that declared the petitioner to have waived his right to cross-examination.

Whether the RTC erred in allowing the perpetuation of the respondent's testimony in a separate proceeding despite the pendency of a related estate case, thus allegedly permitting forum shopping.

\paragraph{Decision:}
\label{6ddc7c10-124e-11ef-aa24-9916ea601717}


The court's rationale for the decision lies in the fundamental principles of due process, particularly the right to cross-examine opposing witnesses. The court found that the petitioner and his counsel were not properly notified of the hearing to cross-examine the respondent, which led to their absence. Despite the trial court's ruling that this absence constituted a waiver of the right to cross-examination, the Supreme Court disagreed. They emphasized that the failure to attend the hearing was not due to any fault on the part of the petitioner or his counsel. The court highlighted discrepancies in the notification process, including delays and incorrect addresses, which prevented the petitioner and his counsel from receiving the notices in a timely manner. As a result, the court concluded that the absence at the hearing did not signify an intention to waive the right to cross-examine the respondent. Therefore, they granted the petition, set aside the decision of the Court of Appeals, and remanded the case to the Regional Trial Court to allow the petitioner to conduct cross-examination.

The Petition for Review is GRANTED.

\chapter{RULE 39 EXECUTION, SATISFACTION \\AND EFFECT OF JUDGMENTS}
\label{d2494470-0a16-11ef-932c-63c852f65e48}


\ssection{GR72645 June 30, 1987}
\label{d25daf70-0a19-11ef-932c-63c852f65e48}


\noindent\textbf{LUZON SURETY COMPANY, INC., Petitioner, v. \\INTERMEDIATE APPELLATE COURT, Respondents. GUTIERREZ, JR., J.}\vspace{0.4cm}

The dispute revolves around the enforcement of a judgment against Gil Puyat and whether the claim against his estate is enforceable or barred by laches due to the delay in securing a writ of execution against him during his lifetime.

In Civil Case Number 59506, judgment was rendered against the defendants, including Gil Puyat, for a sum of money. The judgment became final, but remained unenforced, thus a civil case was filed to revive the judgment in Civil Case Number 59506. Judgment was rendered in the new civil case reaffirming the 59506 judgment. Gil Puyat passed away, and on 1982 a claim against his estate was filed seeking to enforce the judgment against him.

The RTC dismissed the case, ruling that the claim against Gil Puyat's estate was unenforceable and barred by laches due to the failure to secure a writ of execution against him during his lifetime.

The Intermediate Appellate Court affirmed the decision of the lower court and dismissed the appeal. It concurred with the lower court's findings that the claim was unenforceable and barred by laches. Additionally, it cited Section 6, Rule 39 of the Rules of Court, which provides a ten-year prescriptive period for the enforcement of judgments, and concluded that the claim was not timely enforced.

\paragraph{Issue:}
\label{2315a740-1250-11ef-aa24-9916ea601717}


Whether the ten-year prescriptive period to file an action to enforce a judgment commences from the finality of the original judgment or from the finality of the revived judgment.

Whether the petitioner's claim to enforce the judgment against Gil Puyat's estate is barred by prescription.

\paragraph{Decision:}
\label{20e0c5e0-1250-11ef-aa24-9916ea601717}


After the expiration of five years from the date a judgment becomes final, it is reduced to a mere right of action, and the prescriptive period for enforcement of judgments is ten years. The ten-year prescriptive period to enforce a judgment commences from the finality of the original judgment, not from the finality of the revived judgment. The ruling in Philippine National Bank v. \\Deloso is cited, stating that the ten-year prescriptive period starts from the finality of the original judgment.

The petitioner's action to enforce the judgment against Gil Puyat's estate, filed on September 1, 1982, had already prescribed, as more than ten years had elapsed from the finality of the original judgment on April 13, 1967. That failure of the respondents to raise prescription does not imply a waiver of such defense, and all pertinent dates showing prescription can be found in the petitioner's allegations and evidence.

The petition is dismissed, and the decision of the Intermediate Appellate Court, now the Court of Appeals, is affirmed.

\ssection{GR171095 June 22, 2015}
\label{d7fce4f0-0a19-11ef-932c-63c852f65e48}


\noindent\textbf{MAYOR MARCIAL VARGAS and ENGR. RAYMUNDO DEL ROSARIO, Petitioners, v. \\FORTUNATO CAJUCOM, Respondent. PERALTA, J.}\vspace{0.4cm}

The main dispute is about the removal of illegal structures obstructing Cajucom's access to his lot in Aliaga, Nueva Ecija. Cajucom filed a mandamus action against the municipal mayor, municipal engineer, and private individuals to compel them to remove the said structures. The RTC rendered a Decision in favor of Cajucom on February 14, 2001, ordering the municipal mayor and municipal engineer to comply with the law and remove the illegal structures.

A Writ of Execution was issued by the RTC on May 11, 2001, to implement and enforce the decision. The sheriff served the Writ of Execution on Mayor Vargas and Engr. del Rosario, but the judgment was not executed. Mayor Vargas and Engr. Del Rosario filed motions and oppositions regarding the execution of the RTC's decision. Finally, the assailed order was issued by the Regional Trial Court (RTC) on September 15, 2005. In this order, the RTC denied the motion filed by Mayor Marcial Vargas and Engineer Raymundo Del Rosario to quash the writ of execution of the court's decision dated February 14, 2001. 

\paragraph{Issue:}
\label{718c07d0-1254-11ef-aa24-9916ea601717}


Whether grounds exist to quash the writ of execution issued by the RTC.

Whether the writ of execution varies the judgment or exceeds its terms.

Whether the writ is capable of enforcement.

\paragraph{Decision:}
\label{6f61fbe0-1254-11ef-aa24-9916ea601717}


The SC reiterated the principle that once a judgment becomes final and executory, a writ of execution is issued as a matter of course, in the absence of any order restraining its issuance. The issuance of a writ of execution is the trial court's ministerial duty, and it may be considered an abuse of discretion to hold it in abeyance. A final and executory judgment becomes immutable and unalterable, except for specific circumstances like correcting clerical errors or void judgments.

The writ of execution must conform substantially to every essential particular of the judgment promulgated, particularly the orders or decrees in the dispositive portion of the decision. Arguments questioning the substance and merits of the case, which have been decided with finality, cannot be raised to oppose the issuance of the writ of execution.

The enforcement of the writ of execution may be limited to circumstances and obligations stated in the judgment, and its terms must be interpreted in the context of the entire decision. If the judgment obligors fail to perform their obligations under the judgment, such as the removal of illegal structures, the demolition of said structures becomes a necessary means to enforce the decision.

The issuance of a writ of demolition, if necessary, is ancillary to the process of execution and is logically issued as a consequence of the writ of execution earlier issued.

Supreme Court dismissed the petition for lack of merit, affirmed the RTC's order, and ordered the implementation of the writ of execution with dispatch.

\ssection{GR164246 January 15, 2014}
\label{de4d9020-0a19-11ef-932c-63c852f65e48}


\noindent\textbf{ACBANG, Petitioner, v. \\HON. LUCZON, Respondents. BERSAMIN, J.}\vspace{0.4cm}

The dispute revolves around an ejectment suit filed by the Spouses Maximo and Heidi Lopez against Herminia Acbang and her co-defendants for the possession of a land covered by Transfer Certificate of Title Number T-139163. The Municipal Trial Court (MTC) ruled in favor of the Spouses Lopez, ordering the defendants to vacate the land and pay attorney’s fees. Herminia Acbang appealed to the RTC, and the Spouses Lopez moved for the execution of the decision pending appeal, alleging the failure of the defendants to file a supersedeas bond.

The RTC granted the motion for immediate execution filed by the Spouses Lopez, citing the failure of the defendants to file a supersedeas bond. Petitioner Herminia Acbang filed a motion for reconsideration, arguing that she was not notified of the filing of the motion for immediate execution. RTC denied Herminia Acbang's motion, stating that the execution of the decision is proper without a supersedeas bond.

This issues appealed at the Court of Appeals and the appelant court upheld the RTC's ruling, affirming the grant of the motion for immediate execution and the denial of Herminia Acbang's motion for reconsideration. Prompting a filing of an petition to the Suopreme Court. 

\paragraph{Issue:}
\label{b8ad10b0-1253-11ef-aa24-9916ea601717}


Whether RTC committed grave error in granting the motion for immediate execution of the Spouses Lopez without first fixing the supersedeas bond as prayed for by the Acbangs.

Whether the MTC decision in Civil Case Number 64 was null and void as far as petitioner Herminia Acbang was concerned due to lack of jurisdiction over her person.

\paragraph{Decision:}
\label{cbc16750-1253-11ef-aa24-9916ea601717}


Section 19, Rule 70 of the 1997 Rules of Civil Procedure mandates that if judgment is rendered against the defendant in an ejectment case, execution shall issue immediately upon motion unless an appeal has been perfected and the defendant files a sufficient supersedeas bond. The failure of the defendant to comply with any of the requisites for staying the immediate execution, such as failing to file a supersedeas bond, is a ground for the outright execution of the judgment. The duty of the court to issue immediate execution in case of non-compliance with the requisites is ministerial and imperative.

To stay the immediate execution of a judgment in an ejectment case pending appeal, the defendant must perfect the appeal, file a supersedeas bond, and periodically deposit the rentals becoming due during the pendency of the appeal. Filing the notice of appeal alone does not suffice to stay the immediate execution without the filing of a sufficient supersedeas bond and the deposit of accruing rentals.

The SC dismissed the petition for prohibition as moot and academic because the RTC declared the judgment of the MTC void as far as petitioner Herminia Acbang was concerned, for lack of jurisdiction over her person. The declaration of the nullity of the judgment rendered the issue in the special civil action moot and academic.

SC ruled that while the petitioner correctly argued that the Spouses Lopez should have filed a motion for execution pending appeal before the immediate execution of the judgment, the failure of the Acbangs to comply with all requisites for staying immediate execution rendered the issue moot. 

The petition was dismissed without pronouncement on costs of suit.

\ssection{GR136805 January 28, 2000}
\label{e01f9100-0a19-11ef-932c-63c852f65e48}


\noindent\textbf{DIESEL CONSTRUCTION COMPANY, INC., Petitioner, v. \\JOLLIBEE FOODS CORPORATION, Respondents. PANGANIBAN, J.}\vspace{0.4cm}

The main dispute revolves around the recovery of escalated construction costs incurred by DCCI in projects owned by JFC, and JFC's counterclaim for damages due to alleged failure by DCCI to complete the projects on time.

DCCI filed an action before the RTC of Makati City, Branch 61, seeking recovery of escalated construction costs. JFC counterclaimed for damages and attorney's fees. The pretrial conference reduced the issues to completion of projects on time and entitlement to escalated construction costs. After trial, the RTC ruled in favor of DCCI, ordering JFC to pay the escalated costs and attorney's fees.

JFC Appealed at CA, whrein the appelant court ordered immediate execution upon DCCI's posting of a bond but granted a stay of execution upon JFC's posting of a supersedeas bond. The appelant court also denied DCCI's motion for reconsideration regarding the stay of execution, stating that JFC's motion to stay execution was prompted by DCCI's own motion for execution pending appeal and rejected the accusation of forum-shopping, noting that the timing of the motions was due to procedural circumstances. This prompted DCCI to apeal at the Supreme Court.

\paragraph{Issue:}
\label{f74c84a0-1252-11ef-aa24-9916ea601717}


Whether the CA has discretionary power to stay a discretionary execution issued by the trial court under Section 3, Rule 39 of the Rules of Court.

Whether a party may be estopped from questioning the lack of authority of the appellate court to stay a discretionary execution.

Whether the pendency of an appeal is a justifiable ground to stay a discretionary execution.

Whether the trial court's holding that a motion to stay execution by posting a supersedeas bond is prematurely filed constitutes a denial thereof.

Whether the re-filing of a motion in the CA constitutes forum-shopping.

\paragraph{Decision:}
\label{f4ef5ac0-1252-11ef-aa24-9916ea601717}


Rule 45 is not the proper remedy to question interlocutory orders such as those granting a stay of execution pending appeal. The CA's resolutions are interlocutory and do not dispose of the whole subject matter of the case. The appelate court has its own separate and original discretionary jurisdiction to grant or stay execution pending appeal. The trial court's grant of execution pending appeal does not compel the CA to enforce it. The CA's authority to issue a stay of execution is not dependent on the trial court's order.

CA's order to stay execution pending appeal was based on the need to determine the respondent's liability. However, petitioner failed to show paramount and compelling reasons of urgency and justice to justify immediate execution.

On Forum shopping, the respondent's motion for a stay of execution in the CA was not forum-shopping because the trial court did not deny the motion on its merits but forwarded the case to the CA. The respondent was merely protecting its interest at the proper time and opportunity.

The Supreme Court set aside the CA's resolutions, finding no "good reasons" to grant extraordinary execution in the case. 

The Court emphasized that the alleged financial distress of the petitioner was not a compelling reason to order immediate execution, and the mere filing of a supersedeas bond by the respondent did not automatically entitle it to a stay of execution.

\ssection{GR176906 August 4, 2009}
\label{e3e12ce0-0a19-11ef-932c-63c852f65e48}


\noindent\textbf{ANDREW B. NUDO, Petitioner, v. \\HON. CAGUIOA, SPOUSES NUDO, Respondents. NACHURA, J.}\vspace{0.4cm}

The main dispute revolves around the partition of a parcel of land owned by the co-owners, brothers Petronilo and Gumersindo Nudo. Petronilo initiated a complaint for partition and damages against Gumersindo, his brother, seeking to divide the property they co-owned. Gumersindo refused to accede to the partition, leading to the filing of the complaint.

Spouses Petronilo and Marcela Nudo filed a complaint for partition and damages against the spouses, Gumersindo and Zosima Nudo, before the RTC. Gumersindo Nudo, one of the defendants, passed away during the pendency of the case. RTC rendered judgment in favor of the plaintiffs, ordering the partition of the property. Zosima Nudo, the remaining defendant, passed away. Plaintiffs filed a motion for execution, which was granted by the court on July 14, 2004. Andrew B. Nudo, son of Gumersindo and Zosima Nudo, filed a Petition for Annulment of Judgment, seeking to annul the RTC Decision in the partition case. Execution of the judgment faced difficulties due to encroachment by defendants' heirs and refusal to accept proposed partition. The issue was whether the judgment could be annulled on the ground that Andrew B. Nudo was not substituted for his deceased parents is appealed at the CA. The CA dismissed outright the petition for annulment of judgment, stating that the remedy of annulment of judgment could not be availed of since Andrew B. Nudo's predecessors-in-interest had already availed themselves of the remedy of appeal. The CA upheld its decision upon denial of petitioner's motion for reconsideration, which resulted in the escalation of the case to the Supreme Court.

\paragraph{Issue:}
\label{0efc9ff0-1252-11ef-aa24-9916ea601717}


Whether the judgment in the partition case can be annulled due to failure to implead the petitioner.

Whether lack of substitution of the deceased party's heirs constitutes lack of jurisdiction, warranting annulment of judgment.

Whether the petitioner's claim of ignorance about the case is credible.

\paragraph{Decision:}
\label{11031e50-1252-11ef-aa24-9916ea601717}


An action to annul a final judgment is an extraordinary remedy, allowable only in exceptional cases, and is limited to grounds of extrinsic fraud and lack of jurisdiction, as per Section 2, Rule 47 of the Rules of Civil Procedure. Lack of jurisdiction for annulment of judgment pertains to lack of jurisdiction over the person of the defending party or over the subject matter of the claim.

Non-substitution of heirs of a deceased party is not jurisdictional but a matter of due process, ensuring proper representation of the deceased party.

The judgment in the partition case became final and executory prior to the death of the petitioner's mother, and therefore, the petitioner cannot claim nullity of the judgment based on lack of impleadment. Enforcement of the judgment can proceed against the successors-in-interest of the deceased judgment obligor, as per Sec. 7(b), Rule 39 of the Rules of Civil Procedure. The petitioner's claim of ignorance about the case is not credible given the close relationship between the parties involved and the awareness of other family members about the judgment.

The petition is DENIED DUE COURSE. The Resolutions of the Court of Appeals are AFFIRMED.









\end{document}